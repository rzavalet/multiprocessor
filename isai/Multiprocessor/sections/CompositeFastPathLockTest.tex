% -------------------------------------------------------- %
% CompositeFastPath Lock
% by: Isai Barajas Cicourel


% -------------------------------------------------------- %
% Document Start

\section{\textbf{Composite Fast Path Lock}}


% -------------------------------------------------------- %
% Particular Caes

\subsection{Particular Case}
\par
In this experiment, we extend the \textit{CompositeLock} algorithm to encompass a fast path in which a solitary thread acquires an idle lock without acquiring a node and splicing it into the queue.
\par


% -------------------------------------------------------- %
% Solution Information

\subsection{Solution}
\par
We use a \textit{FASTPATH} flag to indicate that a thread has acquired the lock through the fast path. Because we need to manipulate this flag together with the tail field’s reference, we take a high-order bit from the tail field’s integer stamp. The private \textit{fastPathLock()} method checks whether the tail field’s stamp has a clear \textit{FASTPATH} flag and a null reference. If so, it tries to acquire the lock simply by applying \textit{compareAndSet()} to set the FASTPATH flag to true, ensuring that the reference remains null. An uncontended lock acquisition thus requires a single atomic operation. The \textit{fastPathLock()} method returns true if it succeeds, and false otherwise.
\par

% -------------------------------------------------------- %
% Experiment

\subsection{Experiment Description} 
\par
The test creates $2$ threads that need to be coordinate in order to increase a counter. All threads have to cooperate in order to increase the counter into the desire value, this lock implements a explicit lock that the test calls.
If that is not the case, an assertion fail will be raised.
\par


% -------------------------------------------------------- %
% Results

\subsection{Observations and Interpretations}

\par
The tests executed as expected and no errors where found. Since the \textit{CompositeFastPathLock} is used to acquire an explicit lock, the mutual exclusion is guarantee.

\begin{lstlisting}[frame=single,breaklines=true]
Tests run: 1, Failures: 0, Errors: 0, Skipped: 0, Time elapsed: 0.895 sec

test-single:
BUILD SUCCESSFUL (total time: 3 seconds)
\end{lstlisting}




