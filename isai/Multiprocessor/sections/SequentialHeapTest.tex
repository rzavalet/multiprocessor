% -------------------------------------------------------- %
% SequentialHeap
% by: Isai Barajas Cicourel


% -------------------------------------------------------- %
% Document Start

\section{\textbf{Sequential Heap}}


% -------------------------------------------------------- %
% Particular Caes

\subsection{Particular Case}
\par
In this experiment we implement a linearisable priority queue that supports priorities from an unbounded range. It uses fine-grained locking for synchronization.
\par


% -------------------------------------------------------- %
% Solution Information

\subsection{Solution}
\par
According to the theory a sequential heap implementation represent a binary heap by using an array of nodes, where the tree’s root is array entry $1$, and the right and left children of array entry $i$ are entries $2 · i$ and $(2 · i) + 1$, respectively.
Each node has an item and a score field. To add an item, the \textit{add()} method sets child to the index of the first empty array slot.
\par
\begin{lstlisting}[frame=single,breaklines=true]
  public void add(T item, int priority) {
    int child = next++;
    heap[child].init(item, priority);
    while (child > ROOT) {
      int parent = child / 2;
      int oldChild = child;
      if (heap[child].priority < heap[parent].priority) {
        swap(child, parent);
        child = parent;
      } else {
        return;
      }
    }
  }
\end{lstlisting}
\par
To remove and return the highest-priority item, the \textit{removeMin()} method records the root’s item, which is the highest-priority item in the tree.
\par
\begin{lstlisting}[frame=single,breaklines=true]
  public T removeMin() {
    int bottom = --next;
    T item = heap[ROOT].item;
    swap(ROOT, bottom);
    if (bottom == ROOT) {
      return item;
    }
    int child = 0;
    int parent = ROOT;
    while (parent < heap.length / 2) {
      int left = parent * 2; int right = (parent * 2) + 1;
      if (left >= next) {
        break;
      } else if (right >= next || heap[left].priority < heap[right].priority) {
        child = left;
      } else {
        child = right;
      }
      // If child higher priority than parent swap then else stop
      if (heap[child].priority < heap[parent].priority) {
        swap(parent, child); // Swap item, key, and tag of heap[i] and heap[child]
        parent = child;
      } else {
        break;
      }
    }
    return item;
  }
\end{lstlisting}

% -------------------------------------------------------- %
% Experiment

\subsection{Experiment Description} 
\par
The test adds $512$ elements to the heap, does a check of the elements and removes the values with min priority. We verify the priorities of the removed elements and assert a fail if we have non ascending priorities.
\par


% -------------------------------------------------------- %
% Results

\subsection{Observations and Interpretations}

\par
The test execute as expected and no problems are thrown.
\begin{lstlisting}[frame=single,breaklines=true]
Tests run: 1, Failures: 0, Errors: 0, Skipped: 0, Time elapsed: 0.413 sec

------------- Standard Output ---------------
testSequential
OK.
------------- ---------------- ---------------
test-single:
BUILD SUCCESSFUL (total time: 1 second)

\end{lstlisting}

