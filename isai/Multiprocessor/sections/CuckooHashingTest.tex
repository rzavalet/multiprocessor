% -------------------------------------------------------- %
% Cuckoo Hashing
% by: Isai Barajas Cicourel


% -------------------------------------------------------- %
% Document Start

\section{\textbf{Cuckoo Hashing}}


% -------------------------------------------------------- %
% Particular Caes

\subsection{Particular Case}
\par
Cuckoo hashing is a (sequential) hashing algorithm in which a newly added item displaces any earlier item occupying the same slot. 
In this test we modify the sequential Cuckoo hashing in order to change the sequential hashing algorithm to concurrent hashing.
\par


% -------------------------------------------------------- %
% Solution Information

\subsection{Solution}
\par
We break up each method call into a sequence of phases, where each phase adds, removes, or displaces a single item $x$.
We use a two-entry array \textit{table[]} of tables, and two independent hash functions, (denoted as \textit{hash0()} and \textit{hash1()} in the code) mapping the set of possible keys to entries in the array. 
\par
\begin{lstlisting}[frame=single,breaklines=true]
  public TCuckooHashSet(int capacity) {
    locks = new Lock[2][LOCKS];
    table = (T[][]) new Object[2][capacity];
    size = capacity;
    for (int i = 0; i < 2; i++) {
      for (int j = 0; j < LOCKS; j++) {
        locks[i][j] = new ReentrantLock();
      }
    }
  }
  private final int hash0(Object x) {
    return Math.abs(x.hashCode() % size);
  }
  private final int hash1(Object x) {
    random.setSeed(x.hashCode());
    return random.nextInt(size);
  }
\end{lstlisting}
\par
To test whether a value $x$ is in the set, \textit{contains(x)} tests whether either table[0][h0(x)] or $table[1][h1(x)]$ is equal to $x$. Similarly, \textit{remove(x)} checks whether $x$ is in either $table[0][h0(x)]$ or $table[1][h1(x)]$, and removes it if found.
\begin{lstlisting}[frame=single,breaklines=true]
  public boolean remove(T x) {
    if (x == null) {
      throw new IllegalArgumentException();
    }
    int h0 = hash0(x);
    Lock lock0 = locks[0][h0 % LOCKS];
    try {
      lock0.lock();
      if (x.equals(table[0][h0])) {
        table[0][h0] = null;
        return true;
      } else {
        int h1 = hash1(x);
        Lock lock1 = locks[1][h1 % LOCKS];
        try {
          lock1.lock();
          if (x.equals(table[1][h1])) {
            table[1][h1] = null;
            return true;
          }
          return false;
        } finally {
          lock1.unlock();
        }
      }
    } finally {
      lock0.unlock();
    }
  }
\end{lstlisting}
\par
The \textit{add(x)} method is the most interesting. It successively removes conflicting items until every key has a slot. To add $x$, the method swaps $x$ with $y$, the current occupant of $table[0][h0(x)]$. If the prior value $y$ was null, it is done, otherwise, it swaps the newly nestles value $y$ for the current occupant of $table[1][h1(y)]$ in the same way and continues swapping entries (alternating tables) until it finds an empty slot.


% -------------------------------------------------------- %
% Experiment

\subsection{Experiment Description} 
\par
The test creates $8$ threads that perform \textit{enq()} and \textit{deq()} operations in sequential and parallel.
The dequeue values are revised and if a value is missing a fail assertion is thrown.
\par

% -------------------------------------------------------- %
% Results

\subsection{Observations and Interpretations}

\par
When executing the test the test fails on the parallel part due to a missing value when executing the parallel both test.
\begin{lstlisting}[frame=single,breaklines=true]
Exception in thread "Thread-5" Exception in thread "Thread-7" junit.framework.AssertionFailedError: Th 0	DeqThread: missing value: 3
\end{lstlisting}


% -------------------------------------------------------- %
% Results

\subsection{Proposed changes to fix the problem}

\par
By adding terminal output through System.out in the \textit{remove()} method or Debugging the file the execution works, thus becoming a problem of observation, since the fail assertion only happens when its not observed.
\begin{lstlisting}[frame=single,breaklines=true]
compile-test:
.sequential add, contains, and remove
.parallel add
.parallel remove
.parallel both

Time: 0.071

OK (4 tests)
\end{lstlisting}



% -------------------------------------------------------- %
% Why?

\subsection{Proposed solution}
\par
It is not clear why the observation would affect the execution, thus it is possible that the issue is a propagation of the values in the memory which are forced to be executed constantly through the debug implementation.

