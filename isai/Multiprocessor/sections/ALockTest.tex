% -------------------------------------------------------- %
% ALock
% by: Isai Barajas Cicourel


% -------------------------------------------------------- %
% Document Start

\section{\textbf{Array Lock}}


% -------------------------------------------------------- %
% Particular Caes

\subsection{Particular Case}
\par
The ALock, is a simple array-based queue lock. In a queue, each thread can learn if its turn has arrived by checking whether its predecessor has finished.
\par


% -------------------------------------------------------- %
% Solution Information

\subsection{Solution}
\par
The threads share an \textit{AtomicInteger} tail field, initially zero. To acquire the lock, each thread atomically increments tail. Call the resulting value the thread’s slot. The slot is used as an index into a Boolean flag array. If \textit{flag[j]} is true, then the thread with slot \textit{j} has permission to acquire the lock.
\begin{lstlisting}[frame=single,breaklines=true]
public class ALock implements Lock {
  // thread-local variable
  ThreadLocal<Integer> mySlotIndex = new ThreadLocal<Integer> (){
    protected Integer initialValue() {
      return 0;
    }
  };
  volatile AtomicInteger tail;
  volatile boolean[] flag;
  int size;
  /**
   * Constructor
   * @param capacity max number of array slots
   */
  public ALock(int capacity) {
    size = capacity;
    tail = new AtomicInteger(0);
    flag = new boolean[capacity];
    flag[0] = true;
  }
  public void lock() {
    int slot = tail.getAndIncrement() % size;
    mySlotIndex.set(slot);
    while (! flag[mySlotIndex.get()]) {}; // spin
  }
  public void unlock() {
    flag[mySlotIndex.get()] = false;
    flag[(mySlotIndex.get() + 1) % size] = true;
  }
  // any class implementing Lock must provide these methods
  public Condition newCondition() {
    throw new java.lang.UnsupportedOperationException();
  }
  public boolean tryLock(long time,
      TimeUnit unit)
      throws InterruptedException {
    throw new java.lang.UnsupportedOperationException();
  }
  public boolean tryLock() {
    throw new java.lang.UnsupportedOperationException();
  }
  public void lockInterruptibly() throws InterruptedException {
    throw new java.lang.UnsupportedOperationException();
  }
}
\end{lstlisting}
\par


% -------------------------------------------------------- %
% Experiment

\subsection{Experiment Description} 
\par
The test creates $8$ threads that need to be coordinate in order to increment a common counter. All threads have to cooperate to increase this counter from 0 to $1024$. Each of the threads will increase by one the counter $128$ times.
The expected result is that regardless of the order in which each thread executes the increment, at the end the counter must stay at $1024$.
If that is not the case, then it means that mutual exclusion did not work.
\par

% -------------------------------------------------------- %
% Results

\subsection{Observations and Interpretations}

\par
When executing the test values where lost from the queue.
\begin{lstlisting}[frame=single,breaklines=true]
1009
1010
1011
1013
1014
1015
\end{lstlisting}


% -------------------------------------------------------- %
% Results

\subsection{Proposed changes to fix the problem}

\par
By adding the volatile keyword in the ALock class the execution was fixed.
\begin{lstlisting}[frame=single,breaklines=true]
  volatile AtomicInteger tail;
  volatile boolean[] flag;
  int size;
\end{lstlisting}
Once the change is made the execution works fine on every equipment used.


% -------------------------------------------------------- %
% Why?

\subsection{Proposed solution}
\par
Apparently the AtomicInteger keyword is not guarantee that the values are propagated through the memory hierarchy and by making the value atomic the execution works.

