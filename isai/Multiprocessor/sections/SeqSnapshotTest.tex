% -------------------------------------------------------- %
% SeqSnapshot
% by: Isai Barajas Cicourel


% -------------------------------------------------------- %
% Document Start

\section{\textbf{Sequential Snapshot}}


% -------------------------------------------------------- %
% Particular Caes

\subsection{Particular Case}
\par
In order to read multiple register values atomically we use an atomic snapshot.
An atomic snapshot constructs an instantaneous view of an array of atomic registers. We construct a wait-free snapshot, meaning that a thread can take an instantaneous snapshot of memory without delaying any other thread.
\par


% -------------------------------------------------------- %
% Solution Information

\subsection{Solution}
\par
Each thread repeatedly calls \textit{collect()}, and returns as soon as it detects a clean double collect (one in which both sets of timestamps were identical). This construction always returns correct values. The \textit{update()} calls are wait-free, but \textit{scan()} is not because any call can be repeatedly interrupted by \textit{update()}, and may run forever without completing. It is however obstruction-free, since a \textit{scan()} completes if it runs by itself for long enough.
\begin{lstlisting}[frame=single,breaklines=true]
public class SeqSnapshot<T> implements Snapshot<T> {
  private T[] a_value;
  public SeqSnapshot(int capacity, T init) {
    a_value = (T[])new Object[capacity];
    for (int i = 0; i < a_value.length; i++) {
      a_value[i] = init;
    }
  }
  public synchronized void update(T v) {
    a_value[ThreadID.get()] = v;
  }
  public synchronized T[] scan() {
    T[] result = (T[]) new Object[a_value.length];
    for (int i = 0; i < a_value.length; i++)
      result[i] = a_value[i];
    return result;
  }
}
\end{lstlisting}
\par


% -------------------------------------------------------- %
% Experiment

\subsection{Experiment Description} 
\par
The test creates $2$ threads that write the register twice with the the values \textit{FIRST} and \textit{SECOND} and later parallel reads the values.
\par

% -------------------------------------------------------- %
% Results

\subsection{Observations and Interpretations}

\par
When executing the test the test fails on the parallel part due to a index out of bound exception.
\begin{lstlisting}[frame=single,breaklines=true]
Testcase: testSequential(register.SeqSnapshotTest):	Caused an ERROR
2
java.lang.ArrayIndexOutOfBoundsException: 2
	at register.SeqSnapshot.update(SeqSnapshot.java:26)
	at register.SeqSnapshotTest.testSequential(SeqSnapshotTest.java:40)
\end{lstlisting}


% -------------------------------------------------------- %
% Results

\subsection{Proposed changes to fix the problem}

\par
By resetting the thread id at the beginning of the test the id is guarantee to be the experted while doing the parallel test.
\begin{lstlisting}[frame=single,breaklines=true]
  class MyThread extends Thread {
    public void run() {
      ThreadID.reset();
      instance.update(FIRST);
      instance.update(SECOND);
      Object[] a = instance.scan();
      for (Object x : a) {
        Integer i = (Integer) x;
        int me = ThreadID.get();
        for (int j = 0; j < THREADS; j++) {
          results[me][j] = (Integer)a[j];
        }
      }
    }
  }
\end{lstlisting}
Once the change is made the execution works fine on every equipment used.
\begin{lstlisting}[frame=single,breaklines=true]
Tests run: 2, Failures: 0, Errors: 0, Skipped: 0, Time elapsed: 0.097 sec

------------- Standard Output ---------------
sequential
parallel
------------- ---------------- ---------------
test-single:
BUILD SUCCESSFUL (total time: 0 seconds)
\end{lstlisting}


% -------------------------------------------------------- %
% Why?

\subsection{Proposed solution}
\par
Once the ThreadID is reset before changing from the sequential to the parallel test the execution never fails, since the out of bound happens due to the id being incremental.

