% -------------------------------------------------------- %
% RevBarrierTes 
% by: Isai Barajas Cicourel


% -------------------------------------------------------- %
% Document Start

\section{\textbf{Reusable Barrier}}


% -------------------------------------------------------- %
% Particular Caes

\subsection{Particular Case}
\par
In this experiment we implement an n-thread reusable barrier from an n-wire counting network and a single Boolean variable.
\par


% -------------------------------------------------------- %
% Solution Information

\subsection{Solution}
\par
To implement a reusable barrier, we use a Combining Tree Barrier implementation and change the constructor in order to support the reuse of the barrier. 
\par
\begin{lstlisting}[frame=single,breaklines=true]
  public RevBarrier(int _size, int _radix) {
    radix = _radix;
    leaves = 0;
    this.leaf = new Node[_size / _radix];
    int depth = 0;
    threadSense = new ThreadLocal<Boolean>() {
      protected Boolean initialValue() { return true; };
    };
    // compute tree depth
    while (_size > 1) {
      depth++;
      _size = _size / _radix;
      assert _size > 0;
    }
    Node root = new Node();
    build(root, depth - 1);
  }
  
  // recursive tree constructor
  void build(Node parent, int depth) {
    // are we at a leaf node?
    if (depth == 0) {
      leaf[leaves++] = parent;
    } else {
      for (int i = 0; i < radix; i++) {
        Node child = new Node(parent);
        build(child, depth - 1);
      }
    }
  }
\end{lstlisting}


% -------------------------------------------------------- %
% Experiment

\subsection{Experiment Description} 
\par
The test creates $8$ threads that need to use the barrier \textit{await()} in order to write values into the log and let the writer finish. 
If the round doesn't correspond to the value in the log then a System.out showing the expected value and found are displayed.
\par


% -------------------------------------------------------- %
% Results

\subsection{Observations and Interpretations}

\par
The tests executed as expected and no errors where found.
\begin{lstlisting}[frame=single,breaklines=true]
Tests run: 1, Failures: 0, Errors: 0, Skipped: 0, Time elapsed: 0.511 sec

------------- Standard Output ---------------
Testing 8 threads, 8 rounds
------------- ---------------- ---------------
test-single:
BUILD SUCCESSFUL (total time: 0 seconds)
\end{lstlisting}




