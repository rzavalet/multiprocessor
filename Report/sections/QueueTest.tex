% -------------------------------------------------------- %
% Queue Lock
% by: Isai Barajas Cicourel


% -------------------------------------------------------- %
% Document Start

\section{\textbf{Queue Lock}}


% -------------------------------------------------------- %
% Particular Caes

\subsection{Particular Case}
\par
In this experiment, the particular case we are trying to study is mutual exclusion where several threads try to use a queue.
\par
We are trying to do so using bounded partial queue.
\par


% -------------------------------------------------------- %
% Solution Information

\subsection{Solution}
\par
According to the theory the \textit{enq()} and \textit{deq()} methods operate on opposite ends of the queue, so as long as the queue is neither full nor empty, an \textit{enq()} call and a \textit{deq()} call should, in principle, be able to proceed without interference, guarantee mutual exclusion for the methods.
\par
Here, we implement a bounded queue as a linked list of entries as shown by the Entry Class.
\par
\begin{lstlisting}[frame=single,breaklines=true]
  /**
   * Individual queue item.
   */
  protected class Entry {
    /**
     * Actual value of queue item.
     */
    public T value;
    /**
     * next item in queue
     */
    public Entry next;
    /**
     * Constructor
     * @param x Value of item.
     */
    public Entry(T x) {
      value = x;
      next = null;
    }
  }
\end{lstlisting}

% -------------------------------------------------------- %
% Experiment

\subsection{Experiment Description} 
\par
The test creates $8$ threads that need to be coordinate in order to \textit{enq()} and \textit{deq()} a range of numbers. All threads have to cooperate to add and remove elements from the queue. Each of the threads will enqueue and dequeue values into the queue, if everything works according to the test there will be mutual exclusion and the mapping of elements will corresponds to the queue elements.
If that is not the case, a duplicate fail will be raised.
\par


% -------------------------------------------------------- %
% Results

\subsection{Observations and Interpretations}

\par
The tests executed as expected and no errors where found. Since the ReentrantLock is used to acquire an explicit lock, the mutual exclusion is guarantee by the Java java.util.concurrent.locks package.

\begin{lstlisting}[frame=single,breaklines=true]
  /**
   * Lock out other enqueuers (dequeuers)
   */
  ReentrantLock enqLock, deqLock;
\end{lstlisting}




