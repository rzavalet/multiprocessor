% -------------------------------------------------------- %
% DEQueue Lock
% by: Isai Barajas Cicourel


% -------------------------------------------------------- %
% Document Start

\section{\textbf{Double Ended Queue}}


% -------------------------------------------------------- %
% Particular Caes

\subsection{Particular Case}
\par
In this experiment each thread keeps a pool of tasks waiting to be executed in the form of a double-ended queue (DEQueue), providing \textit{pushBottom()}, \textit{popBottom()}, and \textit{popTop()} methods.
\par


% -------------------------------------------------------- %
% Solution Information

\subsection{Solution}
\par
According to the theory when a thread creates a new task, it calls\textit{ pushBottom()} to push that task onto its \textit{DEQueue}. When a thread needs a task to work on, it calls \textit{popBottom()} to remove a task from its own \textit{DEQueue}. If the thread discovers its queue is empty, then it becomes a thief, it chooses a victim thread at random, and calls that thread’s DEQueue’s \textit{popTop()} method to steal a task for itself.
\par
\begin{lstlisting}[frame=single,breaklines=true]
  public Runnable popTop() {
    int[] stamp = new int[1];
    int oldTop = top.get(stamp), newTop = oldTop + 1;
    int oldStamp = stamp[0], newStamp = oldStamp + 1;
    if (bottom <= oldTop) // empty
      return null;
    Runnable r= tasks[oldTop];
    if (top.compareAndSet(oldTop, newTop, oldStamp, newStamp))
      return r;
    return null;
  }
\end{lstlisting}

% -------------------------------------------------------- %
% Experiment

\subsection{Experiment Description} 
\par
The test creates $16$ threads, eight that need to be coordinate in order to \textit{pushBottom()} and \textit{popBottom()} an array of values. All threads have to cooperate to add and remove elements from the queue and eight \textit{Dummy()} threads will be stealing execution from the running ones. After each thread execution, if everything works according to the test the map will have all values set to true.
If that is not the case, a duplicate or missing fail will be raised.
\par


% -------------------------------------------------------- %
% Results

\subsection{Observations and Interpretations}

\par
The tests executed as expected and no errors where found.
\begin{lstlisting}[frame=single,breaklines=true]
Tests run: 2, Failures: 0, Errors: 0, Skipped: 0, Time elapsed: 1.162 sec

------------- Standard Output ---------------
sequential pushBottom and popBottom
concurrent pushBottom and popBottom
------------- ---------------- ---------------
test-single:
BUILD SUCCESSFUL (total time: 2 seconds)
\end{lstlisting}




