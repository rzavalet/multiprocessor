\section{\textbf{ListTest (ofree)}}

\subsection{Particular Case (problem)}
The problem we try to solve here is how to implement a concurrent list
by using software transactional (\C{STM}) memory.

\subsection{Solution}
The solution uses the ``obstruction-free'' flavor of the \C{STM}, where
the building blocks of the list use the so called \C{FreeObject}. Each
one of these free objects has three logical fields: an \C{owner} field
to keep track of what thread is using it, \C{oldVersion} which is the
object state prior beginning of transaction, and \C{newVersion}
version that reflects current transaction updates. The possible values
for \C{oldVersion} and \C{newVersion} fields are \C{COMMITTED},
\C{ABORTED} and \C{ABORTED}. \\ 

Each time a transaction \C{A} accesses one of these objects for the
first time, it ``opens'' that object (possibly resetting the logical
fields mentioned above). If we call \C{B} the previous owner
transaction, the following rules govern this solution (quoted from the
book): \\

\begin{itemize}
  \item If \C{B} was \C{COMMITTED}, then the new version is current. A
installs itself as the object's current owner, sets the old version to
the prior new version, and the new version to a copy of the prior new
version (if the call is a setter), or to the new version itself (if
the call is a getter). \\  

 \item Symmetrically, if \C{B} was \C{ABORTED}, then the old version is 
current. A installs itself as the object’s current owner, sets the old
version to the prior old version, and the new version to a copy of the
prior old version (if the call is a setter), or to the old version
itself (if the call is a getter).  \\

  \item If \C{B} is still \C{ACTIVE}, then \C{A} and \C{B} conflict, so 
\C{A} consults the contention manager for advice whether to abort
\C{B}, or to pause, giving \C{B} a chance to finish. One transaction
aborts another by successfully calling \C{compareAndSet()} to change the
victim's status to \C{ABORTED}. 
\end{itemize}
\hfill

Contrary to other chapters, the transactional memory one implies much
more code to consider; so we did not consider relevant to paste it all
here. The explanation given above hopefully suffices for our purposes.

\subsection{Experiment Description}
The test program populates serially the list protected by STM (initial
size of 1024), and then creates certain amount of threads to perform
indefinitely, alternating random insertions and deletions. The main
test thread interrupts them and performs sanity checks. This test is
performed with 1 (sequential) and 32 threads (parallel).

\subsection{Observations and Interpretations}
The test using a single thread (sequential), gives the following null
pointer exception on the 24 cores machine: \\

\begin{verbatim}
.TestSequential
TinyTM.exceptions.PanicException: java.lang.NullPointerException
at TinyTM.TThread.doIt(TThread.java:51)
at TinyTM.list.ofree.ListTest$TestThread.run(ListTest.java:117)
Caused by: java.lang.NullPointerException
at TinyTM.list.ofree.TNode.getNext(TNode.java:54)
at TinyTM.list.ofree.List.add(List.java:40)
at TinyTM.list.ofree.ListTest$TestThread$1.call(ListTest.java:119)
at TinyTM.list.ofree.ListTest$TestThread$1.call(ListTest.java:117)
at TinyTM.TThread.doIt(TThread.java:41)
... 1 more
inserts: 0, removes: 0, missed: 0, delta: 0
commits: 0, aborts: 0
\end{verbatim}
\hfill

While debugging the sequential exception above, we found that in method
\C{FreeObject:openRead}, a new location was never getting a
\C{newVersion} value after its construction: \\

\begin{lstlisting}[style=nonumbers]
Locator newLocator = new Locator();
... more code ...
return newLocator.newVersion <<== NULL !!!
\end{lstlisting}
\hfill

As possible solutions for this problem, we realized that either we
could use constructor that takes \C{newVersion}  as argument or we set
it like with method \C{openWrite}. We opted for the first option, using
existing \C{locator} object: \\

\begin{lstlisting}[style=nonumbers]
Locator newLocator = new Locator(locator.newVersion);
\end{lstlisting}
\hfill

With above fix the sequential test  no longer presented problems on 2
nor on 24 cores. But the parallel one was still failing with assertion
errors like the ones below: \\

\begin{verbatim}
Testcase: testParallel(TinyTM.list.ofree.ListTest): FAILED
Bad size, expected:<1518> but was:<1502>
junit.framework.AssertionFailedError: Bad size, expected:<1518> but
                was:<1502>
at TinyTM.list.ofree.ListTest.sanityCheck(ListTest.java:196)
at TinyTM.list.ofree.ListTest.testParallel(ListTest.java:84)
\end{verbatim}
\hfill

Due time constraints, we could not dig further into the reasons for
the above error; but professor acknowledged it was ok (after all, we
covered the entire set of programs from the book).






