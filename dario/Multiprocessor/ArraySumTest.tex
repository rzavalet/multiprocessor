\section{\textbf{ArraySumTest}}

\subsection{Particular Case (problem)}
The problem we want to solve is that of summing up all the elements of
an array; but of course, we want to do it in parallel.

\subsection{Solution}
The solution uses the divide and conquer approach, creating a tree of
tasks (\C{Future} from java concurrency library); which represent
the processing we do while splitting the arrays in two halves and summing up
each half recursively. Base case are arrays of a single element. Below
the heart of the solution, the recursive function which create new tasks
for subarrays: \\

\begin{lstlisting}[style=numbers]
    public Integer call() throws InterruptedException, ExecutionException {
      if (size == 1) {
        return a[start];
      } else {
        int lhsSize = size / 2;
        int rhsStart = lhsSize + 1;
        int rhsSize = size - lhsSize;
        Future<Integer> lhs = exec.submit(new SumTask(a, start, lhsSize));
        Future<Integer> rhs = exec.submit(new SumTask(a, rhsStart, rhsSize));
        return rhs.get() + lhs.get();
      }
    }
\end{lstlisting}
\hfill

\subsection{Experiment Description}
The test merely consists in trying the algorithm with a couple of
arrays of consecutive positive integers, of respective sizes 3 and
4. The test validates that the expected sum is 6 and 10 respectively.

\subsection{Observations and Interpretations}
The test presents failures like the one below, where the array of 3
elements gave as sum 7 instead of 6: \\

\begin{verbatim}
[oraadm@gdlaa008 orig]$ junit steal.ArraySumTest
.F
Time: 0.009
There was 1 failure:
1) testRun(steal.ArraySumTest)junit.framework.AssertionFailedError:
                expected:<6> but was:<7>
at steal.ArraySumTest.testRun(ArraySumTest.java:31)
at sun.reflect.NativeMethodAccessorImpl.invoke0(Native Method)
at
                
sun.reflect.NativeMethodAccessorImpl.invoke(NativeMethodAccessorImpl.java:57)
at
                
sun.reflect.DelegatingMethodAccessorImpl.invoke(DelegatingMethodAccessorImpl.java:43)

FAILURES!!!
Tests run: 1,  Failures: 1,  Errors: 0
\end{verbatim}
\hfill

The problem lied on the program \C{ArraySum.java}, on the assignment
of variable \C{rhsStart}, which represents the starting point of the
right subarray. While the original program had $lhsSize + 1$
as assignment, the proper expression was $lhsSize + start$ (so we take
into account both the relative start of the initial subarray we got,
as well as the length of the left sub-subarray). After this fix, the test
worked fine on both 2 and 24 cores machines.



