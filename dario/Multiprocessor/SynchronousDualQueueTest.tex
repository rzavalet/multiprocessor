\section{\textbf{SynchronousDualQueueTest}}

\subsection{Particular Case (problem)}
The problem is to reduce the synchronization overhead of a synchronous
queue.

\subsection{Solution}
The solution is given by using what is called a dual data structure;
which splits the \C{enq} and \C{deq} operations in two parts.  
When a dequeuer tries to
remove an item from the queue, it inserts a reservation object
indicating that is waiting for an enqueuer. Later when an enqueuer
thread realizes about the reservation, it fulfills the same by
depositing an item and setting the reservation's flag. On the same
way, an enqueuer thread can make another reservation when it wants to
add an item and spin on its reservation's flag. \\

This solution has some nice properties: \\

\begin{itemize}
  \item Waiting threads can spin on a locally cached flag
    (scalability).
  \item Ensures fairness in a natural way. Reservations are queued in
    the order they arrive.
  \item This data structure is linearizable, since each partial method
    call can be ordered when it is fulfilled.
\end{itemize}

\subsection{Experiment Description}
The test creates 16 threads, grouping them on enqueuers and dequeuers;
for each group it divides the worlkoad (512) evenly. Then each thread
proceeds to either enqueue or dequeue, as many times as its share of
the workload indicated (512/8). There are some assertions to ensure
that we do not repeat values (each dequeuer marks an array at the
index of the value it got). 
among them. 

\subsection{Observations and Interpretations}
The test runs normally most of the times, but in some occasions it
hangs (in the test machine with two cores is a rare error, but on the
one with 24 cores it almost always occurs) . During those cases we can
see a null pointer exception, and some never ending enqueuers:

\begin{verbatim}
.parallel both
Exception in thread "Thread-9" java.lang.NullPointerException
	at queue.SynchronousDualQueueTest$DeqThread.run(SynchronousDualQueueTest.java:67)
2015-10-18 23:20:41
Full thread dump Java HotSpot(TM) 64-Bit Server VM (25.60-b23 mixed mode):

"Thread-12" #21 prio=5 os_prio=0 tid=0x00007f4314112800 nid=0x75cf runnable [0x00007f42fc980000]
   java.lang.Thread.State: RUNNABLE
	at queue.SynchronousDualQueue.enq(SynchronousDualQueue.java:49)
	at queue.SynchronousDualQueueTest$EnqThread.run(SynchronousDualQueueTest.java:60)

"Thread-8" #17 prio=5 os_prio=0 tid=0x00007f431410e800 nid=0x75cd runnable [0x00007f42fcb82000]
   java.lang.Thread.State: RUNNABLE
	at queue.SynchronousDualQueue.enq(SynchronousDualQueue.java:49)
	at queue.SynchronousDualQueueTest$EnqThread.run(SynchronousDualQueueTest.java:60)
\end{verbatim}
\hfill

The \C{NullPointerException} and the hanging enqueuers are related;
the former appears cause the \C{deq} method returns null hence it
fails to cast into integer for a dequeuer thread (leaving the overall
situation unbalanced, as now nobody will consume some of the items
being pushed into the queue). \\

The queue is populated with integers, so the \C{null} value returned
must be either a defect of the tested class \C{SynchronousDualQueue};
or that the test case is badly designed. On either case, is shall be
related with concurrency, as error occurs much more often on the machine
with 24 cores. The easiest way to handle this is to modify the test to
handle \C{null} values on dequeuers thread class: \\

\begin{lstlisting}[style=nonumbers]
  class DeqThread extends Thread {
    public void run() {
      for (int i = 0; i < PER_THREAD; i++) {
        Object v = null;
        while ( (v = instance.deq()) == null ); 
        int value = (Integer) v;
        if (map[value]) {
          fail("DeqThread: duplicate pop");
        }
        map[value] = true;
      }
    }
  }  
\end{lstlisting}
\hfill

With above modification, the test does not hang anymore (tried in both
test machines, with 2 and 24 cores). 

