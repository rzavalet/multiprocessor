\section{\textbf{LazyListTest}}

\subsection{Particular Case (problem)}
The problem is that of gradually improving what coarse grain locking
offers for concurrent data structures like sets (implemented with
linked lists).

\subsection{Solution}
The \C{LazyList} solution is a refinement of the \C{OptimisticList}
solution which does not lock while searching, but locks one it finds
the interesting nodes (and then confirms that the locked nodes are
correct). As one drawback of \C{OptimisticList} is that the most
common method \C{contains} method locks, the next logical improvement
is to make this method wait-free while keeping the \C{add} and
\C{remove} methods locking (but reducing their transversings of the
list from two to just one). \\

The refinement mentioned above is precisely that of \C{LazyList},
which adds a new bit to each node to indicate whether they still
belong to the set or not (this prevents transvering the list to detect
if the node is reachable, as the new bit introduces such invariant: if
a transversing thread does not find a node or it is marked in this
bit, then the corresponding item does not belong to the set. This
behavior implies that \C{contains} method does a single wait-free
transversal of the list. \\

For adding an element to the list, \C{add} method traverses the list,
locks the target's predecessor and successor nodes, and inserts the
new node in between. The \C{remove}   
method is lazy (hence the name of the solution), as it splits its task in
two parts: first marks the node in the new bit, logically removing it;
and second, update its predecessor's next field, physically removing
it. \\

The three methods ignore the locks while transversing the list,
possibly passing over both logically and physically deleted nodes. The
\C{add} and \C{remove} methods still lock \C{pred} and \C{curr} nodes
as with \C{OptimisticList} solution, but the validation reduces to
check that \C{curr} node has not been marked; as well as validating
the same for \C{pred} node, and that it still points to \C{curr}
(validating a couple of nodes is much better than transversing whole
list though). Finally, the introduction of logical removals implies a
new contract for detecting that an item still belongs to set: it does
so, if still referred by an unmarked reachable node. \\

The most relevant methods, \C{remove} and \C{contains}, are listed
below: \\

\begin{lstlisting}[style=numbers]
  public boolean remove(T item) {
    int key = item.hashCode();
    while (true) {
      Node pred = this.head;
      Node curr = head.next;
      while (curr.key < key) {
        pred = curr; curr = curr.next;
      }
      pred.lock();
      try {
        curr.lock();
        try {
          if (validate(pred, curr)) {
            if (curr.key != key) {    // present
              return false;
            } else {                  // absent
              curr.marked = true;     // logically remove
              pred.next = curr.next;  // physically remove
              return true;
            }
          }
        } finally {                   // always unlock curr
          curr.unlock();
        }
      } finally {                     // always unlock pred
        pred.unlock();
      }
    }
  }

  public boolean contains(T item) {
    int key = item.hashCode();
    Node curr = this.head;
    while (curr.key < key)
      curr = curr.next;
    return curr.key == key && !curr.marked;
  }
\end{lstlisting}
\hfill

\subsection{Experiment Description}
The test is pretty much the same described for \C{LockFreeQueueTest},
with a few differences.

\begin{itemize}
\item The data structure here is a set, rather than a queue.
\item The exception that it uses 8 threads instead of two for each
  operation (\C{add} / \C{remove}).
\item The threads that remove elements do not care only in
  successfully removing certain number of times (like with the queue);
  here they expect to remove a particular subset of the values.
\end{itemize}
  

\subsection{Observations and Interpretations}

The test works as expected on a two cores machine, sample output below:

\begin{verbatim}
.parallel deq
.parallel both
.sequential push and pop
.parallel enq

Time: 0.03

OK (4 tests)
\end{verbatim}
\hfill

Interestingly, on a 24 cores machine, sometimes the test case
\C{testParallelBoth} fails with exceptions like the one below: \\

\begin{verbatim}
junit.framework.AssertionFailedError: RemoveThread: duplicate remove
at junit.framework.Assert.fail(Assert.java:57)
at junit.framework.TestCase.fail(TestCase.java:227)
at lists.LazyListTest$RemoveThread.run(LazyListTest.java:142)
\end{verbatim}
\hfill

While debugging the error above, we found that the message of
``duplicate remove'' is a bit misleading; is not really that someone
else tried to delete that value (as each \C{RemoveThread} cares about
a unique set of values). The real problem is that the removing threads
just try once to remove each value, and fail if they did not find any
of them. Since both the adder and remover threads are started
concurrently, there is no guarantee that the adders will come first
than the removers; so it could be that the removers try to pull out
something that has not been inserted yet (leaving to the exception
shown above). \\

Since the \C{LazyList} solution does not include an error-and-retry
approach (as with our second rewrite of the \C{LockFreeQueue}, which
used Java condition's await methods), the only way to fix this would
be to rewrite the test program itself. Each remover thread will need
to indefinitely try to remove all its items until completion, rather
than expecting that all of them are available by the time they are to
be removed. We tried that approach and made the proper adjustments
to the test program, after which the problem got solved as expected.
