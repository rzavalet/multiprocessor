\section{\textbf{LazyListTest}}

\subsection{Particular Case (problem)}
The problem is that of gradually improving what coarse grain locking
offers for concurrent data structures like sets (implemented with
linked lists).

\subsection{Solution}
The \C{LazyList} solution is a refinement of the \C{OptimisticList}
solution which does not lock while searching, but locks one it finds
the interesting nodes (and then confirms that the locked nodes are
correct). As one drawback of \C{OptimisticList} is that the most
common method \C{contains} method locks, the next logical improvement
is to make this method wait-free while keeping the \C{add} and
\C{remove} methods locking (but reducing their transversings of the
list from two to just one). \\

The refinement mentioned above is precisely that of \C{LazyList},
which adds a new bit to each node to indicate whether they still
belong to the set or not (this prevents transvering the list to detect
if the node is reachable, as the new bit introduces such invariant: if
a transversing thread does not find a node or it is marked in this
bit, then the corresponding item does not belong to the set. This
behaviour implies that \C{contains} method does a single wait-free
transversal of the list. \\

For adding an element to the list, \C{add} method traverses the list,
locks the target's predecessor, and inserts the node. The \C{remove} 
method is lazy (hece the name of the solution), as it splits its task in
two parts: first marks the node in the new bit, logically removing it;
and second, update its predecessor's next field, physically removing
it. \\

The three methods ignore the locks while transversing the list,
possibly passing over both logically and physically deleted nodes. The
\C{add} and \C{remove} methods still lock \C{pred} and \C{curr} nodes
as with \C{OptimisticList} solution, but the validation reduces to
check that \C{curr} node has not been marked; as well as validating
the same for \C{pred} node, and that it still points to \C{curr}
(validating a couple of nodes is much better than transversing whole
list though). Finally, the introduction of logical removals implies a
new contract for detecting that an item still belongs to set: it does
so, if still referred by an unmarked reachable node. 


\subsection{Experiment Description}
The test is pretty much the same described for \C{LockFreeQueueTest}
(with the exceptiont that it uses 8 threads instead of two).

\subsection{Observations and Interpretations}

The test works as expected, below sample execution:

\begin{verbatim}
.parallel deq
.parallel both
.sequential push and pop
.parallel enq

Time: 0.03

OK (4 tests)
\end{verbatim}
\hfill

