\newpage
\section{\textbf{SimpleSnapshotTest}}

\subsection{Particular Case (problem)}
The problem we want to solve is that of atomic snapshots: we want to
read atomically a set of registers. For this problem we assume the
registers are themselves atomic, and that each one supports
multiple-readers but just one writer. The problem is better defined in 
terms of the Java interface we want to implement: \\

\begin{lstlisting}[style=nonumbers]
public interface Snapshot<T> {
  public void update(T v);
  public T[] scan();
}
\end{lstlisting}
\hfill

While the method \C{update} allows each thread to modify its own
register (single-writer), the \C{scan} method allows any thread to
read all the registers as a single atomic operation
(multiple-readers). Ideally, we would like to have a wait-free
implementation of these two methods.

\subsection{Solution}
The solution presented in the book is a natural evolution to the
sequential implementation (which uses \C{synchronized} methods for
both \C{update} and \C{scan} methods). The solution is called
\C{SimpleSnapshot} and it uses stamped values for the \C{update}
method; there is no need to seek for the maximum, just to increase the
current stamp per thread (each thread only writes to its own
register). This makes the \C{update} method wait-free, which is the
ideal (it was not very hard to achieve indeed, given the single-writer
restriction). \\

For the \C{scan} method we call an auxiliary function \C{collect},
which represents a non atomic read of all the registers (which are
copied into a new array and returned); if two
consecutive \C{collect} calls return same values, it means that between
those two calls there were no writes. When we reach that condition, we
return the array of values; otherwise we repeat the iteration. Please
note that \C{scan} method is not wait-free (we do not know how many
times we will need to iterate), but at least is
obstruction-free (we are not blocking other threads from trying their
own \C{scan} calls). \\

The main parts of the code are presented below for reference: \\

\begin{lstlisting}[style=numbers]
public class SimpleSnapshot<T> implements Snapshot<T> {
  // array of atomic MRSW registers
  private StampedValue<T>[] a_table;  
  ...
  public void update(T value) {
    int me = ThreadID.get();
    StampedValue<T> oldValue = a_table[me];
    StampedValue<T> newValue =
        new StampedValue<T>((oldValue.stamp)+1, value);
    a_table[me] = newValue;
  }
  private StampedValue<T>[] collect() {
    StampedValue<T>[] copy = (StampedValue<T>[]) new StampedValue[a_table.length];
    for (int j = 0; j < a_table.length; j++)
      copy[j] = a_table[j];
    return copy;
  }
  public T[] scan() {
    StampedValue<T>[] oldCopy, newCopy;
    oldCopy = collect();
    collect: while (true) {
      newCopy = collect();
      if (! Arrays.equals(oldCopy, newCopy)) {
        oldCopy = newCopy;
        continue collect;
      }
      // clean collect
      T[] result = (T[]) new Object[a_table.length];
      for (int j = 0; j < a_table.length; j++)
        result[j] = newCopy[j].value;
      return result;
    }
  }
}
\end{lstlisting}
\hfill

\subsection{Experiment Description}

The test program \C{SimpleSnapshotTest} consists of two individual
test cases: 

\begin{itemize}
  \item \C{testSequential}: with a single thread we update its value
    first (11), and then obtain an snapshot (\C{scan}); expectation is
    to have a single value in array (the one we wrote).
  \item \C{testParallel}: we create a couple of threads, and each one
    writes its own register twice (first putting 11, then 22). Then
    each thread proceeds to call \C{scan} method and saves its own
    returned values into a global results table. At the end of test,
    main thread check that both threads got the same results (which
    should be a two element array, both with value 22). 
\end{itemize}

\subsection{Observations and Interpretations}
The test runs fine, both in 2 and 24 core machines; sample output
below: \\

\begin{verbatim}
$ junit register.SimpleSnapshotTest
.sequential
.parallel

Time: 0.002

OK (2 tests)
\end{verbatim}
\hfill

