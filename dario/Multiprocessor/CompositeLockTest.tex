\section{\textbf{CompositeLockTest}}

\subsection{Particular Case (problem)}
The particular problem we are trying to solve, is that of looking for
a good balance between the trade-offs imposed by queue locks vs backoff
locks. On one hand, queue locks provide FIFO fairness, fast lock
release, and low contention, but require nontrivial
protocols for recycling abandoned nodes; on the other hand, backoff
locks support trivial timeout protocols, but are not scalable, and may
have slow lock release if timeout parameters are not well-tuned.

\subsection{Solution}
The solution proposed is to combine both type of locks, and is based
on the observation that only the threads at the front of the queue
need to do lock handoffs; while the rest may be happy doing
backoffs. This solution occupies several pages at the book, and
putting all the code and the details here may be cumbersome; let us
just put the main high-level details. \\

The \C{CompositeLock} solution uses a short array of auxiliary lock
nodes (the array is static, it can not be resized). Once a thread
needs to acquire ``the'' lock it picks a random slot in the array;
which would have a node. If the auxiliary lock protecting that node is
used, the thread uses an adaptive backoff approach and retries (after
sleeping the required time). Once 
the thread could acquire the lock over the picked node, it places
that node into a \C{TOLock}-style queue. Then, the thread spins on
the preceding node, and when that node's owner indicates it has
finished, the thread can finally acquire ``the'' lock (and enter its
critical section). When the thread leaves due completion or time-out,
it releases ownership of the node, and another thread doing backoff 
can acquire it. There are far more details regarding the procedure to
recycle the freed nodes of the array, while multiple threads attempt
to acquire control over them; we omit those details here.

\subsection{Experiment Description}
There is a single test case \C{testParallel}, which creates a couple
of threads and asks each to increment 2048 times a shared counter. The
counter increment is the critical section, and is protected with a
\C{CompositeLock}. At the end the shared counter shall have a value of
4096. 

\subsection{Observations and Interpretations}

The test runs fine on our two and 24 core machines, sample execution below: \\

\begin{verbatim}
.
Time: 0.066

OK (1 test)
\end{verbatim}
\hfill

