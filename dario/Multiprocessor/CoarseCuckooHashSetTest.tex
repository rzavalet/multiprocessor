\section{\textbf{CoarseCuckooHashSetTest}}

\subsection{Particular Case (problem)}
The problem we try to solve is that of concurrent hashing, when we use
the hash to implement a set data structure. 

\subsection{Solution}
The solution uses three techniques combined: \\

\begin{itemize}
  \item cuckoo-hashing: which consists in shifting to the right
    elements, by performing consecutive swaps on elements having a
    collision due hash function. \\
  \item double hashing where we use a couple of hash tables; the first
    one is based on a simple hash function (modulus of the element's
    hash code), and the second table is indexed with random numbers
    generated out of a sequence feed from the hash code of elements.
    \item coarse access: as the whole structure is being protected by
      mutual exclusion with a lock.
\end{itemize}
\hfill

Below the most representative function, \C{add}, along with the two
hash functions used: \\

\begin{lstlisting}[style=numbers]
  private final int hash0(Object x) {
    return x.hashCode() % size;
  }

  private final int hash1(Object x) {
    random.setSeed(x.hashCode());
    return random.nextInt(size);
  }

  public boolean add(T x) {
    lock.lock();
    try {
      if (contains(x)) {
        return false;
      }
      for (int i = 0; i < LIMIT; i++) {
        if ((x = swap(0, hash0(x), x)) == null) {
          return true;
        } else if ((x = swap(1, hash1(x), x)) == null) {
          return true;
        }
      }
      System.out.println("uh-oh");
      throw new CuckooException();
    } finally {
      lock.unlock();
    }
  }
\end{lstlisting}
\hfill

Something additional to mention about this solution, is that when
adding a new element we constraint the maximum number of swaps to
perform (while shifting the elements to the right). In this
implementation is \C{LIMIT} is 32; if after those attempts we could
not arrange all elements on its slot we raise an exception.

\subsection{Experiment Description}
The test program consists of three test cases:

\begin{itemize}
  \item \C{testSequential}: add 512 elements to the hash set
    (sequentially), perform a \C{constains} test for each one
    (sequentially) and finally remove the 512 elements
    (sequentially). \\ 
  \item \C{testParallelAdd}: add 512 elements to the hash set
    (in parallel with 8 threads), perform a \C{constains} test for each one
    (sequentially) and finally remove the 512 elements
    (sequentially). \\ 
  \item \C{testParallelRemove}: add 512 elements to the hash set
    (sequentially), perform a \C{constains} test for each one
    (sequentially) and finally remove the 512 elements
    (in parallel with 8 threads). \\
  \item \C{testParallelBoth}: adds and removes 512 elements to the
    hash set in parallel with 8 adder threads, and 8 remover threads;
    the removers complain if someone else removed its designated range
    of elements.
\end{itemize}

\subsection{Observations and Interpretations}
The tests works fine on both machines (2 and 24 cores respectively),
with the exception that the parallel test remover threads complain
when they try to delete an element which does not exist yet (which is
an expected consequence of the test). A sample output is shown below:

\begin{verbatim}
[oraadm@gdlaa008 orig]$ junit hash.CoarseCuckooHashSetTest

.sequential add, contains, and remove
.parallel both
Exception in thread "Thread-5" Exception in thread "Thread-9"
                Exception in thread "Thread-11" Exception in thread
                
"Thread-1" junit.framework.AssertionFailedError: DeqThread: duplicate
                
remove
at junit.framework.Assert.fail(Assert.java:57)
at junit.framework.TestCase.fail(TestCase.java:227)
at
                
...
                
.parallel add
.parallel remove

Time: 0.053

OK (4 tests)
\end{verbatim}
\hfill

