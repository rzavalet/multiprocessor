\section{\textbf{SenseBarrierTest}}

\subsection{Particular Case (problem)}
The problem we are trying to solve is that of synchronization on a
multi-threaded environment; and the particular mechanism we want to
use for that is a barrier (same as \C{DisBarrierTest}).

\subsection{Solution}
The solution called a ``Sense Barrier'' is well suited for reusing
barrier structures (given that barriers construction, in general, is
kind of expensive). The approach introduces the concept of a \C{sense}; which
is basically a boolean flag used for both the whole barrier structure
and for each thread. An implementation detail is that the barrier's
sense is a shared variable among threads (hence declared \C{volatile})
and the thread's sense is implemented with a \C{ThreadLocal}
variable. \\

The main idea behind this solution is that the barrier's sense is
initialized to the negation of the threads' senses; and that a call to
the \C{await} method will block all threads to wait for the barrier's
sense to be negated. Exception is the last thread to come, which is
allowed to actually change the barrier's sense hence unblocking the
rest of the threads. Prior leaving the
method each thread also negates its own local sense flag (so the
contract between barrier and thread senses is kept for next
usage). The code is small enough to fit here, and is pasted below: \\

\begin{lstlisting}[style=numbers]
public class SenseBarrier implements Barrier {
  AtomicInteger count;     // how many threads have arrived
  int size;                // number of threads
  volatile boolean sense;  // object's sense
  ThreadLocal<Boolean> threadSense;
  
  public SenseBarrier(int n) {
    count = new AtomicInteger(n);
    size = n;
    sense = false;
    threadSense = new ThreadLocal<Boolean>() {
      protected Boolean initialValue() { return !sense; };
    };
  }

  public void await() {
    boolean mySense = threadSense.get();
    int position = count.getAndDecrement();
    if (position == 1) { // I'm last
      count.set(this.size);     // reset counter
      sense = mySense;          // reverse sense
    } else {
      while (sense != mySense) {} // busy-wait
    }
    threadSense.set(!mySense);
  }
}
\end{lstlisting}
\hfill

One specially attractive feature of this approach, that we also saw
while studying Spin locks in chapter 7, is that the threads are spinning
on a local variable (which is expected to be cached on their
respective core). One may think that the while expression is using
both a local and a global variable, but the global one is likely to be
cached for most of the time; and its entry in the cache gets
invalidated only when is time to leave the barrier (after last thread
changed it), so traffic on the cores' bus only occurs at the end. 

\subsection{Experiment Description}
The test consists in having a shared array of 8 slots, one per thread
to be created. Then the test has two phases, both using the barrier;
the first phase is to wait all threads to write on their assigned slot
on the array and the second stage is to allow all threads to validate
that the rest actually finished their writing. These two phases are
repeated 8 times. \\

The test above is basically the same as that described for
\C{DisBarrierTest}.

\subsection{Observations and Interpretations}
Contrary to \C{DisBarrierTest}, this test does not hang on 2, nor on 24
core machines. This is because the shared variable \C{sense} is
properly declared as \C{volatile}. Below a sample successful
execution: \\

\begin{verbatim}
[oraadm@gdlaa008 orig]$ junit barrier.SenseBarrierTest
.Testing 8 threads, 8 rounds

Time: 0.012

OK (1 test)
\end{verbatim}
\hfill

