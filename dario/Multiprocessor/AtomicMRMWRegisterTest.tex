\section{\textbf{AtomicMRMWRegisterTest}}

\subsection{Particular Case (problem)}
The problem we are trying to solve here, is that of implementing an
atomic multi-valued \footnote{As opposed to boolean values, which have
only a couple of values; while multi-valued can range over a big set,
like integers.}
multiple-reader multiple-writer register. Let us 
briefly recall that atomic registers are the most powerful ones, when
compared with the 
other two categories: safe and regular. Informally, atomic registers
behave like one would expect when reads overlap writes: the reads
always ``see'' the last written value; while regular registers allows
to see either previous or latest value. Finally ``safe'' registers
allow reads to see any value (not safe at all then!). 

\subsection{Solution}
The solution from the book uses as primitive blocks the atomic
multi-valued multi-reader single-writer registers; we build an array of them whose
size equals the number of threads we want to support (assuming number
of writers is same as readers). When a given thread \C{A} wants to
write a value, it needs to read all the values on the array and writes
to its own cell an stamped value that is bigger than any one
observed. When same thread wants to read a value, it reads again whole
array and returns the one with biggest stamped value (resolving ties
with thread ids). This is essentially the same as Lamport's Bakery
algorithm from mutual exclusion chapter; the code is small enough to
fit here, so there it goes for reference: \\

\newpage
\begin{lstlisting}[style=numbers]
public class AtomicMRMWRegister<T> implements Register<T>{
  // array of multi-reader single-writer registers
  private StampedValue<T>[] a_table; 
  public AtomicMRMWRegister(int capacity, T init) {
    a_table = (StampedValue<T>[]) new StampedValue[capacity];
    StampedValue<T> value = new StampedValue<T>(init);
    for (int j = 0; j < a_table.length; j++) {
      a_table[j] = value;
    }
  }
  public void write(T value) {
    int me = ThreadID.get();
    StampedValue<T> max = StampedValue.MIN_VALUE;
    for (int i = 0; i < a_table.length; i++) {
      max = StampedValue.max(max, a_table[i]);
    }
    a_table[me] = new StampedValue<T>(max.stamp + 1, value);
  }
  public T read() {
    StampedValue<T> max = StampedValue.MIN_VALUE;
    for (int i = 0; i < a_table.length; i++) {
      max = StampedValue.max(max, a_table[i]);
    }
    return max.value;
  }
}
\end{lstlisting}

\subsection{Experiment Description}
The test program AtomicMRMWRegisterTest includes the following individual
test cases: 

\begin{itemize}
  \item \C{testSequential}: calls \C{write} method first with a value
    of 11, then calls \C{read} method expecting read 11. A single
    thread is used (main thread).
  \item \C{testParallel}: calls twice method \C{write}, putting first
    11 then 22 value. Then proceeds to create 8 reader threads, which
    expect all to read 22. This test is not really exercising
    multi-write capability of the register.
    sequentially.
\end{itemize}

\subsection{Observations and Interpretations}
The test performs well on 2 and 24 core machines, without any race
conditions nor abnormal behaviors. This partly because, the test is
not really exercising the multi-write capability, nor the interlacing
of writes and reads. We believe the authors just copied paste a
previous test (probably one that just cared about single reader and
multiple writers), and forgot to update. Due time constraints we did
not modify the test to exercise those missing features; a sample
execution is shown below: \\\\

\begin{verbatim}
$ junit register.AtomicMRMWRegisterTest
.sequential read and write
.parallel read

Time: 0.004

OK (2 tests)
\end{verbatim}
\hfill
