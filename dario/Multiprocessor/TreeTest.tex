\section{\textbf{TreeTest}}

\subsection{Particular Case (problem)}
This problem belongs to chapter 12, where the idea is to present
problems which look inherently serial but that surprisingly accept
interesting concurrent solutions (though not always easy to
explain). The particular problem this test is exercising is that of
shared 
counting, where we have \C{N} threads to increase a shared numeric
variable up to certain value; but with the goal of producing less
contention than a serial solution would generate. 

\subsection{Solution}

The java program is called \C{Tree.java}, although it corresponds to  
the \C{CombiningTree} from the book, which is a binary tree structure
where each thread is located at a leaf and at most two threads can
share a leaf. The shared counter is at the root of the tree, and the
rest of the nodes serve as intermediate result points. When a couple
of threads collide in their attempt to increment, only one of them
will serve as the representative and go up in the tree with the mission
of propagating the combined increment (2) up to the shared counter at
the root of the tree. In its way up, it may encounter further threads
that collide again with it, making it wait or continue its journey to
the root. \\

When a thread reaches the root it adds its accumulated result to the
shared counter, and propagates down the news that the job is done to
the rest of the threads that waited along the way. This
solution to the counting problem has worse latency than lock-based
solutions (\bigO{1} vs \bigO{log(p)}, where $p$ is the number of
threads or processors; but it offers a better throughput. 

\subsection{Experiment Description}

The program consists of a single unit test \C{testGetAndIncrement},
which creates 8 threads to perform each $2^20$ increments. The program
uses an auxiliary array called \C{test} to record the individual
increments done by each thread; assertions are made to ensure that
none of the attempts results in a duplicated value, as well as to
ensure that all threads completed their task.

\subsection{Observations and Interpretations}

The test performs without much controversy in an elapsed time between
3 and 4 seconds (laptop computer with i5 processor). Below a sample
output: 

\begin{verbatim}
.Parallel, 8 threads, 1048576 tries

Time: 3.641
\end{verbatim}

To make the test a little more interesting, we created another couple
of tests reusing sample template of existing one; difference was the
Thread class used on each case. \\

The first additional test uses a thread class based on Java provider \\
\C{java.util.concurrent.atomic.AtomicInteger}: \\

\begin{lstlisting}[style=nonumbers]
  cnt2 = new AtomicInteger();

  class MyThread2 extends Thread {
    public void run() {
      for (int j = 0; j < TRIES; j++) {
        int i = cnt2.getAndIncrement();
        if (test[i]) {
          System.out.printf("ERROR duplicate value %d\n", i);
        } else {
          test[i] = true;
        }
      }
    }        
  }
\end{lstlisting}
\hfill

while the second additional test was based on a custom class that used
synchronized \C{getAndIncrement} method:\\

\begin{lstlisting}[style=nonumbers]
  class MyThread3 extends Thread {
    public void run() {
      for (int j = 0; j < TRIES; j++) {
        int i = cnt3.getAndIncrement();
        if (test[i]) {
          System.out.printf("ERROR duplicate value %d\n", i);
        } else {
          test[i] = true;
        }
      }
    }        
  }

  class MyCounter {
    private int cnt = 0;

    public synchronized int getAndIncrement()
    {
      return cnt++;
    }
  }
\end{lstlisting}
\hfill

The expectation was that the test test based on a synchronized method
would be the slowest (\C{Parallel3} label on output), the one based on
the \C{CombineTree} would become second (\C{Parallel1} label on
output) and the fastest would be the one based on \C{AtomicInteger}
(\C{Parallel2}); as the latest is likely to take advantage of hardware
atomic instruction of 32bits. Surprisingly, the test based on
\C{CombineTree} was the worst of all, by far: \\

\begin{verbatim}
.Parallel1, 8 threads, 1048576 tries took 3878
.Parallel2, 8 threads, 1048576 tries took 162
.Parallel3, 8 threads, 1048576 tries took 723

Time: 4.827

OK (3 tests)
\end{verbatim}
\hfill

Most likely we are not comparing apples with apples, as the theory
does not match the experiment; perhaps the \C{CombineTest} class is
not meant for real usage, so this comparison is not fair. Anyway, the
test does not hang nor fails on neither the 2 cores nor the 24 machines.





