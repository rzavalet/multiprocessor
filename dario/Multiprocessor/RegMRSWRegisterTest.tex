\section{\textbf{RegMRSWRegisterTest}}

\subsection{Particular Case (problem)}
The problem we face here is that of building a regular multi-valued
multiple-reader single-writer register; out of simpler constructions,
like the rest of chapter 4. 

\subsection{Solution}
The solution, though supporting multi-values (integers), is bounded in
the range of values. We use as primitive blocks regular boolean
multiple-reader single-writer registers; and create an array of them
(as many as values in supported range), and use it to represent
numbers in unary way. That is, each position on the array that is set
to true represents the number corresponding to that position (the
index itself is the number). \\

Initially the boolean array is initialized to zero value, indicated by
having true the cell at index 0. The write method of value \C{x} sets
to true array position \C{x}, and updates to false the lower value
positions (so it updates from right to left). In order to guarantee
the regular property, the read method operates on the opposite
direction: it goes from left to right, starting at zero position, and
returns the first position whose value is true (an invariant of the
array should be that there is at least one cell in with true
value). The code from the book is presented below, for further
reference (the range of values picked
by the book's authors was that of byte Java type): \\

\begin{lstlisting}[style=numbers]
public class RegMRSWRegister implements Register<Byte> {
  private static int RANGE = Byte.MAX_VALUE - Byte.MIN_VALUE + 1;
  // regular boolean mult-reader single-writer register
  boolean[] r_bit = new boolean[RANGE]; 
  public RegMRSWRegister(int capacity) {
    r_bit[0] = true; // least value
  }
  public void write(Byte x) {
    r_bit[x] = true;
    for (int i = x - 1; i >= 0; i--)
      r_bit[i] = false;
  }
  public Byte read() {
    for (int i = 0; i < RANGE; i++)
      if (r_bit[i]) {
        return (byte)i;
      }
    return -1;  // never reached
  }
}
\end{lstlisting}
\hfill

\subsection{Experiment Description}
The test program \C{RegMRSWRegisterTest} includes the following individual
test cases (which is pretty much the same thing as test
\C{AtomicMRMWRegisterTest}):

\begin{itemize}
  \item \C{testSequential}: calls \C{write} method first with a value
    of 11, then calls \C{read} method expecting read 11. A single
    thread is used (main thread).
  \item \C{testParallel}: calls twice method \C{write}, putting first
    11 then 22 value. Then proceeds to create 8 reader threads, which
    expect all to read 22. 
\end{itemize}

\subsection{Observations and Interpretations}
Just like \C{AtomicMRMWRegisterTest}, this does not exhibit any issue
on two nor in 24 core machines. Again, part of that reason is that the
test is not really interlacing writes with reads. A sample successful
execution is shown below: \\

\begin{verbatim}
$ junit register.RegMRSWRegisterTest
.sequential read and write
.parallel read

Time: 0.004

OK (2 tests)
\end{verbatim}
\hfill

