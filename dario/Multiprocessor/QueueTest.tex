\section{\textbf{QueueTest}}

\subsection{Particular Case (problem)}
The problem we want to solve is again mutual exclusion around a shared
bounded queue; just like chapter 2. The difference here is that, instead of
just using locks the authors try to explain the facilities provided by
Java \C{Condition} interface (part of package
\C{java.util.concurrent.locks}). Such interface, along with the
\C{Lock} one, offer a finer grain control over the
Monitors \footnote{Monitors, as explained in the book, are an
  structured way to integrate synchronization with an object's data
  and methods (speaking about OO paradigm).}
capabilities offered by Java language; when compared to merely using
the \C{synchronized} feature. 

\subsection{Solution}
The solution to the mutual exclusion problem around the bounded queue is
solved by using a single Lock of type \C{ReentrantLock}, to control
exclusive access to the data structure (reentrant is important to
ensure that a thread which has gotten the lock already, can request it
again as many times as it want). In addition, the solution uses a
couple of \C{Condition} objects to represent the conditions 
which enqueuers and dequeuers wait on: \\

\begin{itemize}
  \item \C{notFull}: condition enqueuers wait on, prior daring to push.
  \item \C{notEmpty}: condition dequeuers wait on, prior daring to pop.
\end{itemize}
\hfill

The implementations of methods \C{enq} and \C{deq} are similar to
those from the flawed \C{LockFreeQueue} that we analyzed before; there
are a couple of crucial differences though: \\

\begin{itemize}
\item We use a lock to implement mutual exclusion, so we eliminate all
  the issues seen before due race conditions.
\item The potentially infinite loops for both \C{enq} and \C{deq} now
  do something: they call the \C{await} method on their respective
  conditions. This guarantees that we do not have deadlocks: if a
  thread does not find the queue on the expected state to perform the
  operation, it releases the lock and allow the others to proceed. In
  that way we allow enqueuers to put the data we are waiting for, or
  the dequeuers to make the space we need; after which they
  ``wake-up'' the waiting threads with a call to condition's
  \C{signal} method. 
\end{itemize}
\hfill

The code looks quite similar to our second proposed modification of previous
exercise \C{LockFreeQueueTest}; and actually, it also looks quite
similar to the sample code listed on the \href{http://docs.oracle.com/javase/7/docs/api/java/util/concurrent/locks/Condition.html}{Javadoc API of \C{Condition}
  interface} (on which we also based our proposal). Anyway, the book
solution is listed below for further reference: \\

\begin{lstlisting}[style=numbers]
class Queue<T> {
  final Lock lock = new ReentrantLock();
  final Condition notFull  = lock.newCondition();
  final Condition notEmpty = lock.newCondition();  
  final T[] items; 
  int tail, head, count;  
  public Queue(int capacity) {
    items = (T[])new Object[capacity];
  }
  public void enq(T x) throws InterruptedException {
    lock.lock();
    try {
      while (count == items.length)
        notFull.await();
      items[tail] = x;
      if (++tail == items.length)
        tail = 0;
      ++count;
      notEmpty.signal();
    } finally {
      lock.unlock();
    }
  }  
  public T deq() throws InterruptedException {
    lock.lock();
    try {
      while (count == 0)
        notEmpty.await();
      T x = items[head];
      if (++head == items.length)
        head = 0;
      --count;
      notFull.signal();
      return x;
    } finally {
      lock.unlock();
    }
  }
}
\end{lstlisting}
\hfill

\subsection{Experiment Description}

The test is pretty much the same than \C{LockFreeQueueTest}, except
for the number of threads (8 instead of 2) and data (64 instead of
512). 

\subsection{Observations and Interpretations}
Although the test does not exhibit any issue on 2 nor in 24 cores, we
would feel safer replacing the \C{signal} calls by \C{signalAll}, just
to totally discard the potencial problem of lost wake-ups explained on
the book (this is what we did for our second proposal to fix
\C{LockFreeQueueTest}, indeed). A sample successful execution is shown
below: \\ 

\begin{verbatim}
$ junit monitor.QueueTest
.sequential push and pop
.parallel both
.parallel deq
.parallel enq

Time: 0.009

OK (4 tests)
\end{verbatim}
\hfill

