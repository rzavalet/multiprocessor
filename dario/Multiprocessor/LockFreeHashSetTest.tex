\section{\textbf{LockFreeHashSetTest}}

\subsection{Particular Case (problem)}
The problem we try to solve is that of implementing a lock-free hash
set.

\subsection{Solution}
The solution uses a technique called ``Recursive Split-Ordering'',
where we keep all items on a single lock-free linked list, and we keep
separately an increasing array of buckets (which are just references
to different positions of the list). While any element could be found
in the set by transversing the list, idea is to use hash functions to
take advantage of the shortcuts that the buckets represent. The
algorithm takes its name from the fact that the split-ordered-keys on
the list, are the bitwise reverse representation of their items keys
(hash codes). 

\subsection{Experiment Description}
The test is exactly the same as that described for
\C{CoarseCuckooHashSetTest}. 

\subsection{Observations and Interpretations}
The test runs fine in 3 and 24 core machines, and similarly to
\C{CoarseCuckooHashSetTest}, suffers from a design flag on the test
program itself (remover threads may attempt to delete an element which
has not been added yet). Below a sample output of its execution: 

\begin{verbatim}
[oraadm@gdlaa008 orig]$ junit hash.LockFreeHashSetTest
.sequential add, contains, and remove
.parallel both
Exception in thread "Thread-1" junit.framework.AssertionFailedError:
                DeqThread: duplicate pop
at junit.framework.Assert.fail(Assert.java:57)
at junit.framework.TestCase.fail(TestCase.java:227)
at
                
hash.LockFreeHashSetTest$RemoveThread.run(LockFreeHashSetTest.java:143)
.parallel add
.parallel remove

Time: 0.037

OK (4 tests)
\end{verbatim}
\hfill

