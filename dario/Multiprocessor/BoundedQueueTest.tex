\section{\textbf{BoundedQueueTest}}

\subsection{Particular Case (problem)}
The problem is to implement a bounded partial queue (that is, one
which finite capacity which allows waits inside its methods). 

\subsection{Solution}
The solution proposes to allow concurrency among enqueuers and
dequeuers, by using two different locks (one for each group of
threads). The code is small enough to be placed and explained in a bit
more detail here, so let us give it a try:

\begin{lstlisting}[style=numbers]
  public void enq(T x) {
    if (x == null) throw new NullPointerException();
    boolean mustWakeDequeuers = false;
    enqLock.lock();
    try {
      while (size.get() == capacity) {
        try {
          notFullCondition.await();
        } catch (InterruptedException e) {}
      }
      Entry e = new Entry(x);
      tail.next = e;
      tail = e;
      if (size.getAndIncrement() == 0) {
        mustWakeDequeuers = true;
      }
    } finally {
      enqLock.unlock();
    }
    if (mustWakeDequeuers) {
      deqLock.lock();
      try {
        notEmptyCondition.signalAll();
      } finally {
        deqLock.unlock();
      }
    }
  }
\end{lstlisting}
\hfill

The main points of algorithm above are as follows (the code above was
previously corrected, as it had several things inverted):

\begin{itemize}
\item It does not allow null values.
\item There is a lock dedicated to \C{enq} method, which guarantees
  mutual exclusion to the critical section for enqueuers.
\item It spins while the queue is full (the try/catch block was added
  just to allow compilation). This line was incorrectly asking whether
  queue was empty, while it needs to ask whether is still full.
\item  Once we know queue is room, we allocate the new node and update
  \C{tail} pointer.
\item We atomically get and increment the atomic counter for the size
  (this was previously a decrement, which was incorrect).
\item If the atomically retrieved previous value of size was zero, it
  means the queue was empty and there may had been waiting dequeuers;
  therefore we acquire the lock for \C{deq} method and inform all
  dequeuers that the queue has not at least an item.
\end{itemize}
\hfill

The \C{deq} method is symmetric so we do not repeat same explanation
(but worth to say that it had same inverted conditions and actions
than \C{enq} method, in the code; so we needed to apply same
corrective actions ... perhaps that was the purpose of the exercise?).

\subsection{Experiment Description}
The test is pretty much the same described for \C{LockFreeQueueTest}.
(with the exceptiont that it uses 8 threads instead of two).

\subsection{Observations and Interpretations}
The original code for the test made no sense, as conditions and
actions for \C{enq} and \C{deq} methods were inverted; we did not even
care testing such strange version that did not match at all the one
published on the book (we also needed to fix the initialization of the
\C{size} variable, which shall be zero instead of \C{capacity}). After
the corrections mentioned on the Solution section above, the test ran
normally. Sample output below: 

\begin{verbatim}
.parallel both
.sequential push and pop
.parallel enq
.parallel deq

Time: 0.028

OK (4 tests)
\end{verbatim}
\hfill

Even if the original version of this test actually ran (we did not
test it), it turns out quite counter-intuitive.

