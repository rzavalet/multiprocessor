\section{\textbf{BitReversedCounterTest}}

\subsection{Particular Case (problem)}
The problem is actually a subproblem of the \C{FineGrainedHeap}
program, which has a \C{next} variable indicating next slot to use on
insert. The problem is to ensure that two consecutive paths from root
to leaves, where insertion takes place, share no common nodes other
than the root (this is useful to ensure that each insertion thread
locks a disjoint set of nodes, reducing contention on multi-threaded
scenarios). 

\subsection{Solution}
The solution is a bit cryptic, but small enough to fit here. It can be
proved that the \C{reverseIncrement} method below, produces sequence
of numbers with the above property. Given time constraints, we did not
prove such claim (specially cause this test itself, did not exhibit
concurrency): \\

\begin{lstlisting}[style=numbers]
  public int reverseIncrement() {
    if (counter++ == 0) {
      reverse = highBit = 1;
      return reverse;
    }
    int bit = highBit >> 1;
    while (bit != 0) {
      reverse ^= bit;
      if ((reverse & bit) != 0) break;
      bit >>= 1;
    }
    if (bit == 0)
      reverse = highBit <<= 1;
    return reverse;
  }
\end{lstlisting}
\hfill

\subsection{Experiment Description}
The test does not really make any assertions, and simply call the
\C{reverseIncrement} and \C{reverseDecrement} methods printing their
results. 

\subsection{Observations and Interpretations}
The test did not show any failures or hanging, given that is totally
sequential and does not really validate anything. Below a sample
execution output: \\

\begin{verbatim}
[oraadm@gdlaa008 orig]$ junit priority.BitReversedCounterTest
.increment
inc:	0	1
inc:	1	2
inc:	2	3
inc:	3	4
inc:	4	6
inc:	5	5
inc:	6	7
inc:	7	8
inc:	8	12
inc:	9	10
inc:	10	14
inc:	11	9
inc:	12	13
inc:	13	11
inc:	14	15
inc:	15	16
inc:	16	24
inc:	17	20
inc:	18	28
inc:	19	18
inc:	20	26
inc:	21	22
inc:	22	30
inc:	23	17
inc:	24	25
inc:	25	21
inc:	26	29
inc:	27	19
inc:	28	27
inc:	29	23
inc:	30	31
inc:	31	32
inc:	32	48
inc:	33	40
inc:	34	56
inc:	35	36
inc:	36	52
inc:	37	44
inc:	38	60
inc:	39	34
inc:	40	50
inc:	41	42
inc:	42	58
inc:	43	38
inc:	44	54
inc:	45	46
inc:	46	62
inc:	47	33
inc:	48	49
inc:	49	41
inc:	50	57
inc:	51	37
inc:	52	53
inc:	53	45
inc:	54	61
inc:	55	35
inc:	56	51
inc:	57	43
inc:	58	59
inc:	59	39
inc:	60	55
inc:	61	47
inc:	62	63
inc:	63	64
decrement
dec:	63
dec:	47
dec:	55
dec:	39
dec:	59
dec:	43
dec:	51
dec:	35
dec:	61
dec:	45
dec:	53
dec:	37
dec:	57
dec:	41
dec:	49
dec:	33
dec:	62
dec:	46
dec:	54
dec:	38
dec:	58
dec:	42
dec:	50
dec:	34
dec:	60
dec:	44
dec:	52
dec:	36
dec:	56
dec:	40
dec:	48
dec:	32

Time: 0.022

OK (1 test)
\end{verbatim}
\hfill

