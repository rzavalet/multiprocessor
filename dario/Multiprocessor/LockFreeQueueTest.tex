\newpage
\section{\textbf{LockFreeQueueTest}}

\subsection{Particular Case (problem)}
This is a particular case of the mutual exclusion problem, where the
shared resource is a queue.

\subsection{Solution}
In order to guarantee the correctness of
the multi-threaded access to the queue, it implements a lock free
scheme on its \C{enq} and \C{deq} methods by putting waits
before modifying the state of the queue. It does it incorrectly
though, as we will see later. \\

\begin{lstlisting}[style=numbers]
class LockFreeQueue {
  public int head = 0;   // next item to dequeue
  public int tail = 0;   // next empty slot
  Object[] items; // queue contents
  public LockFreeQueue(int capacity) {
    head = 0; tail = 0;
    items = new Object[capacity];
  }

 public void enq(Object x) {
     // spin while full
     while (tail - head == items.length) {}; // spin
     items[tail % items.length] = x;
     tail++;
  }

  public Object deq() {
      // spin while empty
      while (tail == head) {}; // spin
      Object x = items[head % items.length];
      head++;
      return x;
  }
}
\end{lstlisting}
\hfill

\subsection{Experiment Description}
The test program LockFreeQueueTest includes the following individual
test cases; the parallel degree is two threads for each operation
(queue or dequeue), with a TEST\_SIZE of 512 (number of items to
enqueue and dequeue) which is spread evenly among the two threads on
each group:

\begin{itemize}
  \item \C{testSequential}: calls \C{enq} method as many times as
    TEST\_SIZE, and later calls \C{deq} method the same number of
    times checking that the FIFO order is preserved. 
  \item \C{testParallelEnq}: enqueues in parallel (two threads) but dequeues
    sequentially.
  \item \C{testParallelDeq}: enqueues sequentially but dequeues in
    parallel (two threads)
  \item \C{testParallelBoth}: enqueues and dequeues in parallel (with
    a total of four threads, two for each operation).
\end{itemize}

The test program was run on two types of machines; one with two cores
only and another with 24 cores.

\subsection{Observations and Interpretations}
The bottom line of this exercise is most likely, to show several types
of problems that can occur if we do not use mutual exclusion; the
queue implementation fails implementing it, making the test vulnerable
to several cases of race conditions. Below we explain some of them.

\subsubsection{Non atomic dequeue: referencing invalid registers}

The symptom for this problem is a NullPointerException, and it
occurred on both test machines:

\begin{verbatim}
1) testParallelEnq(mutex.LockFreeQueueTest)java.lang.NullPointerException
at mutex.LockFreeQueueTest.testParallelEnq(LockFreeQueueTest.java:67)
at sun.reflect.NativeMethodAccessorImpl.invoke0(Native Method)
at sun.reflect.NativeMethodAccessorImpl.invoke(NativeMethodAccessorImpl.java:57)
at sun.reflect.DelegatingMethodAccessorImpl.invoke(DelegatingMethodAccessorImpl.java:43)
\end{verbatim}
\hfill

where the offending line is a cast to an Integer from the value
returned by the \C{deq} method; this implies that such function is
returning \C{null}. A possible scenario to produce this outcome is as
follows. Prefixes of T1 or T2 indicate the thread running the action,
and they refer to the line numbers of the code of method \C{deq}
posted above.

\begin{itemize}
\item Assume that queue holds a single element which lives in \C{items}
  array at position 0; the array has \C{null} on position 1 (per
  initialization).
\item T1: Executes method up to line 19. 
\item T2: Executes method up to line 19. 
\item T1: Executes lines 20, getting \C{items[0]}.
\item T1: Executes lines 21, increasing \C{head} to 1.
\item T2: Executes line 20, but as \C{head} was changed already, it
  gets \C{items[1]}.
\item T2: Executes lines 21, increasing \C{head} to 2.
\item T1: Executes line 22 returning \C{items[0]}
\item T2: Executes line 22 returning \C{items[1]}, which was \C{null}.
\end{itemize}
\hfill

The problem with above interlacing derives from the fact that the
\C{deq} method is not an atomic operation, hence it allowed both threads to
enter into the critical section and compete for updating the shared
variables. 

\subsubsection{Non atomic dequeue: returning duplicate values}

The symptom for this problem is a duplicate pop warning, and it
occurred on both test machines:

\begin{verbatim}
.parallel deq
Exception in thread "Thread-5" junit.framework.AssertionFailedError: DeqThread: duplicate pop
at junit.framework.Assert.fail(Assert.java:57)
at junit.framework.TestCase.fail(TestCase.java:227)
at mutex.LockFreeQueueTest$DeqThread.run(LockFreeQueueTest.java:129)
\end{verbatim}
\hfill

where the offending line is an assertion that validates that nobody
else has pop such value from the queue; as the threads which populate
do have non overlapping ranges of values, the pop operations (\C{deq})
shall never return a duplicate one. But duplication is possible
indeed, if we have a sequence like the one below between the two threads:

\begin{itemize}
\item Assume that queue holds a single element which lives in \C{items}
  array at position 0.
\item T1: Executes method up to line 19. 
\item T2: Executes method up to line 19. 
\item T1: Executes line 20, getting \C{items[0]}.
\item T2: Executes line 20, getting as well \C{items[0]}.
\item T1: Executes line 21, setting \C{head} to 1.
\item T2: Executes line 21, setting \C{head} to 2.
\item T1: Executes line 22 returning \C{items[0]}.
\item T2: Executes line 22 returning \C{items[0]}.
\end{itemize}
\hfill

Not only we left the queue in an inconsistent state (head has
incorrect value), but we also returned the same element twice,
triggering then the violation on the test. The underlying problem is
the same as previous case: lack of atomicity of the \C{deq} method.

\subsubsection{Non atomic auto-increment: loosing values}

The symptom for this problem is a never ending program, hanging on
either the test \C{testParallelBoth} or \C{testParallelEnq}; this
issue occurred on both test machines (though it was easier to
reproduce on the one with two cores). When produced several thread
dumps of the Java program, we can see either two hanging threads
(\C{testParallelBoth} case): \\

\begin{verbatim}
"Thread-7" #16 prio=5 os_prio=0 tid=0x00007f3140102000 nid=0x3f51
           runnable [0x00007f31226e9000]
   java.lang.Thread.State: RUNNABLE
	at mutex.LockFreeQueue.deq(LockFreeQueue.java:33)
	at mutex.LockFreeQueueTest$DeqThread.run(LockFreeQueueTest.java:141)

"main" #1 prio=5 os_prio=0 tid=0x00007f3140009800 nid=0x3f3c in Object.wait() 
          [0x00007f3148382000]
   java.lang.Thread.State: WAITING (on object monitor)
	at java.lang.Object.wait(Native Method)
	- waiting on <0x00000000d6effea8> (a mutex.LockFreeQueueTest$DeqThread)
	at java.lang.Thread.join(Thread.java:1245)
	- locked <0x00000000d6effea8> (a mutex.LockFreeQueueTest$DeqThread)
	at java.lang.Thread.join(Thread.java:1319)
	at mutex.LockFreeQueueTest.testParallelBoth(LockFreeQueueTest.java:123)
	at sun.reflect.NativeMethodAccessorImpl.invoke0(Native Method)
	at sun.reflect.NativeMethodAccessorImpl.invoke(NativeMethodAccessorImpl.java:62)
	at sun.reflect.DelegatingMethodAccessorImpl.invoke(DelegatingMethodAccessorImpl.java:43)
	at java.lang.reflect.Method.invoke(Method.java:497)
	at junit.framework.TestCase.runTest(TestCase.java:164)
	at junit.framework.TestCase.runBare(TestCase.java:130)
\end{verbatim}
\hfill

The main thread is waiting for a dequeue thread to finish; but that
one is on an infinite loop at line 19 of \C{deq} method (line numbers
per our listing in this document, not in the file). This means that we
have lost some of the inserted elements in queue, and that one of the
dequeue threads will never finish; as they expect each to pop a fixed
amount of elements per the test. \\

The other hanging scenario that comes out of (\C{testParallelEnq}), is
a single hanging thread, which was actually the main thread doing a 
serial dequeue. The common factor for both scenarios was a parallel
enqueue. \\ 

But the amount of times we request a dequeue operation, among all
threads, is the same as the number of elements we queued; how come we
end up loosing some of those? One possible explanation is again, the
lack of atomicity but this time of the \C{enq} method; to be more
specific, the lack of atomicity of its auto-increment operation 
\C{tail++} (line 14). Let us remember that the auto-increment
operator in Java is nothing but syntactic sugar for the following
sequence of operations (when applied to \C{tail} variable): \\

\begin{lstlisting}[style=nonumbers]
  tmp = tail;
  tmp = tmp + 1;
  tail = tmp;
\end{lstlisting} 
\hfill

If two threads execute the lower level operations above, we can see
how they can end up loosing increments in the shared variable
\C{tail}; the following sequence is on example of such scenario:

\begin{itemize}
\item T1: executes \C{tmp = tail}.
\item T2: executes \C{tmp = tail}.
\item T1: executes \C{tmp = tmp + 1}.
\item T2: executes \C{tmp = tmp + 1}.
\item T1: executes \C{tail = tmp}.
\item T2: executes \C{tail = tmp}.
\end{itemize}
\hfill

We can appreciate that in the above interlacing, the final value of
the shared variable is $\C{tail} + 1$; instead of the expected value
of $\C{tail} + 2$. It would be enough to loose a single value this
way, in order to make the enqueue threads think that they inserted the
total of 512, while they really inserted 511; as each dequeue thread
will try to pop 256 each, only one of them will be able to finish
while the other will get blocked after having removed 255
entries. That is most likely the explanation for the hung threads we
pasted above \footnote{Note that it does not matter that the array
  \C{items} has populated all the correct entries, because the flow is
  controlled by the counters \C{tail} and \C{head}.}; the solution is
again to really implement mutual 
exclusion around the methods \C{deq} / \C{enq}; in such a way that
they become atomic operations.

\subsection{Proposed changes to fix the problems}

In class it was mentioned that the \C{head} and \C{tail} variables
were not declared as \C{volatile}; which in Java jargon means they are
susceptible to caching on each core. This means, that a write to any
of those two variables (which control the queue size) are not meant to
be reflected immediately to the other cores, unless we declare the
variable as \C{volatile}. We did several attempts to achive lock-free
implementation of this queue, by making the variables \C{volatile} but
it was enough; then we tried by making atomic only portions of the
\C{enq} and \C{deq} methods; but it did not work either. \\

All the three scenarios described before can be eliminated, if we make
the methods \C{enq} and \C{deq} atomic; this can be easily achieved in
Java by making them \C{synchronized}. However, by doing that, we will
loose parallelism among the two groups of threads (those calling
\C{enq} and those calling \C{deq}); this is because the
\C{synchronized} keyword uses as lock the whole object, so at any
moment in time, only one synchronized method can actually run within any object. In
order to overcome this limitation, we can use synchronized blocks
against two different lock objects (one for each operation). \\

Even with the changes above, we can still have issues; what if the
very first thread running is one calling \C{deq} method? It will find
the queue empty and loop forever. In order to prevent that, we should
remove the waiting operation out of the queue methods, and put them in
the test code itself. This is because, on the cited scenario that we
try to dequeue with an empty queue, we would expect to simply try
again (giving the chance to parallel enqueue threads to produce something for
us to pop). The final code which incorporates these fixes is listed
below: \\

\begin{lstlisting}[style=nonumbers]
class LockFreeQueue {
  private static Object enqLock = new Object();
  private static Object deqLock = new Object();
    
  public int head = 0;   // next item to dequeue
  public int tail = 0;   // next empty slot
  Object[] items; // queue contents
  public LockFreeQueue(int capacity) {
    head = 0; tail = 0;
    items = new Object[capacity];
  }

 public  boolean enq(Object x) {
     synchronized(enqLock)
         {
             if (tail - head == items.length) {
                 return false;
             }
             items[tail % items.length] = x;
             tail++;
             return true;
         }
  }

  public Object deq() {
      synchronized(deqLock)
          {
              if (tail == head) {
                 return null;
              }
              Object x = items[head % items.length];
              head++;
              return x;
          }
  }
}

\end{lstlisting}
\hfill

The test code was also modified, to make the enqueue and dequeue
threads to iterate until they have successfully called their
respective methods 256 times (total size of queue divided by number of
threads). A successful call is one that does not return \C{false} nor
\C{null}. The modified code was executed ten thousand times and it did
not produce any of the original problems we explained. Though not a
formal proof of its correctness, it is a good indication of the same
(the original program produced one of the cited problems quite often,
usually within 10 executions). \\

A probably more elegant solution we came out with, after reading about
Java facilities for locks and conditions, was the following; it has
the advantage that it does not require us to change the test case: \\ 

\newpage
\begin{lstlisting}[style=nonumbers]
class LockFreeQueue {
  int head = 0;   // next item to dequeue
  int tail = 0;   // next empty slot
  Object[] items; // queue contents

  final Lock lock = new ReentrantLock();
  final Condition notFull  = lock.newCondition(); 
  final Condition notEmpty = lock.newCondition(); 

  public LockFreeQueue(int capacity) {
    head = 0; tail = 0;
    items = new Object[capacity];
  }

  public void enq(Object x) {
    lock.lock();
    try {
      // spin while full
      while (tail - head == items.length) 
        notFull.await();
      items[tail % items.length] = x;
      tail++;
      notEmpty.signalAll();
    }
    catch(InterruptedException e) {
      throw new RuntimeException(e);
    }
    finally {
      lock.unlock();
    }
  }

  public Object deq() {
    lock.lock();
    try {
      // spin while empty
      while (tail == head)
        notEmpty.await();
      Object x = items[head % items.length];
      head++;
      notFull.signalAll();
      return x;
    }
    catch(InterruptedException e) {
      throw new RuntimeException(e);
    }
    finally {
      lock.unlock();
    }
  }
}  
\end{lstlisting}
\hfill

This second option is interesting on its usage of the \C{await} call
to the condition objects; such call provokes that a thread which got
granted a lock releases it to give a change to other threads to come
in. For example, an enqueuer may give a chance to a dequeuer to make
room for it; or, a dequeuer may give an enqueuer a chance to produce
something for it. Without those calls to \C{await}, we could end up
falling into never ending loops. \\

Another relevant detail of this second solution is the usage of
\C{signalAll}  method on conditions; this is to awake the threads
which were waiting on that particular condition. In our 
case, it corresponds to the enqueuers or dequeuers which
called \C{await} on their respective \C{notFull} or \C{notEmpty}
conditions. Thus, as soon as there is room an enqueuer is awakened
and as soon as there is data a dequeuer is awakened. As there are
tests where we can have more than one thread on each condition, we
prefer to use \C{signalAll} method rather than \C{signal} (which will
awake a single thread). This is to prevent the lost awakes problem
that is mentioned on the book (on the chapter about monitors, if we
recall correctly).



