\section{\textbf{TreeBarrierTest}}

\subsection{Particular Case (problem)}
The problem we are trying to solve is that of synchronization on a
multi-threaded environment; and the particular mechanism we want to
use for that is a barrier (same as other cases from same chapter
17). The subproblem we care about here, is to reduce the contention
generated on the last phase of solutions like \C{SenseBarrierTest}.

\subsection{Solution}
This solution is a natural extension of the ``Sense Barrier'' seen for
exercise \C{SenseBarrierTest}; where the barrier as a whole is split
into a several smaller sense barriers, and they are placed as the
leaves of a tree (each leaf node is setup with same amount of
elements, and this size is also called the \C{radix}). Then inside
each leaf node, we will suffer smaller 
contention than with a single sense barrier (simply cause the number
of threads to be unblocked is smaller). A key difference with the single
sense barrier approach though, is that the children nodes need to call
\C{await} method on the parent; this effectively means that all calls
will eventually reach the root node (and that is the way to know all
the threads have reached a consensus, and they can move on). The
relevant parts of code are shown below: \\

\begin{lstlisting}[style=numbers]
public class TreeBarrier implements Barrier {
  int radix;    // tree fan-in
  Node[] leaf;  // array of leaf nodes
  int leaves;   // used to build tree
  ThreadLocal<Boolean> threadSense; // thread-local sense
  public TreeBarrier(int n, int r) {
    radix = r;
    leaves = 0;
    this.leaf = new Node[n / r];
    int depth = 0;
    threadSense = new ThreadLocal<Boolean>() {
      protected Boolean initialValue() { return true; };
    };
    // compute tree depth
    while (n > 1) {
      depth++;
      n = n / r;
    }
    Node root = new Node();
    build(root, depth - 1);
  }

  void build(Node parent, int depth) {
    // are we at a leaf node?
    if (depth == 0) {
      leaf[leaves++] = parent;
    } else {
      for (int i = 0; i < radix; i++) {
        Node child = new Node(parent);
        build(child, depth - 1);
      }
    }
  }

  public void await() {
    int me = ThreadID.get();
    Node myLeaf = leaf[me / radix];
    myLeaf.await();
  }
  
  private class Node {
    AtomicInteger count;
    Node parent;
    volatile boolean sense;
    // construct root node
    public Node() {
      sense = false;
      parent = null;
      count = new AtomicInteger(radix);
    }
    public Node(Node parent) {
      this();
      this.parent = parent;
    }
    public void await() {
      boolean mySense = threadSense.get();
      int position = count.getAndDecrement();
      if (position == 1) {    // I'm last
        if (parent != null) { // root?
          parent.await();
        }
        count.set(radix);     // reset counter
        sense = mySense;
      } else {
        while (sense != mySense) {};
      }
      threadSense.set(!mySense);
    }
  }
}
\end{lstlisting}
\hfill

\subsection{Experiment Description}
The test consists in having a shared array of slots, as big as the
number of threads to use. Then the test has two phases, both using the
barrier; the first phase is to wait all threads to write on their
assigned slot on the array and the second stage is to allow all
threads to validate that the rest actually finished their
writing. These two phases are repeated 8 times. \\

The test above is basically the same as that described for
\C{DisBarrierTest} and \C{SenseBarrierTest}; except that we have a
couple of tests instead of one (each one tries a different radix
\footnote{The \C{radix} parameter controls the
  size of the leaf nodes, which represent smaller sense barriers}
and number of threads; the combinations of these two parameters are
$threads=8$,$radix=2$ and $threads=27$,$radix=3$ respectively.

\subsection{Observations and Interpretations}
The test runs fine in 2 and 24 core machines, a sample execution is
shown below: \\

\begin{verbatim}
[oraadm@gdlaa008 orig]$ junit barrier.TreeBarrierTest
.Testing 27 threads, 8 rounds
finished radix 3
.Testing 8 threads, 8 rounds
finished radix 2

Time: 0.023

OK (2 tests)
\end{verbatim}
\hfill

