\section{\textbf{MMThreadTest}}

\subsection{Particular Case (problem)}
The problem is that of matrix multiplication, but done in parallel. \\

Note: this exercise is exactly the same as the homonym from chapter
14; we believe the authors of the book made a mistake and ended up
with this repetition. Still we are repeating here for the sake of
completeness.

\subsection{Solution}
The solution creates one thread per cell on the result matrix, and for
each one it creates a new worker thread; which in turn calculates the
cell $(i,j)$ value by computing the dot product of row $i$ of left
matrix by column $j$ of right matrix (according to the definition of
the matrix product). Below the most relevant sections of the code: \\

\begin{lstlisting}[style=numbers]
  public MMThread(double[][] a, double[][]  b) {
    n = a.length;
    this.a = a;
    this.b = b;
    this.c = new double[n][n];
  }
  
  void multiply() {
    Worker[][] worker = new Worker[n][n];
    // create one thread per matrix entry
    for (int row = 0; row < n; row++) {
      for (int col = 0; col < n; col++) {
        worker[row][col] = new Worker(row,col);
      }
    }
    ...
  }

  class Worker extends Thread {
    int row, col;
    Worker(int row, int col) {
      this.row = row; this.col = col;
    }
    public void run() {
      double dotProduct = 0.0;
      for (int i = 0; i < n; i++) {
        dotProduct += a[row][i] * b[i][col];
      }
      c[row][col] = dotProduct;
    }
  }  
\end{lstlisting}
\hfill

\subsection{Experiment Description}
The test consists of multiplying two identify matrices of dimensions
$3 \times 3$, and compare the result with one of the operands (it
should be identity matrix again).

\subsection{Observations and Interpretations}
The test runs fine on 2 and 24 core machines, below a sample output: \\

\begin{verbatim}
[oraadm@gdlaa008 orig]$ junit steal.MMThreadTest
.run

Time: 0.004

OK (1 test)
\end{verbatim}
\hfill

