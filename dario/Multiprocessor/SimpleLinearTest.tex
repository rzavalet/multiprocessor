\section{\textbf{SimpleLinearTest}}

\subsection{Particular Case (problem)}
The problem is to implement a bounded priority queue, which are a
multi-set data structure; each set has an associated priority. The
contract or interface for a priority queue is essentially two methods:
\C{add(x,k)} which adds element $x$ to the set of priority $k$, and
  \C{removeMin} which returns and removes the element with smaller
  (higher) priority from all the sets. This type of priority queues
  are called ``bounded'', because the priority of elements is taken from
  a finite predefined set. 

\subsection{Solution}
Given that the problem is to implement a bounded priority queue, an
array shall suffice. So we create an array of sets as big as the range
of values for the priority. The \C{add} and \C{removeMin} methods are
straightforward, but they are built on the assumption that the
auxiliary class \C{Bin} is thread safe (which is the case, as it is
implemented with locks). Below the relevant snippets of code: \\

\begin{lstlisting}[style=numbers]
public class SimpleLinear<T> implements PQueue<T> { 
  int range;
  Bin<T>[] pqueue; 

  ...
  
  public void add(T item, int key) {
    pqueue[key].put(item);
  }

  ... 
  
  public T removeMin() {
    for (int i = 0; i < range; i++) { 
       T item = pqueue[i].get();
       if (item != null) {
         return item;
       }
    }
    return null; 
  }
}
\end{lstlisting}
\hfill

\subsection{Experiment Description}
The test program consists of the following test cases: \\

\begin{itemize}
  \item \C{testAdd}: Serial test which adds 512 random numbers in a
    fixed range (these numbers become the priority), and later validates
    that they are removed in ascending order with \C{removeMin}. \\
  \item \C{testParallelAdd}: Same as \C{testAdd}, with the exception
    that the addition of the elements is done in parallel with 8
    threads. \\
  \item \C{testParallelBoth}: Same as \C{testParallelAdd}, except that
    the removal of items is also performed in parallel with 8 threads.
\end{itemize}

\subsection{Observations and Interpretations}
The test runs fine in 2 and 24 core machines, below a sample output:
\\

\begin{verbatim}
[oraadm@gdlaa008 orig]$ junit priority.SimpleLinearTest
.add
OK.
...
OK.
OK.
.testParallelBoth
OK.
.testParallelAdd
OK.

Time: 0.116

OK (3 tests)
\end{verbatim}
\hfill
