\newpage
\section{\textbf{LockFreeQueueRecycleTest}}

\subsection{Particular Case (problem)}
From the generic problem of improving coarse grained lock approaches,
the particular approach followed on this exercise corresponds to the
extreme case: lock-free data structure. The particular case is for a
queue, and it has the additional bonus of recycling memory.

\subsection{Solution}
The solution is based on an 1996 ACM article from Maged M. Michel and
Michael L. Scott, who based on previous publications, suggest a new
way of implementing a lock-free queue with recycles. They implement the
queue with a single-linked list using \C{tail} and \C{head} pointers;
where \C{head} always points to a dummy or sentinel node which is the
first in the list, and \C{tail} points to either the last or second to
last node in the list. The algorithm uses ``compare and swap''
(\C{CAS}) with modification counters to avoid the ABA problem. \\

Dequeuers are allowed to free dequeued nodes by ensuring that \C{tail}
does not point to the dequeued node nor to any of its predecessors
(that is, dequeued nodes may be safely reused). To obtain consistent
values of various pointers the authors relied on sequences of reads
that re-check earlier values to ensure they have not changed (these
reads are claimed to be simpler than snapshots). 

\subsection{Experiment Description}
The test is pretty much the same described for \C{LockFreeQueueTest}
(with the exceptiont that it uses 8 threads instead of two).

\newpage
\subsection{Observations and Interpretations}
The test does not exhibit any pitfall, which suggests that the theory
works just fine on the tested machines (we tried the one with 2 and
with 24 cores). Sample output below:  

\begin{verbatim}
.parallel enq
.parallel deq
.parallel both
.sequential push and pop

Time: 0.023

OK (4 tests)
\end{verbatim}
\hfill

