\section{\textbf{FibTaskTest}}

\subsection{Particular Case (problem)}
The problem is calculating the sum of the first $n$ Fibonacci numbers,
in parallel. 

\subsection{Solution}
The solution merely follows the recursive definition of the Fibonacci
sequence, whose base cases are 1 (for first and second elements of the
sequence). Each recursive call of the function creates a new task
(\C{Future} from java concurrency library). Below the code of the main
function: \\

\begin{lstlisting}[style=numbers]
  public Integer call() {
    try {
      if (arg > 2) {
        Future<Integer> left = exec.submit(new FibTask(arg-1));
        Future<Integer> right = exec.submit(new FibTask(arg-2));
        return left.get() + right.get();
      } else {
        return 1;
      }
    } catch (Exception ex) {
      ex.printStackTrace();
      return 1;
    }
  }
\end{lstlisting}
\hfill

Note that \C{Future} class represents asynchronous calculations, and
that the \C{get} method is blocking (waits for calculation to be
complete); hence we would effectively create a  tree of tasks until we
reach the base cases and start evaluating bottom up.

\subsection{Experiment Description}
The experiment merely request the calculation of the sum up to a known
number (16), and asserts the expected result (987).

\subsection{Observations and Interpretations}
The test works as expected in both 2 and 24 cores machines. Below a
sample output: \\

\begin{verbatim}
[oraadm@gdlaa008 orig]$ junit steal.FibTaskTest
.run

Time: 0.128

OK (1 test)
\end{verbatim}
\hfill

