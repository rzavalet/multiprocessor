% --------------------------- %
% LockFreeQueueTest Start
% --------------------------- %
\section{\textbf{LockFreeQueueTest Test}}
%%%%%%%%%%%%%%%%%%%%%%%%%%%%%%%%%%
\subsection{Particular Case}
\par
In this exercise we deal with the problem of implementing a Queue data
structure with the particular characteristic of being lock free. 
\par
%%%%%%%%%%%%%%%%%%%%%%%%%%%%%%%%%%
\subsection{Solution}
\par
Here we present the implementation of the Lock Free Queue that is based on the paper by Michael and Scott. 
\par
Here is the \textit{enqueue()} method:
\par
\hfill
\begin{lstlisting}[style=numbers]
  /**
   * Append item to end of queue.
   * @param item
   */
  public void enq(T item) {
    if (item == null) throw new NullPointerException();
    Node node = new Node(item); // allocate & initialize new node
    while (true) {		 // keep trying
      Node last = tail.get();    // read tail
      Node next = last.next.get(); // read next
      if (last == tail.get()) { // are they consistent?
        AtomicReference[] target = {last.next, tail};
        Object[] expect = {next, last};
        Object[] update = {node, node};
        if (multiCompareAndSet(
            (AtomicReference<T>[]) target,
            (T[]) expect,
            (T[]) update)){
          return;
        }
      }
    }
  }
\end{lstlisting}
\hfill
\par
Here is the dequeue method.
\par
\hfill
\begin{lstlisting}[style=numbers]
  /**
   * Remove and return head of queue.
   * @return remove first item in queue
   * @throws queue.EmptyException
   */
  public T deq() throws EmptyException {
    while (true) {
      Node first = head.get();
      Node last = tail.get();
      Node next = first.next.get();
      if (first == head.get()) {// are they consistent?
        if (first == last) {	// is queue empty or tail falling behind?
          if (next == null) {	// is queue empty?
            throw new EmptyException();
          }
          // tail is behind, try to advance
          tail.compareAndSet(last, next);
        } else {
          T value = next.value; // read value before dequeuing
          if (head.compareAndSet(first, next))
            return value;
        }
      }
    }
\end{lstlisting}
\hfill
\par
%%%%%%%%%%%%%%%%%%%%%%%%%%%%%%%%%%
\subsection{Experiment Description}
\par
The test consists of 4 test cases. Let us briefly describe each one of them:
\par
\begin{itemize}
\item testSequential. In this test case 512 elements are enqueued in the data
structure. After that, each of the elements are dequeued. At each dequeue, we
verify that the dequeued elements are in sequential order (just as they were
inserted).
\item testParallelEnqueue. This test cases spawns $8$ enqueuer threads. Each of
them will add 512 elements to the queue. When they are done, the main thread
dequeues each of the elements performing the same check described above.
\item testParallelDequeue. In this test case, the master thread enqueues 512
elements. When done, 8 dequeuer threads are spawned. The job of each of these
dequeuer threads is to remove the elements. Additionally, an array that keeps
track of which elements have been already dequeued is mantained. The test
checks that no element is removed more than once.
\item testParallelBoth. This is the most complete test in the suite. It spawns
8 enqueuer threads and 8 dequeuer threads. Each enqueuer adds 512 elements to
the queue. The dequeuers must remove each of these elements. A check similar to
the one described in the \textit{testParallelDequeue} test is performed.
\end{itemize}
\par
%%%%%%%%%%%%%%%%%%%%%%%%%%%%%%%%%%
\subsection{Sample Results and Interpretation}
\par
As it is, the test fails as follows:
\par
\hfill
\begin{verbatim}
[oraadm@gdlaa008 Queue]$ junit queue.LockFreeQueueTest
.sequential push and pop
F.parallel both
.parallel deq
.parallel enq
E
Time: 0.012
There was 1 error:
1) testParallelEnq(queue.LockFreeQueueTest)queue.EmptyException
  at queue.LockFreeQueue.deq(LockFreeQueue.java:64)
  at queue.LockFreeQueueTest.testParallelEnq(LockFreeQueueTest.java:69)
  at sun.reflect.NativeMethodAccessorImpl.invoke0(Native Method)
  at
sun.reflect.NativeMethodAccessorImpl.invoke(NativeMethodAccessorImpl.java:57)
  at
sun.reflect.DelegatingMethodAccessorImpl.invoke(DelegatingMethodAccessorImpl.java:43)
There was 1 failure:
1) testSequential(queue.LockFreeQueueTest)junit.framework.AssertionFailedError:
unexpected empty queue
  at queue.LockFreeQueueTest.testSequential(LockFreeQueueTest.java:51)
  at sun.reflect.NativeMethodAccessorImpl.invoke0(Native Method)
  at
sun.reflect.NativeMethodAccessorImpl.invoke(NativeMethodAccessorImpl.java:57)
  at
sun.reflect.DelegatingMethodAccessorImpl.invoke(DelegatingMethodAccessorImpl.java:43)

FAILURES!!!
Tests run: 4,  Failures: 1,  Errors: 1
\end{verbatim}
\hfill
\par
The problem with it, is that it is using a stub function for
multi-compareAndSet:
\par
\hfill
\begin{lstlisting}[style=numbers]
  private static <T> boolean multiCompareAndSet(
      AtomicReference<T>[] target,
      T[] expect,                                                                                                                                                                    
      T[] update) {
    return true;
  }
\end{lstlisting}
\hfill
\par
Actually, what we did to get around it, is to implement the enqueue code as the
book says, that is:
\hfill
\begin{lstlisting}[style=numbers]
  /**                                                                                                                                                                                
   * Append item to end of queue.
   * @param item
   */
  public void enq(T item) {
    if (item == null) throw new NullPointerException();
    Node node = new Node(item); // allocate & initialize new node
    while (true) {     // keep trying
      Node last = tail.get();    // read tail
      Node next = last.next.get(); // read next
      if (last == tail.get()) { // are they consistent?
        if (next == null) {
          if (last.next.compareAndSet(next,node)) {
            tail.compareAndSet(last,node);
            return;
          }   
        }   
        else {
          tail.compareAndSet(last,next);
        }   
      }   
    }
  }   
\end{lstlisting}
\hfill
\par
After that, we were able to succesfully execute the test cases:
\par
\hfill
\begin{verbatim}
[oraadm@gdlaa008 Queue]$ junit queue.LockFreeQueueTest
.sequential push and pop
.parallel both
.parallel deq
.parallel enq

Time: 0.016

OK (4 tests)
\end{verbatim}
\hfill
\par
%%%%%%%%%%%%%%%%%%%%%%%%%%%%%%%%%%
% --------------------------- %
% LockFreeQueueTest End
% --------------------------- %
