\section{\textbf{Lock-Free List}}
% ---------------------------------------------------------------------------- %
\subsection{Particular Case}
\par
We have seen some algorithms for implementing a concurrent list. Now it is time
to see how a Lock-Free implementation can be done. That is our current problem.
\par
% ---------------------------------------------------------------------------- %
\subsection{Solution}
\par
There are two main ideas behind the Lock-Free implementation. The first one is
to use a lazy approach. That is, do not remove an element to the list
immediately. Instead, mark it as \textit{deleted}. The second idea is to treat the node's
\textit{next} and a new field \textit{marked} as a single atomic unit. 
\par
The second idea is important because that will allow us to have concurrent
\textit{add()} and \textit{remove()} with out missing updates.
\par
Three methods are important in the implementation of the Lock-Free List we are
discussing. The first one is the \textit{find()} method which encapsulates the
mechanism to search for a \textit{current} and \textit{previous} nodes that will
be affected by an \textit{add()} or \textit{remove()} operation. These two
fields are grouped in a \textit{Window} object.
\par
This method traverses the list. If it finds a node that is marked as deleted,
then it tries to actually remove it from the list. It does so by applying a
\textit{compareAndSet()} to both the \textit{pred.next} and its \textit{mark}.
This latter field is implicit in the \textit{AtomicMarkableReference<T>} object.
\par
\hfill
\begin{lstlisting}[style=numbers]
  public Window find(Node head, int key) {
    Node pred = null, curr = null, succ = null;
    boolean[] marked = {false}; // is curr marked?
    boolean snip;
    retry: while (true) {
      pred = head;
      curr = pred.next.getReference();
      while (true) {
        succ = curr.next.get(marked); 
        while (marked[0]) {           // replace curr if marked
          snip = pred.next.compareAndSet(curr, succ, false, false);
          if (!snip) continue retry;
          curr = pred.next.getReference();
          succ = curr.next.get(marked);
        }
        if (curr.key >= key)
          return new Window(pred, curr);
        pred = curr;
        curr = succ;
      }
    }
  }
\end{lstlisting}
\hfill
\par
The next interesting method is \textit{add()} which after finding its window, if
it sees that the value we are trying to insert is already there, returns false.
Otherwise it adds the new node using the \textit{compareAndSet()} operation.
\par
\hfill
\begin{lstlisting}[style=numbers]
  /**
   * Add an element.
   * @param item element to add
   * @return true iff element was not there already
   */
  public boolean add(T item) {
    int key = item.hashCode();
    boolean splice;
    while (true) {
      // find predecessor and curren entries
      Window window = find(head, key);
      Node pred = window.pred, curr = window.curr;
      // is the key present?
      if (curr.key == key) {
        return false;
      } else {
        // splice in new node
        Node node = new Node(item);
        node.next = new AtomicMarkableReference(curr, false);
        if (pred.next.compareAndSet(curr, node, false, false)) {
          return true;
        }
      }
    }
  }
\end{lstlisting}
\hfill
\par
Finally, we have the \textit{remove()} method. After finding its window, if it
does not see the value it is trying to delete, it returns false. Otherwise, it
marks the node as \textit{logically deleted}. After that, it tries to
\textit{physically delete} the node once. For the method, it is OK if it fails
since any other thread trying to add or remove will attempt to do it again.
\par
\hfill
\begin{lstlisting}[style=numbers]
  /**
   * Remove an element.
   * @param item element to remove
   * @return true iff element was present
   */
  public boolean remove(T item) {
    int key = item.hashCode();
    boolean snip;
    while (true) {
      // find predecessor and curren entries
      Window window = find(head, key);
      Node pred = window.pred, curr = window.curr;
      // is the key present?
      if (curr.key != key) {
        return false;
      } else {
        // snip out matching node
        Node succ = curr.next.getReference();
        snip = curr.next.attemptMark(succ, true);
        if (!snip)
          continue;
        pred.next.compareAndSet(curr, succ, false, false);
        return true;
      }
    }
  }
\end{lstlisting}
\hfill
\par
% ---------------------------------------------------------------------------- %
\subsection{Experiment Description}
\par
Again, we have the same four test cases that we have been dealing with:
\par
\begin{enumerate}
\item \textit{testSequential()}. The main thread inserts multiple elements to
the list, then checks if all elements are there and finally removes all elements
one by one.
\item \textit{testParallelAdd()}. The main thread spawns multiple threads that
add elements to the list concurrently. When they are done, the main thread
checks if all elements are in there and finally removes all elements.
\item \textit{testParallelRemove()}. The main thread inserts multiple elements
to the list, then checks if all elements are in there and finally it spawns
multiple threads that remove the elements from the list concurrently.
\item \textit{testParallelBoth()}. The main thread creates multiple threads.
Some of them insert elements to the list and some of them remove elements.
\end{enumerate}
\par
% ---------------------------------------------------------------------------- %
\subsection{Sample Results}
\par
Similar to the other implementations, the test cases turns out to have a
problem. Here is the output:
\par
\begin{verbatim}
[oraadm@gdlaa008 Lists]$ junit lists.LockFreeListTest
.sequential add, contains, and remove
.parallel both
Exception in thread "Thread-13" junit.framework.AssertionFailedError:
RemoveThread: duplicate remove: 109
        at junit.framework.Assert.fail(Assert.java:57)
        at junit.framework.TestCase.fail(TestCase.java:227)
        at lists.LockFreeListTest$RemoveThread.run(LockFreeListTest.java:145)
.parallel add
.parallel remove

Time: 0.022

OK (4 tests)
\end{verbatim}
\par
The fix for this is the same that has been discussed before. Here is the test
output after the fix:
\begin{verbatim}
[oraadm@gdlaa008 Lists]$ junit lists.LockFreeListTest
.sequential add, contains, and remove
.parallel both
.parallel add
.parallel remove

Time: 0.016

OK (4 tests)
\end{verbatim}
\par
% ---------------------------------------------------------------------------- %
\subsection{Interpretation}
% ---------------------------------------------------------------------------- %
We observed a way in which a list could be implemented allowing it to be lock
free. When using the same set of unit tests as before, we see that the
implementation works correctly.
