% --------------------------- %
% GTreeBarrierTest Start
% --------------------------- %
\section{\textbf{GTreeBarrierTest}}
%%%%%%%%%%%%%%%%%%%%%%%%%%%%%%%%%%
\subsection{Particular Case}
\par
The problem we are trying to solve is how to implement a Barrier that mitigates
the problem of memory contention of the Sense-Reversing Barrier.
\par
%%%%%%%%%%%%%%%%%%%%%%%%%%%%%%%%%%
\subsection{Solution}
\par
We present a Generic Tree Barrier, which actually uses the Sense Reversing
Barrier, leading again to what the book called the Combining Tree Barrier.
However, the implementation is done in such a way that each node's barrier can
be changed to any other barrier. In fact, this exercise is a solution for
problem $201$ in the book.
\par
As implied by the name, this Barrier is implemented in a tree, notice its
similarity with the Simple Barrier Tree. The only difference is that we
factorized the call to the node's barrier allowing it to actually use any type
of barrier.
\par
\hfill
\begin{lstlisting}[style=numbers]
    public void await(int me) {
      boolean sense = threadSense.get();
      barrier.await();
      if (me % radix == 0) {
        if (parent != null) { // root?
          parent.await(me / radix);
        }
        done = sense;
      } else {
        while (done != sense) {};
      }
      threadSense.set(!sense);
    }
\end{lstlisting}
\hfill
\par
%%%%%%%%%%%%%%%%%%%%%%%%%%%%%%%%%%
\subsection{Experiment Description}
\par
The experiment consists of 8 threads and 8 rounds. The test will create a
GTreeBarrier with radix 2. Each thread will verify that its siblings are
performing the same round. This is all the work they have to do. If siblings
happen to be in a different round, it means that they are not honoring  the
imposed barrier.
\par
%%%%%%%%%%%%%%%%%%%%%%%%%%%%%%%%%%
\subsection{Sample Results}
\par
\par
%%%%%%%%%%%%%%%%%%%%%%%%%%%%%%%%%%
\subsection{Interpretation}
\par
\par
%%%%%%%%%%%%%%%%%%%%%%%%%%%%%%%%%%
\subsection{Proposed solution}
\par
\par
%%%%%%%%%%%%%%%%%%%%%%%%%%%%%%%%%%
% --------------------------- %
% GTreeBarrierTest End
% --------------------------- %
