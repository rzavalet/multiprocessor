% --------------------------- %
% FibThreadTest Start
% --------------------------- %
\section{\textbf{FibThreadTest Test}}
%%%%%%%%%%%%%%%%%%%%%%%%%%%%%%%%%%
\subsection{Particular Case}
\par
Here we are dealing of computing the Fibonacci numbers in parallel. In
particular we want to do so by spawning slave threads.
\par
%%%%%%%%%%%%%%%%%%%%%%%%%%%%%%%%%%
\subsection{Solution}
\par
The way to implement a parallel generator of Fibonacci number is by creating two
threads to calculate $F(n-1)$ and $F(n-2)$. This process is repeated
recursively. When each thread finishes calculating its values, the parent thread
adds the value of the children. Here is the code:
\par
\hfill
\begin{lstlisting}[style=numbers]
  public FibThread(int n) {
    arg = n;
    result = -1;
  }

  public void run() {
    FibThread left, right;
    if (arg < 2) {
      result = arg;
    } else {
      left = new FibThread(arg-1);
      right = new FibThread(arg-2);
      left.start();
      right.start();
      try {
       left.join();
       right.join();
      } catch (InterruptedException e) {};
      result = left.result + right.result;
    }
  }
\end{lstlisting}
\hfill
\par
%%%%%%%%%%%%%%%%%%%%%%%%%%%%%%%%%%
\subsection{Experiment Description}
\par
The test provided for this exercise consists of calculating $F(16)$. This result
must be equal to $987$. If that is the case, the test passes. Otherwise, it
fails.
\par
%%%%%%%%%%%%%%%%%%%%%%%%%%%%%%%%%%
\subsection{Sample Results}
\par
\par
%%%%%%%%%%%%%%%%%%%%%%%%%%%%%%%%%%
\subsection{Interpretation}
\par
As it is mentioned in the book, for this kind of applications where a lot of
short living threads is very inefficient. Instead, we should consider using a
thread pool.
\par
%%%%%%%%%%%%%%%%%%%%%%%%%%%%%%%%%%
\subsection{Proposed solution}
\par
\par
%%%%%%%%%%%%%%%%%%%%%%%%%%%%%%%%%%
% --------------------------- %
% FibThreadTest End
% --------------------------- %
