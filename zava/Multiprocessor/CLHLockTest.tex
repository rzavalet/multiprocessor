\section{\textbf{CLH Queue Lock}}
%%%%%%%%%%%%%%%%%%%%%%%%%%%%%%%%%%
\subsection{Particular Case}
\par
Using queue locks solve two problems that appear in simpler locking protocols.
Firstly, by using a queue, cache-coherence traffic is reduced because each
thread spin on a different memory location; and secondly, by using a queue,
threads can know whether their turn has come by inspecting the status of their
predecesor in the queue.
\par
The particular case that we want to study in this experiment is \textit{How can
we implemente a space efficient Queue Lock?}
\par
%%%%%%%%%%%%%%%%%%%%%%%%%%%%%%%%%%
\subsection{Solution}
\par
One solution for this problem is given by the CLHLock protocol. The algorithm
that implement this protocol need two fields local to the thread: A pointer to
the current node and a pointer to the previous node. The queue is constructed by
having the pointer \textit{previous} pointing to the previous node's
\textit{current} field. This field contains a boolean value which is false when
the thread releases the lock. When this is the case, the next thread is allowed
to acquire the lock. 
\par
%%%%%%%%%%%%%%%%%%%%%%%%%%%%%%%%%%
\subsection{Experiment Description}
\par
To demonstrate that this implementation works, the test that was provided does
the following:
\begin{enumerate}
\item Initiate a shared counter with a value of $0$
\item Start $8$ threads
\item Each thread has to increment the counter by one $1024$ times
\item At the end, the counter must hold a value of $8*1024$
\end{enumerate}
\par
%%%%%%%%%%%%%%%%%%%%%%%%%%%%%%%%%%
\subsection{Sample Results}
\par
In this case, we found that in our 24-cores machine, the test hung. To make it
work, we had to make the \textit{locked} variable in the 
\par
\hfill
\begin{verbatim}
[oraadm@gdlaa008 Spin]$ junit spin.CLHLockTest 
[8] Leaving critical zone
[9] Entering critical zone
[9] Leaving critical zone
[8] Entering critical zone
[8] Leaving critical zone
[9] Entering critical zone
[9] Leaving critical zone
[8] Entering critical zone
[9] Entering critical zone
...
[9] Leaving critical zone
[8] Entering critical zone
[9] Entering critical zone
[8] Leaving critical zone
[8] Entering critical zone
[8] Leaving critical zone

Time: 0.755

OK (1 test)
\end{verbatim}
\hfill
\par
%%%%%%%%%%%%%%%%%%%%%%%%%%%%%%%%%%
\subsection{Interpretation}
It was shown that the algorithm seems to work and allows synchronization of
Multiple threads trying to access a shared variable.
\par
One important thing about this algorithm is that threads spin checking for
different memory addresses which avoids invalidation of cached copies. Another
important advantage is that it requieres a limited number of memory and also
provides first-in-first-out fairness. 
\par
The case were this algorithm is not good is when, in a NUMA architecture, the
\textit{previous} pointer points to a remote location. In that case, the
performance of the algorithm degrades.
