% --------------------------- %
% FineGrainedHeapTest Start
% --------------------------- %
\section{\textbf{FineGrainedHeapTest Test}}
%%%%%%%%%%%%%%%%%%%%%%%%%%%%%%%%%%
\subsection{Particular Case}
\par
Now we look at the problem of implementing an Unbounded Priority Queue in a
Fine-Grained fashion. 
\par
Again, remember that Unbounded in this context means that the range is
unbounded, for example the set of 32-bits integers or floating point values.
\par
%%%%%%%%%%%%%%%%%%%%%%%%%%%%%%%%%%
\subsection{Solution}
\par
The solution proposed in this exercise is the Fine-Grained heap. The particular
characteristic of this implementation is that percolation hapens in a sequence
of discrete atomic steps that can be interleaved with other such steps. The same
happens with the \textit{removeMin()} method.
\par
The \textit{add()} method first locks the heap and then takes the lock to the
next available child and puts the new value there. After that, it releases the
heap lock. After this point, it must percolate this new node. To do this, it has
to swap child and parent after locking them. This only happens if the parent is
available and the child has a higher priority. Otherwise, the node is already
where it belongs. At the end of this step, child and parent are unlocked. Then
it goes up one node in the heap, and does the same.
\par
\hfill
\begin{lstlisting}[style=numbers]
  /**
   * Add item to heap.
   * @param item Uninterpreted item.
   * @param priority item priority
   */
  public void add(T item, int priority) {
    heapLock.lock();
    int child = next++;
    heap[child].lock();
    heapLock.unlock();
    heap[child].init(item, priority);
    heap[child].unlock();
    while (child > ROOT) {
      int parent = child / 2;
      heap[parent].lock();
      heap[child].lock();
      int oldChild = child;
      try {
        if (heap[parent].tag == Status.AVAILABLE && heap[child].amOwner()) {
          if (heap[child].score < heap[parent].score) {
            swap(child, parent);
            child = parent;
          } else {
            heap[child].tag = Status.AVAILABLE;
            heap[child].owner = NO_ONE;
            return;
          }
        } else if (!heap[child].amOwner()) {
          child = parent;
        }
      } finally {
        heap[oldChild].unlock();
        heap[parent].unlock();
      }
    }
    if (child == ROOT) {
      heap[ROOT].lock();
      if (heap[ROOT].amOwner()) {
        heap[ROOT].tag = Status.AVAILABLE;
        heap[child].owner = NO_ONE;
      }
      heap[ROOT].unlock();
    }
  }

\end{lstlisting}
\hfill
\par
The \textit{removeMin()} method first locks the heap, the bottom and the root
nodes. If the root is empty, then we are done. Otherwise, it swaps root and
bottom and unlocks bottom. After that, it has to percolate the new root
element. It follows a similar approach as the one mentioned in the
\textit{add()} method. One difference is that this percolation goes from top to
bottom.
\par
\hfill
\begin{lstlisting}[style=numbers]
  /**
   * Returns and removes lowest-priority item in heap.
   * @return lowest-priority item.
   */
  public T removeMin() {
    heapLock.lock();
    int bottom = --next;
    heap[bottom].lock();
    heap[ROOT].lock();
    heapLock.unlock();
    if (heap[ROOT].tag == Status.EMPTY) {
      heap[ROOT].unlock();
      heap[bottom].lock();
      return null;
    }
    T item = heap[ROOT].item;
    heap[ROOT].tag = Status.EMPTY;
    swap(bottom, ROOT);
    heap[bottom].owner = NO_ONE;
    heap[bottom].unlock();
    if (heap[ROOT].tag == Status.EMPTY) {
      heap[ROOT].unlock();
      return item;
    }
    int child = 0;
    int parent = ROOT;
    while (parent < heap.length / 2) {
      int left = parent * 2;
      int right = (parent * 2) + 1;
      heap[left].lock();
      heap[right].lock();
      if (heap[left].tag == Status.EMPTY) {
        heap[right].unlock();
        heap[left].unlock();
        break;
      } else if (heap[right].tag == Status.EMPTY || heap[left].score < heap[right].score) {
        heap[right].unlock();
        child = left;
      } else {
        heap[left].unlock();
        child = right;
      }
      if (heap[child].score < heap[parent].score) {
        swap(parent, child);
        heap[parent].unlock();
        parent = child;
      } else {
        heap[child].unlock();
        break;
      }
    }
    heap[parent].unlock();
    return item;
  }
\end{lstlisting}
\hfill
\par
%%%%%%%%%%%%%%%%%%%%%%%%%%%%%%%%%%
\subsection{Experiment Description}
\par
The same three test cases were provided in this experiment, so let us go
directly to the results.
\par
%%%%%%%%%%%%%%%%%%%%%%%%%%%%%%%%%%
\subsection{Sample Results}
\par
\par
%%%%%%%%%%%%%%%%%%%%%%%%%%%%%%%%%%
\subsection{Interpretation}
\par
\par
%%%%%%%%%%%%%%%%%%%%%%%%%%%%%%%%%%
\subsection{Proposed solution}
\par
\par
%%%%%%%%%%%%%%%%%%%%%%%%%%%%%%%%%%
% --------------------------- %
% FineGrainedHeapTest End
% --------------------------- %
