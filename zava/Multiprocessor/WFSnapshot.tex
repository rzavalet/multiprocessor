\section{\textbf{Wait-Free Snapshots}}
% ---------------------------------------------------------------------------- %
\subsection{Particular Case}
\par
In this experiment we are dealing with the problem of creating an algorithm to
create a Snapshot of a set of Register in such a way that the algorithm
garantees a Wait-Free operation.
\par
% ---------------------------------------------------------------------------- %
\subsection{Solution}
\par
The purpose of a Wait-Free snapshot is to overcome the problems of the Simple
snapshot presented before. Let us remember that such snapshot executes sucesive
collect() operations. Once it achieves a \textit{clean double collect}, it
returns the snapshot. Otherwise, it uses the following idea:
\par
When a double collect fails, it is because a update interfered. That means that
the updater could take a snapshot right before its update and other threads
could use it as their snapshot too. 
\par
Since this algorithm is a bit more complex, let us explain in more detail how
the implementation works.
\par
\hfill
\begin{lstlisting}[style=numbers]
  public WFSnapshot(int capacity, T init) {
    a_table = (StampedSnap<T>[]) new StampedSnap[capacity];
    for (int i = 0; i < a_table.length; i++) {
      a_table[i] = new StampedSnap<T>(init);
    }   
  }
\end{lstlisting}
\hfill
\par
Firstly, the algorithm uses an array to store the information.
The size of the array is, in this case, $n$, where $n$ is the number of
threads. The way in which this array is constructed is as an array of
\textit{StampedSnap}s. A StampedSnap is internally an array whose elements are
of type \textit{T}. A StampedSnap is a subclass of StampedValue which has 3
fields:
\par
\begin{itemize}
\item \textit{stamp}, which is a counter
\item \textit{owner}, which contains the thread id
\item \textit{value}, the actual value in the register
\end{itemize}
\par
Now let us look at an auxiliar operation which is \textit{collect()}. This
operation returns a copy of the array. Observe that the method creates a new
array of StampedSnap values and then iterates the existing array to copy each element.
\par
\hfill
\begin{lstlisting}[style=numbers]
  private StampedSnap<T>[] collect() {
    StampedSnap<T>[] copy = (StampedSnap<T>[]) new StampedSnap[a_table.length];
    for (int j = 0; j < a_table.length; j++)
      copy[j] = a_table[j];
    return copy;
  }
\end{lstlisting}
\hfill
\par
The next method to understand is \textit{scan()}. The method first creates a
backup or an old copy of the array. Then it takes a second collect. It then
compares both copies. If both copies are equal, then it means that we have a
double clean collect. And the result of the scan operation is any of the copies
just as in the Sequentian Snapshot case.
\par
On the other hand, if the copies are different, then there are two cases. For
this, we have a counter that indicates how many times, the copies have been
compared and found different. If it is the first time, then we take another
collect. If in the second collect, the copies are equal, then we fall in the
case described in the previous paragraph. If the copies are different again,
then it means that we can take the old copy as a result of the scan. 
\par
The justification of this is that if a thread A sees a thread B move twice while
it is performing repeated collects, then B executed a complete \textit{update()}
call within the interval of A's scan, and so it is correct for A to use B's
snapshot.
\par
\hfill
\begin{lstlisting}[style=numbers]
  public T[] scan() {
    StampedSnap<T>[] oldCopy;
    StampedSnap<T>[] newCopy;
    boolean[] moved = new boolean[a_table.length];
    oldCopy = collect();
    collect: while (true) {
      newCopy = collect();
      for (int j = 0; j < a_table.length; j++) {
        // did any thread move?
        if (oldCopy[j].stamp != newCopy[j].stamp) {
          if (moved[j]) {       // second move
            return oldCopy[j].snap;
          } else {
            moved[j] = true;
            oldCopy = newCopy;
            continue collect;
          }   
        }   
      }   
      // clean collect
      T[] result = (T[]) new Object[a_table.length];
      for (int j = 0; j < a_table.length; j++)
        result[j] = newCopy[j].value;
      return result;
    }   
  }
\end{lstlisting}
\hfill
\par
Finally, we have the \textit{update()} operation which simply updates the
register with the new stamp.
\par
\hfill
\begin{lstlisting}[style=numbers]
  public void update(T value) {
    int me = ThreadID.get();
    T[] snap = this.scan();
    StampedSnap<T> oldValue = a_table[me];
    StampedSnap<T> newValue =
      new StampedSnap<T>(oldValue.stamp+1, value, snap);
    a_table[me] = newValue;
  }
\end{lstlisting}
\hfill
% ---------------------------------------------------------------------------- %
\subsection{Experiment Description}
\par
In this experiment, the test looks a bit different. Let us explain how. In this
case we have two experiments:
\par
\begin{enumerate}
\item testSequential. We spawn a new thread that updates the register with a
value of \textit{FIRST} and right after that, it scans the value of the
register. The expected result is that this read retrieves the value of
\textit{FIRST}
\item testParallel. We spawn \textit{THREADS} number of threads (in our case it
is $2$). Each of them will first update its register with a value of
\textit{FIRST} and then with a value of \textit{SECOND}. Each thread then
updates a position in a matrix of size \textit{THREADS}$x$\textit{THREADS} with
the result of doing a $scan()$ operation. At the end we compare consecutive rows
in the matrix. The comparison is done entry by entry and we should see that for
some entries the first entry is greater and for some others it is smaller than
the second. What we must see is that all entries are equal and we allow the
first to sometimes be either greater or smaller than the second one.
\end{enumerate}
% ---------------------------------------------------------------------------- %
\subsection{Sample Results}
\par
For this test, we saw that in every try, both test cases passed. Here is the
output:
\par
\begin{verbatim}
[oraadm@gdlaa008 Register]$ junit register.WFSnapshotTest
.sequential
.parallel

Time: 0.003

OK (2 tests)
\end{verbatim}
\par
% ---------------------------------------------------------------------------- %
\subsection{Interpretation}
% ---------------------------------------------------------------------------- %
In this experiment we were able to observe how a Wait-Free Snapshot can be
constructed based on the idea of \textit{clean double collects} and on the idea
of using the updater's collect when the clean double collect method fails.
