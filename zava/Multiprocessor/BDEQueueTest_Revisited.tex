% --------------------------- %
% BDEQueueTest_Revisited Start
% --------------------------- %
\section{\textbf{BDEQueueTest Test}}
%%%%%%%%%%%%%%%%%%%%%%%%%%%%%%%%%%
\subsection{Particular Case}
\par
The problem we want to solve in this exercise is how do we perform Work
Distribution among threads in an effective way.
\par
%%%%%%%%%%%%%%%%%%%%%%%%%%%%%%%%%%
\subsection{Solution}
\par
The solution proposed in this exercise is by using Work-Stealing Dequeues.
Specifically, we will use a bounded version.
\par
The idea is that we will have a pool of tasks and each task has a queue of Work
to be done. It is possible that some threads finish the jobs in their queues
faster than other. In such situation, this algorithm will allow faster threads
to steal jobs from other threads' queues. 
\par
The implementation of this algorithm requires a DEQueue (Double Ended Queue).
When picking a job from a queue, we distinguish two cases. The first one, which
is the most common one, is that one thread picks a job from its own queue. In
that case, the thread uses the \textit{popBottom()} method.
\par
The other case is when a thread steals a job from another queue. In that case,
the thread uses the \textit{popTop()} method.
\par
Let us take a look at the interesting methods.
\par
\textit{popBottom()} distinguishes between two cases. If there is a conflict
between a \textit{popBottom()} and \textit{popTop()}, then it resets the
bookmark of the top of the queue using a \textit{compareAndSet()}. If it
succeeds, then it means that our method won and it returns the element. Other
wise it means that the stealer won, and we have to return null.
\par
If there is no conflict between the two pop methods, then we simply return the
element. There is no need to call a \textit{compareAndSet()}.
\par
\hfill
\begin{lstlisting}[style=numbers]
  Runnable popBottom() {
    // is the queue empty?
    if (bottom == 0)  // empty  `\label{line:steal:empty}`
      return null;
    bottom--;
    // bottom is volatile to assure all reads beyond this line see it
    Runnable r = tasks[bottom];
    int[] stamp = new int[1];
    int oldTop = top.get(stamp), newTop = 0;
    int oldStamp = stamp[0], newStamp = oldStamp + 1;
    // no conflict with thieves `\label{line:steal:noconflict}`
    if (bottom > oldTop)
      return r;
    // possible conflict: try to pop it ourselves
    if (bottom == oldTop) {
      // even if stolen by other, queue will be empty, reset bottom
      bottom = 0;
      if (top.compareAndSet(oldTop, newTop, oldStamp, newStamp))
        return r;
      } 
    return null;
  }
\end{lstlisting}
\hfill
\par
%%%%%%%%%%%%%%%%%%%%%%%%%%%%%%%%%%
\subsection{Experiment Description}
\par
Two test cases are provided with this program:
\begin{itemize}
\item testSequential. 16 threads are spawned. Each thread pushes a value in our
DEQueue. After that, even threads call \textit{popTop()} and odd threads call
\textit{popBottom()}. Depending on the pop'd value, an associated slot in an
array is marked as true. The test checks that no slot in the array is marked
true twice, since that would mean that a conflict between the pop methods was
not resolved correctly.
\item testConcurrent. Again, 16 threads are spawned and each thread first pushes
a value into our DEQueue. After that, all threads are in competition to pop the
values. The idea is that some threads will call \textit{popTop()} to steal from
other's queues. However, at the end we should still see the same invariant as in
the previous test case.
\end{itemize}
\par
%%%%%%%%%%%%%%%%%%%%%%%%%%%%%%%%%%
\subsection{Sample Results and Interpretation}
\par
The result of the execution of the test cases was as follows:
\par
\hfill
\begin{verbatim}
[oraadm@gdlaa008 Steal]$ junit steal.BDEQueueTest
.sequential pushBottom and popBottom
.concurrent pushBottom and popBottom

Time: 0.055

OK (2 tests)
\end{verbatim}
\hfill
\par
The tests passed every time
%%%%%%%%%%%%%%%%%%%%%%%%%%%%%%%%%%
% --------------------------- %
% BDEQueueTest_Revisited End
% --------------------------- %
