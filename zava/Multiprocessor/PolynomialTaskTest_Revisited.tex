% --------------------------- %
% PolynomialTaskTest_Revisited Start
% --------------------------- %
\section{\textbf{PolynomialTaskTest}}
%%%%%%%%%%%%%%%%%%%%%%%%%%%%%%%%%%
\subsection{Particular Case}
\par
The particular problem we want to solve in this exercise is how can we implement
an efficient polynomial parallel adder and multiplier.
\par
%%%%%%%%%%%%%%%%%%%%%%%%%%%%%%%%%%
\subsection{Solution}
\par
The solution to this problem requires the use of a pool of threads, which are
called \textit{executor services} in Java.  In this case, we will use Runnable
objects. These Runnable objects return what are called Futures. Futures of
Runnables provide the \textit{get()} method. This method blocks the caller till
the computation finishes. 
\par
Let us examine the code to understand how the Futures are used to solve the
Polynomial problem.
\par
Here is the \textit{MulTask} Runnable. It requires 3 references to polynomials:
the multipliers and the result. When it is executed, it will split each of the
multipliers into two, say $p_1$, $p_2$ and $q_1$, $q_2$. The concatenation of
$p_1$ and $p_2$ is the original $p$ polynomial. Given this partition, the result
of a multiplication should be $p_1*q_1 + p_1*q_2 + p_2*q_1 + p_2*q_2$. So, the
task of multiplying two polynomials is decomposed into $4$ multiplications and
$3$ sums. 
\par
\hfill
\begin{lstlisting}[style=numbers]
  static class MulTask implements Runnable {
    Polynomial p, q, r;
    Polynomial[] temp;
    public MulTask(Polynomial p, Polynomial q, Polynomial r) {
      this.p = p; this.q = q; this.r = r;
      int newDegree = 2 * p.getDegree();
      temp = new Polynomial[6];
      for (int i  = 0; i < temp.length; i++) {
        temp[i] = new Polynomial(newDegree);
      }
    }
    public void run() {
      try {
        if (p.getDegree() == 0) {
          r.set(0, p.get(0) * q.get(0));
        } else {
          Polynomial[] pp = p.split();
          Polynomial[] qq = q.split();
          Future<?>[] future = (Future<?>[]) new Future[7];
          // launch parallel multiplications
          future[0] = exec.submit(new MulTask(pp[0], qq[0], temp[0]));
          future[1] = exec.submit(new MulTask(pp[0], qq[1], temp[1]));
          future[2] = exec.submit(new MulTask(pp[1], qq[0], temp[2]));
          future[3] = exec.submit(new MulTask(pp[1], qq[1], temp[3]));
          // wait for them to finish
          for (int i = 0; i < 4; i++)
            future[i].get();
          // do partial sums
          future[5] = exec.submit(new AddTask(temp[0], temp[1], temp[4]));
          future[6] = exec.submit(new AddTask(temp[2], temp[3], temp[5]));
          // wait for them to finish
          future[5].get();
          future[6].get();
          // final sum
          future[7] = exec.submit(new AddTask(temp[4], temp[5], r));
          future[7].get();
        }
      } catch (Exception ex) {
        ex.printStackTrace();
      }
    }
  }
\end{lstlisting}
\hfill
\par
Here we have now the \textit{AddTask}. This Runnable is simpler. It again
requires $3$ references to polynomials. The execution consists of spliting each
polynomial into two. After that, sub-polynomials are added one to one. 
\par
\hfill
\begin{lstlisting}[style=numbers]
  static class AddTask implements Runnable {
    Polynomial p, q, r;
    public AddTask(Polynomial p, Polynomial q, Polynomial r) {
      this.p = p; this.q = q; this.r = r;
    }
    public void run() {
      try {
        int n = p.getDegree();
        if (n == 1) {
          r.set(0, p.get(0) + q.get(0));
        } else {
          Polynomial[] pp = p.split(), qq = q.split(), rr = r.split();
          Future<?>[] future = (Future<?>[]) new Future[2];
          // create asynchronous computations
          future[0] = exec.submit(new AddTask(pp[0], qq[0], rr[0]));
          future[1] = exec.submit(new AddTask(pp[1], qq[1], rr[1]));
          // wait for them to finish
          future[0].get();
          future[1].get();
        }
      } catch (Exception ex) {
        ex.printStackTrace();
      }
    }
  }
\end{lstlisting}
\hfill
\par
%%%%%%%%%%%%%%%%%%%%%%%%%%%%%%%%%%
\subsection{Experiment Description}
\par
Now let us discuss the experiment. It is actually a sum of the
polynomial $x^7+x^6+x^5+x^4+x^3+x^2+x+1$ by itself. The result of this
sum should be $2x^7+2x^6+2x^5+2x^4+2x^3+2x^2+2x+2$. 
\par
%%%%%%%%%%%%%%%%%%%%%%%%%%%%%%%%%%
\subsection{Sample Results}
\par
\par
%%%%%%%%%%%%%%%%%%%%%%%%%%%%%%%%%%
\subsection{Interpretation}
\par
\par
%%%%%%%%%%%%%%%%%%%%%%%%%%%%%%%%%%
\subsection{Proposed solution}
\par
\par
%%%%%%%%%%%%%%%%%%%%%%%%%%%%%%%%%%
% --------------------------- %
% PolynomialTaskTest_Revisited End
% --------------------------- %
