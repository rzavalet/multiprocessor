\section{\textbf{Coarse List}}
% ---------------------------------------------------------------------------- %
\subsection{Particular Case}
\par
In this exercise we are trying to solve the problem of accessing a List data
structure by multiple threads simultaneously. While allowing multiple threads
accessing the data structure, we still want to maintain its consistency. 
\par
As mentioned in the book, the invariants that we want to satisfy are:
\par
\begin{enumerate}
\item Sentinels are neither added or removed (ie. the head and the tail)
\item Nodes are sorted by key
\item Keys are unique
\end{enumerate}
\par
% ---------------------------------------------------------------------------- %
\subsection{Solution}
\par
Our first approach to solving this problem is then through a coarse grained
list. This means that we use a lock in such a way that any access to the list is
sequential because this lock must be hold and there is only one lock.
\par
Let's take a quick look at the add and remove items to grasp the idea behind
this algorithm. Notice how the whole logic of the \textit{add()} and
\textit{remove()} operations are guarded by the calls to \textit{lock()} and
\textit{unlock()}.
\par
\hfil
\begin{lstlisting}[style=numbers]
  /** 
   * Add an element.
   * @param item element to add
   * @return true iff element was not there already
   */
  
  public boolean add(T item) {
    Node pred, curr;
    int key = item.hashCode();
    lock.lock();
    try {
      pred = head;
      curr = pred.next;
      while (curr.key < key) {
        pred = curr;
        curr = curr.next;
      }   
      if (key == curr.key) {
        return false;
      } else {
        Node node = new Node(item);
        node.next = curr;
        pred.next = node;
        return true;
      }   
    } finally {
      lock.unlock();
    }   
  }

  /** 
   * Remove an element.
   * @param item element to remove
   * @return true iff element was present
   */
  public boolean remove(T item) {
    Node pred, curr;
    int key = item.hashCode();
    lock.lock();
    try {
      pred = this.head;
      curr = pred.next;
      while (curr.key < key) {
        pred = curr;
        curr = curr.next;
      }   
      if (key == curr.key) {  // present
        pred.next = curr.next;
        return true;
      } else {
        return false;         // not present
      }   
    } finally {               // always unlock
      lock.unlock();
    }   
  }
\end{lstlisting}
\hfill
\par
% ---------------------------------------------------------------------------- %
\subsection{Experiment Description}
\par
We have four test cases:
\par
\begin{enumerate}
\item \textit{testSequential()}. The main thread inserts multiple elements to
the list, then checks if all elements are there and finally removes all elements
one by one.
\item \textit{testParallelAdd()}. The main thread spawns multiple threads that
add elements to the list concurrently. When they are done, the main thread
checks if all elements are in there and finally removes all elements.
\item \textit{testParallelRemove()}. The main thread inserts multiple elements
to the list, then checks if all elements are in there and finally it spawns
multiple threads that remove the elements from the list concurrently.
\item \textit{testParallelBoth()}. The main thread creates multiple threads.
Some of them insert elements to the list and some of them remove elements.
\end{enumerate}
\par
% ---------------------------------------------------------------------------- %
\subsection{Sample Results}
\par
For this test, we saw that it failed consistently with the following error:
\begin{verbatim}
[oraadm@gdlaa008 Lists]$ junit lists.CoarseListTest
.sequential add, contains, and remove
.parallel both
Exception in thread "Thread-3" Exception in thread "Thread-5" Exception in
thread "Thread-9" Exception in thread "Thread-15"
junit.framework.AssertionFailedError: RemoveThread: duplicate remove: 32
        at junit.framework.Assert.fail(Assert.java:57)
        at junit.framework.TestCase.fail(TestCase.java:227)
        at lists.CoarseListTest$RemoveThread.run(CoarseListTest.java:145)
junit.framework.AssertionFailedError: RemoveThread: duplicate remove: 121
        at junit.framework.Assert.fail(Assert.java:57)
        at junit.framework.TestCase.fail(TestCase.java:227)
        at lists.CoarseListTest$RemoveThread.run(CoarseListTest.java:145)
junit.framework.AssertionFailedError: RemoveThread: duplicate remove: 17
        at junit.framework.Assert.fail(Assert.java:57)
        at junit.framework.TestCase.fail(TestCase.java:227)
        at lists.CoarseListTest$RemoveThread.run(CoarseListTest.java:145)
junit.framework.AssertionFailedError: RemoveThread: duplicate remove: 66
        at junit.framework.Assert.fail(Assert.java:57)
        at junit.framework.TestCase.fail(TestCase.java:227)
        at lists.CoarseListTest$RemoveThread.run(CoarseListTest.java:145)
.parallel add
.parallel remove

Time: 0.017

OK (4 tests)
\end{verbatim}
\par
So, the test that fails consistently is \textit{parallel both}. It complains
saying that an specific element was removed multiple times by one or more
threads.
\par
%%%%%%%%%%%%%%%%%%%%%%%%%%%%%%%%%%
\subsection{Proposed solution}
\par
This failure turns out to be a test issue. Each remover thread is in charge of
removing a specific set of elements, say from $25$ to $50$. In the same way, an
inserter thread is in charge of adding a specific set of elements. Since the
threads for a given interval are spawned concurrently, it might be the case that
a remover thread is trying to remove an element that hasn't been inserted yet. 
\par
One way of solving this problem is by asking the remover thread to retry till it
is successful. (Obviously this fix should be improved further since it might
lead to a hang in user's application -- Imagine the case where an specific
element is never inserted. However, for this specific case, the fix is good
enought.)
\par
This is the code of the remover thread before the fix:
\par
\hfill
\begin{lstlisting}[style=numbers]
  class removethread extends thread {
    int value;
    removethread(int i) {
      value = i;
    }   
    public void run() {
      for (int i = 0; i < per_thread; i++) {
        if (!instance.remove(value + i)) {
          fail("removethread: duplicate remove: " + (value + i));
        }
      }   
    }   
  }
\end{lstlisting}
\hfill
\par
And here is the code with the fix:
\par
\hfill
\begin{lstlisting}[style=numbers]
  class removethread extends thread {
    int value;
    removethread(int i) {
      value = i;
    }   
    public void run() {
      for (int i = 0; i < per_thread; i++) {
        while (!instance.remove(value + i)) {}
      }   
    }   
  }
\end{lstlisting}
\hfill
\par
And here is the test output after the fix:
\par
\begin{verbatim}
[oraadm@gdlaa008 Lists]$ junit lists.CoarseListTest
.sequential add, contains, and remove
.parallel both
.parallel add
.parallel remove

Time: 0.016

OK (4 tests)
\end{verbatim}
% ---------------------------------------------------------------------------- %
\subsection{Interpretation}
% ---------------------------------------------------------------------------- %
In this experiment we observed how to implement a simple list that allows
concurrent access by making all the access sequential. We saw that eventhough
the algorithm is correct, a correct usage of the API is also required.
