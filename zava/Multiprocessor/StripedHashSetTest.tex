% --------------------------- %
% StripedHashSetTest Start
% --------------------------- %
\section{\textbf{StripedHashSetTest Test}}
%%%%%%%%%%%%%%%%%%%%%%%%%%%%%%%%%%
\subsection{Particular Case}
\par
In this exercise we deal with the problem of building a hash data structure that
can be accessed by multiple threads without losing performance. 
\par
In the coarse version of our hash, we saw that any access to the data structure
ended up being serial. So, in other words, in this experiment we want to
understand how to create a hash that allows more concurrent accesses.
\par
%%%%%%%%%%%%%%%%%%%%%%%%%%%%%%%%%%
\subsection{Solution}
\par
A solution for this problem is the \textit{Striped Hash}. The main
characteristic of this implementation is that the data structure contains an
array of locks. Each lock protect specific sections of the hash table. Here is
the implementation of the \textit{acquire()} and \textit{release()} methods:
\par
\hfill
\begin{lstlisting}[style=numbers]
  /**
   * Synchronize before adding, removing, or testing for item
   * @param x item involved
   */
  public final void acquire(T x) {
    int myBucket = Math.abs(x.hashCode() % locks.length);
    locks[myBucket].lock();
  }

  /**
   * synchronize after adding, removing, or testing for item
   * @param x item involved
   */
  public void release(T x) {
    int myBucket = Math.abs(x.hashCode() % locks.length);
    locks[myBucket].unlock();
  }
\end{lstlisting}
\hfill
\par
Note that when we acquire or release a lock, first we need to choose which lock
we should work with. 
\par
%%%%%%%%%%%%%%%%%%%%%%%%%%%%%%%%%%
\subsection{Experiment Description}
\par
The proposed experiment is exactly the same as the one described in the previous
experiment. As mentioned before, there are four variants:
\textit{testSequential}, \textit{testParallelEnq}, \textit{testParallelRemove}
and \textit{testParallelBoth}.
\par
%%%%%%%%%%%%%%%%%%%%%%%%%%%%%%%%%%
\subsection{Sample Results}
\par
\par
%%%%%%%%%%%%%%%%%%%%%%%%%%%%%%%%%%
\subsection{Interpretation}
\par
\par
%%%%%%%%%%%%%%%%%%%%%%%%%%%%%%%%%%
\subsection{Proposed solution}
\par
\par
%%%%%%%%%%%%%%%%%%%%%%%%%%%%%%%%%%
% --------------------------- %
% StripedHashSetTest End
% --------------------------- %
