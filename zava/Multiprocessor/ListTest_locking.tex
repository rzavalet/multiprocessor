% --------------------------- %
% ListTest_locking Start
% --------------------------- %
\section{\textbf{List (Lock-Based) Test}}
%%%%%%%%%%%%%%%%%%%%%%%%%%%%%%%%%%
\subsection{Particular Case}
\par
Here we are again experimenting with the STM implementation of Herlihy, which is
called TinyTM. In this case, we are implementing a concurrent list data
structure using STM. 
\par
However, this time we are using another approach. Here we are going to use
Lock-Based Atomic objects.
\par
%%%%%%%%%%%%%%%%%%%%%%%%%%%%%%%%%%
\subsection{Solution}
\par
One problem with the obsctruction-free implementation that has been described in
another experiment, is that writes allocate locators continously. Also, reads
have to go through two levels of indirection. 
\par
The Lock-Based approach is called so because it locks the object when in
performs a write. When writing to the object, the \textit{openWrite()} method
will create a new tentative version and adds it to the so called write set.
\par
Reading an object on the other hand is done optimistically and then it checks
for conflicts. These conflicts are detected by using a version clock. The
\textit{openRead()} method will check whether the object is in the write set. If
it is, then it creates a new tentative version. If not, it checks whether the
object is locked. If so, the transaction aborts, otherwise the version is added
to the read set.
\par
As for the commit, the following steps are performed:
\par
\begin{enumerate}
\item Lock the objects in the write set
\item Increment the global version clock in a thread-local stamp
\item The transaction checks that objects in the read set is not locked by
another thread and that eachs object's timestamp is not greater than the
transaction's read stamp. If this validation succeds, then it commits.
\item Update the stamp field of each object in the write set.
\item Release any lock.
\end{enumerate}
\par
For using these Atomic objects in our list implementation, all we have to do is
us the TNode class which internally uses the LockObject version of the
AtomicObject interface.
\par
%%%%%%%%%%%%%%%%%%%%%%%%%%%%%%%%%%
\subsection{Experiment Description}
\par
Three test cases are provided for this exercise:
\begin{itemize}
\item testParallel. $1024$ elements are added initially to the list, then 32
threads are spawned. On even turns, the threads will insert a new element to the
list. On odd turns, the thread removes an element from the list. The test checks
that the values in the list at the end of the test are in order, that there are
no duplicate values, that the number of commits matches the number of operations
on the list, and that the size of the list is as expected.
\item testSequential. $1024$ elements are added initially to the list. Then a
thread is spawned with the same behaviour for even and odd turns described
above. After $6$ seconds, the thread is interruped and the same check as above
is performed.
\item testIterator. It adds $100$ consecutive elements to the list and then it
checks that the list added these elements in the correct order and that the
number of elements in the list is as expected.
\end{itemize}
\par
%%%%%%%%%%%%%%%%%%%%%%%%%%%%%%%%%%
\subsection{Sample Results}
\par
\par
%%%%%%%%%%%%%%%%%%%%%%%%%%%%%%%%%%
\subsection{Interpretation}
\par
\par
%%%%%%%%%%%%%%%%%%%%%%%%%%%%%%%%%%
\subsection{Proposed solution}
\par
\par
%%%%%%%%%%%%%%%%%%%%%%%%%%%%%%%%%%
% --------------------------- %
% ListTest_locking End
% --------------------------- %
