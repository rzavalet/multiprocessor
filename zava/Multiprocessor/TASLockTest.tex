\section{\textbf{Test-And-Set Locks}}
% ------------------------------------------------------------------------ %
\subsection{Particular Case}
\par
The particular case we are trying to solve is that of mutual exclusion with two
participants. The problem with previous approaches for solving this problem
(e.g. Peterson Algorithm) is that in the real world the assumption of having a
``sequential consistent memory'' and a ``program order guarantee among
reads-writes by a given thread'' are not true.
\par
There are at least two reasons for that. The first one is that compilers, in
order to improve performance of the program, can in some cases reorder certain
operations . As the book clearly mentions, most programming languages preserve
execution order that affect individual variables, but not across variables. This
means that reads and/or writes of different variables can end up in a different
order that the one appearing in the source code once the program is compiled or
executed. This can clearly affect our Peterson Algorithm.
\par
The second reason that has to do with the contradiction of sequential consistent
memory is due to the hardware design which in most multiprocessor systems allow
writes to be written to store buffers. This buffers are written to memory only
when needed. Hence, race conditions arise, where one variable might be read
before another because the latter was in a store buffer.
\par
The generic way to solve this problem is by using instructions called a
\textit{memory barrier} or \textit{memory fence}. These instructions are able to
prevent reordering operations due to write buffering. The idea is that these
instructions can be called in specific parts of the code to guarantee our
proposed order of execution.
\par
The problem in this section is \textit{how can we implement an actually working
mutual exclusion algorithm with consensus number of two using memory barrier}?
\par
% ------------------------------------------------------------------------ %
\subsection{Solution}
\par
One solution for this problem is using a synchronization instruction that
includes memory barriers. For example, we can use the $testAndSet()$ operation
which has a consensus number of two and which atomically stores $true$ in a
variable an returns the previously stored value. 
\par
To design an algorithm using this instruction the idea is the following. A value
of false is used to indicate that a lock is free. Then, we store a value of true
to indicate that we are using the lock. Once we are done, we can set the value
back to false to allow other threads take the lock.
\par
In the java code, the $testAndSet()$ is done with the $getAndSet()$ method. And
setting the value to false is achieved with the method $set()$.
\par
% ------------------------------------------------------------------------ %
\subsection{Experiment Description}
\par
To demonstrate that this implementation works, the test that was provided does
the following:
\begin{enumerate}
\item Initiate a shared counter with a value of $0$
\item Start $8$ threads
\item Each thread has to increment the counter by one $1024$ times
\item At the end, the counter must hold a value of $8*1024$
\end{enumerate}
\par
These are the details of the system we used to run the experiments:
\begin{itemize}
\item Processor: Intel Core i5 @2.5 GHz. 2 Cores.
\item L2 Cache per Core: 256 KB
\item L3 Cache: 3 MB
\item System Memory: 16 GB
\end{itemize}
% ------------------------------------------------------------------------ %
\subsection{Sample Results}
\par
For this test, we saw that in every try, it always passed.
\par
\begin{figure}[h]
  \centering
  \includegraphics[width=13cm]{TAS00.png}
  \caption{Successful execution of the TAS Lock test}
  \label{fig:TAS00}
\end{figure}
\par
\begin{figure}[h]
  \centering
  \includegraphics[width=13cm]{TAS01.png}
  \caption{Successful execution of the tests  for TAS Lock}
  \label{fig:TAS01}
\end{figure}
\par
% ------------------------------------------------------------------------ %
\subsection{Interpretation}
It was shown that the algorithm seems to work and allows synchronization of two
threads (consensus number was two) to access a shared resource.
