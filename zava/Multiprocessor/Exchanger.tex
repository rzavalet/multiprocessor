\section{\textbf{Exchanger}}
% ---------------------------------------------------------------------------- %
\subsection{Particular Case}
\par
The problem we are trying to solve is how to allow two threads exchange values
they hold. The idea is that if a Thread A calls an $exchange()$ method with a given
argument and thread B calls the $exchange()$ method with another arguments, then
thread A will return B's argument and B will return A's argument.
\par
Let us discuss a little bit why such method is required. The motivation of this
comes from the problem of implementing a concurrent queue. Imagine that two
threads are trying to access the queue and one of them wants to enqueue an
element and the other one wants to dequeue an element. In this situation, we can
avoid contention in the top of the stack by letting the threads simply exchange
their values. That is, the enqueuer can pass its value to the dequeuer thread.
This approach is called the \textit{Elimination back-off stack}.
\par
% ---------------------------------------------------------------------------- %
\subsection{Solution}
\par
The solution in this exercise is based on the \textit{A Scalable
Elimination-based Exchange Channel} paper. Since, this algorithm is a bit more
involved and it is not discussed in the book, let us describe it in more detail.
\par
The algorithm uses an array of \textit{AtomicReference}s. Note that the size of
this array depends on the number of processors in the system. Actually, it has
to be of half the number of processors. The reason for this is that if we have
$n$ threads, then we can only have up to $n/2$ concurrent exchanges and each
slot in the array is used for an exchange.
\par
\hfill
\begin{lstlisting}[style=numbers]
public class Exchanger<V> {
  private static final int SIZE =
      (Runtime.getRuntime().availableProcessors() + 1) / 2;
  private static final long BACKOFF_BASE = 128L;
  static final Object FAIL = new Object();
  private final AtomicReference[] arena;
  private final Random random;
  public Exchanger() {
    arena = new AtomicReference[SIZE + 1];
    random = new Random();
    for (int i = 0; i < arena.length; ++i)
      arena[i] = new AtomicReference();
  }
\end{lstlisting}
\hfill
\par
Now, from our program we can call the \textit{exchange()} methods in two ways.
One in which we can specify a timeout for the exchange operation and another
where there is no timeout.
\par
\hfill
\begin{lstlisting}[style=numbers]
  public V exchange(V x) throws InterruptedException {
    try {
      return (V)doExchange(x, Integer.MAX_VALUE);
    } catch (TimeoutException cannotHappen) {
      throw new Error(cannotHappen);
    }
  }
  public V exchange(V x, long timeout, TimeUnit unit)
  throws InterruptedException, TimeoutException {
    return (V)doExchange(x, unit.toNanos(timeout));
  }
\end{lstlisting}
\hfill
\par
The interesting method is \textit{doExchange()}. Imagine that a thread first
tries to do an exchange. So it calls this method and gets into the whie loop. It
takes the first slot in the arena and finds it is not occupied and hence it will
try to occupy the slot setting its id in the slot using a \textit{compareAndSet}
operation. Then it will sit there and wait for someone else to come and exchange
its value in the \textit{waitForHole()} call.
\par
Now imagine that another thread comes and tries to exchange its value with the
previous thread. It then enters the loop, looks into the first slot and sees
that someone has put a request in that slot. In other words, some other thread
is willing to exchange a value. To do this, it fills the hole with the requested
value and the other thread is signaled to continue. Our current thread then
returns with the value of the other thread.
\par
The first thread, as we mentioned, gets signaled and sets the slot to null to
indicate that the exchange has finished and returns with the value of the other
thread. 
\par
\hfill
\begin{lstlisting}[style=numbers]
  private Object doExchange(Object item, long nanos)
      throws InterruptedException, TimeoutException {
    Node me = new Node(item);
    long lastTime = System.nanoTime();
    int idx = 0;
    int backoff = 0;
    while (true) {
      AtomicReference<Node> slot = (AtomicReference<Node>)arena[idx];
      // If this slot is occupied, an item is waiting...
      Node you = slot.get();
      if (you != null) {
        Object v = you.fillHole(item);
        slot.compareAndSet(you, null);
        if (v != FAIL) // ... unless it's cancelled
          return v;
      }
      // Try to occupy this slot
      if (slot.compareAndSet(null, me)) {
        Object v = ((idx == 0)?
          me.waitForHole(nanos) :
          me.waitForHole(randomDelay(backoff)));
        slot.compareAndSet(me, null);
        if (v != FAIL)
          return v;
        if (Thread.interrupted())
          throw new InterruptedException();
        long now = System.nanoTime();
        nanos -= now - lastTime;
        lastTime = now;
        if (nanos <= 0)
          throw new TimeoutException();
        me = new Node(item);
        if (backoff < SIZE - 1)
          ++backoff;
        idx = 0; // Restart at top
      } else // Retry with a random non-top slot <= backoff
        idx = 1 + random.nextInt(backoff + 1);
    }
  }
  ...
\end{lstlisting}
\hfill
\par
The algorithm adds some other improvements to the base case already described.
The main improvement is the back-off logic which comes to play when two threads
are already exchanging their values in a slot. In that case, the thread will try
using another slot.
% ---------------------------------------------------------------------------- %
\subsection{Experiment Description}
\par
The experiment here creates an array of integers of size $THREADS$. It then
spawns $THREADS$ threads. Each thread will exchange its thread id with another
one and will record this value in the array. For example, suppose that thread 1
and thread 2 are the participants. In this case thread 2 will store $array[2]=1$
and thread 1 will store $array[1]=2$. The condition that the test must satisfy
is $i \neq array[array[i]]$. In this case $1 \neq array[array[1]]$. 
\par
% ---------------------------------------------------------------------------- %
\subsection{Sample Results}
\par
For this test, we observed that the test always passes:
\par
\begin{verbatim}
.exchange
OK

Time: 0.007

OK (1 test)
\end{verbatim}
\par
% ---------------------------------------------------------------------------- %
\subsection{Interpretation}
% ---------------------------------------------------------------------------- %
We see that the proposal of Scherer et. al works correctly. As we mentioned
previously, this algorithm is very important if we want to implement a
concurrent queue because this exchanger mechanism allows for what is called
\textit{Elimination}.
