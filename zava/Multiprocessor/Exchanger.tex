\section{\textbf{Exchanger}}
% ---------------------------------------------------------------------------- %
\subsection{Particular Case}
\par
The problem we are trying to solve is how to allow two threads exchange values
they hold. The idea is that if a Thread A calls an $exchange()$ method with a given
argument and thread B calls the $exchange()$ method with another arguments, then
thread A will return B's argument and B will return A's argument.
\par
% ---------------------------------------------------------------------------- %
\subsection{Solution}
\par
The solution in this excercise is based on the \textit{A Scalable
Elimination-based Exchange Channel} paper. 
\par
% ---------------------------------------------------------------------------- %
\subsection{Experiment Description}
\par
The experiment here creates an array of integers of size $THREADS$. It then
spawns $THREADS$ threads. Each thread will exchange its thread id with another
one and will record this value in the array. For example, suppose that thread 1
and thread 2 are the participants. In this case thread 2 will store $array[2]=1$
and thread 1 will store $array[1]=2$. The condition that the test must satisfy
is $i \neq array[array[i]]$. In this case $1 \neq
array[array[1]]$. 
\par
% ---------------------------------------------------------------------------- %
\subsection{Sample Results}
\par
For this test, we observed that the test always passes:
\par
\begin{verbatim}
.exchange
OK

Time: 0.007

OK (1 test)
\end{verbatim}
\par
% ---------------------------------------------------------------------------- %
\subsection{Interpretation}
% ---------------------------------------------------------------------------- %
We see that the proposal of Scherer et. al works correctly.
