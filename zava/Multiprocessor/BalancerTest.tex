% --------------------------- %
% BalancerText Start
% --------------------------- %
\section{\textbf{Balancer Test}}
%%%%%%%%%%%%%%%%%%%%%%%%%%%%%%%%%%
\subsection{Particular Case}
\par
The problem we are dealing with in this experiment is that of creating a 
balancer class.
\par
As stated in the text book, a \textit{Balancer} is a simple switch with two
input wires and two output wires (top and bottom wires). The token arrives to
the input lines and the balancer must output the input one at a time.
\par
A balancer has two states: \textit{up} and \textit{down}. If the state is
\textit{up}, the next token exits on the top of the wire. Otherwise, it exits
on the bottom wire.
\par
%%%%%%%%%%%%%%%%%%%%%%%%%%%%%%%%%%
\subsection{Solution}
\par
The following is a simple implementation of a Balancer
class. As described above, the balancer consists of a
toggle. 
\par
\hfill
\begin{lstlisting}[style=numbers]
  public synchronized int traverse(int input) {
    try {
      if (toggle) {
        return 0;
      } else {
        return 1;
      }
    } finally {
      toggle = !toggle;
    }
  }
\end{lstlisting}
\hfill
\par
As we can observe, the traverse method simply switches the state of the toggle
and outputs either $0$ or $1$. 
%%%%%%%%%%%%%%%%%%%%%%%%%%%%%%%%%%
\subsection{Experiment Description}
\par
The balancer class described above is used in one test case. The test case
consists of a 2-slot array. Each slot keeps track of how many times the
balancer has output a signal on the \textit{up} wire and how many on the
\textit{down} wire. 
\par
So, the test activates the balancer exactly $256$ times. The final result must be that each slot has the number $128$. If that is the case, we declare the test as passed, otherwise it failed.
\par
%%%%%%%%%%%%%%%%%%%%%%%%%%%%%%%%%%
\subsection{Sample Results and Interpretation}
\par
Here is  the result of the execution. It passed every time:
\par
\hfill
\begin{verbatim}
[oraadm@gdlaa008 Counting]$ junit counting.BalancerTest
.
Time: 0.002

OK (1 test)
\end{verbatim}
\hfill
%%%%%%%%%%%%%%%%%%%%%%%%%%%%%%%%%%
% --------------------------- %
% BalancerText End
% --------------------------- %
