% --------------------------- %
% SkipListSetTest Start
% --------------------------- %
\section{\textbf{SkipList (Obstruction-Free) Test}}
%%%%%%%%%%%%%%%%%%%%%%%%%%%%%%%%%%
\subsection{Particular Case}
\par
In this exercise we are experimenting with the STM implementation mentioned in
the book, which is called TinyTM. 
\par
The problem we are trying to solve is how to implement a concurrent Skip List by
using STM with Obstruction-Free Atomic Objects.
\par
%%%%%%%%%%%%%%%%%%%%%%%%%%%%%%%%%%
\subsection{Solution}
\par
Since the solution consists pretty much on using the TinyTM package and it has
been described before (both in the Obstruction-Free and Lock-Based approach) we
will not do it again here.
\par
Please refer to previous exercises in this chapter for a better understanding of
how TinyTM works.
\par
%%%%%%%%%%%%%%%%%%%%%%%%%%%%%%%%%%
\subsection{Experiment Description}
\par
In this exercise, three test cases are provided:
\begin{itemize}
\item testParallel. $1024$ elements are added initially to the list, then 32
threads are spawned. On even turns, the threads will insert a new element to the
list. On odd turns, the thread removes an element from the list. The test checks
that the values in the list at the end of the test are in order, that there are
no duplicate values, that the number of commits matches the number of operations
on the list, and that the size of the list is as expected.
\item testSequential. $1024$ elements are added initially to the list. Then a
thread is spawned with the same behaviour for even and odd turns described
above. After $6$ seconds, the thread is interruped and the same check as above
is performed.
\item testIterator. It adds $100$ consecutive elements to the list and then it
checks that the list added these elements in the correct order and that the
number of elements in the list is as expected.
\end{itemize}
\par
%%%%%%%%%%%%%%%%%%%%%%%%%%%%%%%%%%
\subsection{Sample Results}
\par
\par
%%%%%%%%%%%%%%%%%%%%%%%%%%%%%%%%%%
\subsection{Interpretation}
\par
\par
%%%%%%%%%%%%%%%%%%%%%%%%%%%%%%%%%%
\subsection{Proposed solution}
\par
\par
%%%%%%%%%%%%%%%%%%%%%%%%%%%%%%%%%%
% --------------------------- %
% SkipListSetTest End
% --------------------------- %
