\section{\textbf{Safe Boolean MRSW Register}}
% ------------------------------------------------------------------------------------ %
\subsection{Particular Case}
\par
In class we saw the difference between Safe, Regular and Atomic Registers. In
this exercise we are experimenting with a Safe Boolean Register. In particular,
we are trying a register that allows one single writer whose written value can be
examined by multiple readers. We need to remember that Safe registers are those
that :
\begin{itemize}
\item If a read does not overlap with a write, then the read returns the last
written value
\item If a read overlaps with a write, then the read can return any value in the
domain of the register's values
\end{itemize}
% ------------------------------------------------------------------------------------ %
\subsection{Solution}
\par
One way to achieve this behavior is using a SRSW Safe register. The code that is
provided with the book does this by having a boolean SRSW register per thread,
in other words, it is an array of boolean values. When doing a write, the writer
iterates over this array and writes the new value. The readers simply access its
assigned slot in the array (based on the thread id) and return the stored
value. The code for this type of register is shown next:
\par
\begin{lstlisting}[style=numbers]
  public Boolean read() {
    return s_table[ThreadID.get()];
  }

  public void write(Boolean x) {
    for (int i = 0; i < s_table.length; i++)
      s_table[i] = x;
  }
\end{lstlisting}
\hfill
\par
Please note how this type of register does not make use of any synchronization
mechanism.
% ------------------------------------------------------------------------------------ %
\subsection{Experiment Description}
\par
This program provides two test cases. The first one is a sequential test and the
second one is a parallel test. The former simply calls a write and then a read
from the same thread. This is not rocket science. The reader must retrieve what
the writer wrote.
\par
The second test case is slightly more interesting. The writer writes first one
value and then another one. After that, 8 reader threads are started and they
read the last written value. According to the rules of a safe register, all
threads must read the same value because the reads and the writes did not
overlap.
\par
% ------------------------------------------------------------------------------------ %
\subsection{Sample Results}
\par
We found out that the tests failed occasionally. The manifestation of the failure is a 
\par
\textit{indexOutOfBoundsException} as shown above.
\begin{verbatim}
[oraadm@gdlaa008 Register]$ junit register.SafeBooleanMRSWRegisterTest
.sequential read and write
.parallel read
Exception in thread "Thread-7" java.lang.ArrayIndexOutOfBoundsException: 8
        at register.SafeBooleanMRSWRegister.read(SafeBooleanMRSWRegister.java:26)
        at register.SafeBooleanMRSWRegisterTest$ReadThread.run(SafeBooleanMRSWRegisterTest.java:62)

Time: 0.008

OK (2 tests)
\end{verbatim}
\par
So, the way to fix this problem was to make sure that in each test case
the threads ids are reset calling the static method $ThreadID.reset()$. With
this hack, the tests passed.
\par
\begin{lstlisting}[style=numbers]
  /**
   * Sequential calls.
   */
  public void testSequential() {
    ThreadID.reset(); // We have to reset the thread ids
    System.out.println("sequential read and write");
    instance.write(true);
    boolean result = instance.read();
    assertEquals(result, true);
  }

  /**
   * Parallel reads
   */
  public void testParallel() throws Exception {
    ThreadID.reset(); // We have to reset the thread ids
    instance.write(false);
    instance.write(true)
    System.out.println("parallel read"
    for (int i = 0; i < THREADS; i++) {
      thread[i] = new ReadThread(
    }
    for (int i = 0; i < THREADS; i ++) {
      thread[i].start();
    }
    for (int i = 0; i < THREADS; i ++) {
      thread[i].join();
    }
  }  
\end{lstlisting}
\par
And here is the resulting execution after this fix:
\begin{verbatim}
[oraadm@gdlaa008 Register]$ junit register.SafeBooleanMRSWRegisterTest
.sequential read and write
.parallel read

Time: 0.004

OK (2 tests)
\end{verbatim}
\par
% ------------------------------------------------------------------------------------ %
\subsection{Interpretation}
\par
In this experiment we saw a way of implementing Safe Multi-Reader Single-Writer
registers. These registers were able of storing boolean values.
\par
Unfortunately, none of the test cases exercises the case where we have
concurrent writes and reads. Hence, this experiment only exercised one of the
two rules of a safe register.
% ------------------------------------------------------------------------------------ %
