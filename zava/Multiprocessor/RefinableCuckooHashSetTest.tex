% --------------------------- %
% RefinableCuckooHashSetTest Start
% --------------------------- %
\section{\textbf{RefinableCuckooHashSetTest Test}}
%%%%%%%%%%%%%%%%%%%%%%%%%%%%%%%%%%
\subsection{Particular Case}
\par
The particular problem we are trying to solve here is how to implement a
concurrent open-addressed hash (specifically a hash set).
\par
We ought to remember that an open-addressed hash table is one in which each
table entry holds one and only one item. In previous examples we had seen that
an entry in the table can handle multiple elements which are stored in a linked
list within the bucket.
\par
The other thing we have to consider is that in this exercise we are willing to
have a concurrent hash. Previously we saw a open-addressed hash that allowed
coarse granularity. Here we want to go beyond that.
\par
%%%%%%%%%%%%%%%%%%%%%%%%%%%%%%%%%%
\subsection{Solution}
\par
The solution to this problem is given by what is called the \textit{Refinable
Concurrent Cuckoo Hash}. The difference between this implementation and
previously presented \textit{Cuckoo Hashes} is that this one is a fine-grained
approach where a lock is always responsible for protecting a fixed number of
elements in the hash.
\par
The interesting bits of this implementation are the \textit{acquire()},
\textit{release()} and \textit{resize()} methods. Let us take a quick look at those:
\par
\hfill
\begin{lstlisting}[style=numbers]
  /**
   * Synchronize before adding, removing, or testing for item
   * @param x item involved
   */
  public void acquire(T x) {
    boolean[] mark = {true};
    Thread me = Thread.currentThread();
    Thread who;
    while (true) {
      do { // wait until not resizing
        who = owner.get(mark);
      } while (mark[0] && who != me);
      ReentrantLock[][] oldLocks = this.locks;
      ReentrantLock oldLock0 = oldLocks[0][hash0(x) % oldLocks[0].length];
      ReentrantLock oldLock1 = oldLocks[1][hash1(x) % oldLocks[1].length];
      oldLock0.lock();  // acquire locks
      oldLock1.lock();
      who = owner.get(mark);
      if ((!mark[0] || who == me) && this.locks == oldLocks) { // recheck
        return;
      } else {  //  unlock & try again
        oldLock0.unlock();
        oldLock1.unlock();
      }
    }
  }
\end{lstlisting}
\hfill
\par
The \textit{acquire()} method first spins till if finds that no one is doing a
resizing. This means that resizing and adding are exclusive operations. After
this step, the thread acquires the two locks that protect the intended slot in
the table. If after this point, the lock array is still the same we saw before
taking the lock, then we are free to continue. Otherwise we attempt to take the
locks again.
\par
The \textit{release()} method simply unlocks the slot: 
\par
\hfill
\begin{lstlisting}[style=numbers]
  /**
   * synchronize after adding, removing, or testing for item
   * @param x item involved
   */
  public final void release(T x) {
    Lock lock0 = locks[0][hash0(x) % locks[0].length];
    Lock lock1 = locks[1][hash1(x) % locks[1].length];
    lock0.unlock();
    lock1.unlock();
  }
\end{lstlisting}
\hfill
\par
The \textit{resize()} method is esentially the same as the one showed in the
previous exercise.
\par
%%%%%%%%%%%%%%%%%%%%%%%%%%%%%%%%%%
\subsection{Experiment Description}
\par
As for the tests provided for this experiment. They are the same discussed in
previous exercises in this chapter. Nothing new has been added, so let us go
directly to the results.
\par
%%%%%%%%%%%%%%%%%%%%%%%%%%%%%%%%%%
\subsection{Sample Results}
\par
\par
%%%%%%%%%%%%%%%%%%%%%%%%%%%%%%%%%%
\subsection{Interpretation}
\par
\par
%%%%%%%%%%%%%%%%%%%%%%%%%%%%%%%%%%
\subsection{Proposed solution}
\par
\par
%%%%%%%%%%%%%%%%%%%%%%%%%%%%%%%%%%
% --------------------------- %
% RefinableCuckooHashSetTest End
% --------------------------- %
