% --------------------------- %
% Count Down Latch Start
% --------------------------- %
\section{\textbf{Count Down Latch}}
%%%%%%%%%%%%%%%%%%%%%%%%%%%%%%%%%%
\subsection{Particular Case}
\par
The purpose of this experiment to our eyes is to demonstrate how monitor locks
and conditions are used. Let us remember both concepts before going further.
\par
The concept of monitor consists on encapsulating three components: data, methods
and synchronization mechanism. This allows to have our data structures (or
classes) take care of synchronization instead of overwhelming the user of the
API with this task. 
\par
Now, the concept of \textit{conditions} comes into play because with them we can
signal threads about an event. For example, instead of having threads spinning
waiting for a value to become false, we can instead signal threads and let them
know that they can now re-check for their locking condition.
\par
%%%%%%%%%%%%%%%%%%%%%%%%%%%%%%%%%%
\subsection{Solution}
\par
The java interface for \textit{Lock} suggests two methods to implement the
protocol drafted above. First, there is a method \textit{await()} which
basically asks the thread wait till it is signaled about an event. The
accompanying method \textit{signalAll} achieves this latter task.
\par
Let us explain how the implementation of the methods \textit{countDown()} and
\textit{await()} work:
\par
\hfill
\begin{lstlisting}[style=numbers]
    public void await() throws InterruptedException {
      lock.lock(); 
      try {
        while (counter > 0)
          condition.await();
      } finally {
        lock.unlock();
      }
    } 
      
    public void countDown() {
      lock.lock();
      try {
        counter--;
        if (counter == 0) {
          condition.signalAll();
        }
      } finally {
        lock.unlock();
      }
    }   
\end{lstlisting}
\hfill
\par
Basically, when we create an object of type \textit{CountDownLatch}, we
initialize a counter with a value. The goal of the object is to count down to
zero. To achieve this, we need to execute the method \textit{countDown()} a
number of times. 
\par
For example, one thread can create a \textit{CountDownLatch} object with a
counter of $10$ and then it sits to wait, i.e. calls the \textit{await()}
function. Other threads then call the method \textit{countDown} to decrement the
counter. When the counter reaches zero, then the first thread gets signaled.
\par
%%%%%%%%%%%%%%%%%%%%%%%%%%%%%%%%%%
\subsection{Experiment Description}
\par
This experiment is a bit different from others showed before. The idea in this
experiment is to show how the methods \textit{await()} and \textit{signalAll()}
are combined in a working program. 
\par
So, the idea of the test is the following. Initially we have 8 threads. We start
them as usual but we do not allow them to proceed normally. Instead, we will ask
them to wait for a signal. Internally, what we do is decrement a shared
variable. When this variable becomes 0, the signal is triggered. Initially this
signal is 1, meaning that we only need to decrement the variable once.
\par
After that, each thread will started and wait for the next signal. The variable
that controls this other signal is initially set to 8. Each thread will be in
charge of decrementing this variable by one. Once it becomes 0, the second
signal will be triggered announcing that our test must finish.
\par
Let us quickly review the test case:
\par
At the beginning we see the two \textit{CountDownLatch} objects. One is used to
start and one is used to finish. The one for start is initialized with 1 because
only one thread will trigger it. The one for start is initialized with 8 because
we will create 8 threads and each of them will call \textit{countDown()} once.
\par
\hfill
\begin{lstlisting}[style=numbers]
  ...
  static final int THREADS = 8; // number threads to test
  static final int START = 1492;
  static final int STOP  = 1776;
  CountDownLatch startSignal = new CountDownLatch(1);
  CountDownLatch doneSignal = new CountDownLatch(THREADS);
  ...
  public void testCountDownLatch() {                                                                                
    System.out.format("Testing %d threads\n", THREADS);                                                             
    CountDownLatch startSignal = new CountDownLatch(1);
    CountDownLatch doneSignal  = new CountDownLatch(THREADS);
    Thread[] thread = new Thread[THREADS];
    int[] log = new int[THREADS];
    // create threads
    for (int j = 0; j < THREADS; j++)
      thread[j] = new TestThread(j, THREADS, startSignal, doneSignal, log);
    // start threads
    for (int j = 0; j < THREADS; j++)
      thread[j].start();
    // do something
    for (int i = 0; i < THREADS; i++)
      log[i] = START;
    // give threads permission to start
    startSignal.countDown();
    // wait for threads to complete
    try {
      doneSignal.await();
    } catch (InterruptedException e) {
      fail("interrupted exception");
    }
    for (int i = 0; i < THREADS; i++)
      if (log[i] != STOP)
        fail("found bad value: " + log[i]);
  }
\end{lstlisting}
\hfill
\par
So, at start the main thread creates 8 threads and make them run. When these
threads start, the are put to wait immediately. Then the main thread signals
them to actually start and it puts himself to wait.
\par
\hfill
\begin{lstlisting}[style=numbers]
  public class TestThread extends Thread {
    int threads;
    CountDownLatch startSignal;
    CountDownLatch doneSignal;
    int[] log;
    int index;
    public TestThread(int index, int threads,
        CountDownLatch startSignal, CountDownLatch doneSignal, int[] log) {
      this.threads = threads;
      this.log = log;
      this.index = index;
      this.startSignal = startSignal;
      this.doneSignal = doneSignal;
    }
    public void run() {
      ThreadID.set(index);
      try {
        startSignal.await(); // wait for permission to start
      } catch (InterruptedException ex) {
        ex.printStackTrace();
      }
      if (log[index] != START) {
        System.out.format("%d\tError expected %d found %d at %d\n", START,
log[index], index);
      }
      log[index] = STOP;
      doneSignal.countDown();  // notify driver we are done
    }
  }
\end{lstlisting}
\hfill
\par
Then each of the 8 threads will run and do their stuff. At the end, each of  the
threads calls \textit{countDown()} for the doneSignal. When the counter for this
latch reaches 0, the program ends.
\par
%%%%%%%%%%%%%%%%%%%%%%%%%%%%%%%%%%
\subsection{Sample Results}
\par
For this test, we saw that in every try, it always passed. Here is the output:
\par
\begin{verbatim}
[oraadm@gdlaa008 Monitor]$ junit monitor.CountDownLatchTest
.Testing 8 threads

Time: 0.01

OK (1 test)
\end{verbatim}
\par
%%%%%%%%%%%%%%%%%%%%%%%%%%%%%%%%%%
\subsection{Interpretation}
This experiment showed us the way in which Monitors are used. We saw that we did
not need to continuously poll for a flag to acquire a lock. Instead, the monitor
sent the threads a signal to indicate them that a particular condition has been
met. In our particular test, these signals indicated: (1) that the threads can
start; and (2) that the threads must stop.
% --------------------------- %
% Count Down Latch End
% --------------------------- %
