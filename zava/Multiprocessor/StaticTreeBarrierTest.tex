% --------------------------- %
% StaticTreeBarrierTest Start
% --------------------------- %
\section{\textbf{StaticTreeBarrierTest}}
%%%%%%%%%%%%%%%%%%%%%%%%%%%%%%%%%%
\subsection{Particular Case}
\par
Here we want to implement a Barrier algorithm that does not suffer of either
contention or excessive communication cost. 
\par
Just as a reminder, a barrier is a way of forcing asynchronous threads to act
almost as if they were synchronous by adding synchronization points.
\par
%%%%%%%%%%%%%%%%%%%%%%%%%%%%%%%%%%
\subsection{Solution}
\par
The solution for this problem is given by the Static Tree Barrier. The way it
works is: each thread is assigned to a node in a tree. A thread at a node has to
wait until all nodes bellow it in the tree have finished and then let its parent
know that he is done. After that, it keeps spinning since the root of the tree
signals a global sense bit change. So, one the root knows that all its children
are done, this bit is flipped.
\par
Here is the interesting code. Notice that the three is constructed within an
array.
\par
When a thread finishes its work, it calls the \textit{await()} method below. Notice
that it sits there till the sense is changed.
\par
\hfill
\begin{lstlisting}[style=numbers]
    public void await() {
      boolean mySense = threadSense.get();
      while (childCount.get() > 0) {};  // spin until children done
      childCount.set(children);         // prepare for next round
      if (parent != null) { // not root?
        parent.childDone();           // indicate child subtree completion
        while (sense != mySense) {}; // wait for global sense to change
      } else {
        sense = !sense;   // am root: toggle global sense
      }
      threadSense.set(!mySense); // toggle sense
    }
\end{lstlisting}
\hfill
\par
%%%%%%%%%%%%%%%%%%%%%%%%%%%%%%%%%%
\subsection{Experiment Description}
\par
Two test cases are provided:
\par
\begin{itemize}
\item testBarrier. Creates a tree barrier of radix 2. 7 threads are created,
meaning that we will have 3 levels in the tree. What is thread does is confirm
that all siblings are executing at the same time. 
\item testBarrier3. Does the same bit using a tree barrier of radix 3. This
means that we will have only 2 leves in the tree.
\end{itemize}
\par
%%%%%%%%%%%%%%%%%%%%%%%%%%%%%%%%%%
\subsection{Sample Results}
\par
In our 24-cores machine, we got the following result:
\par
\hfill
\begin{verbatim}
[oraadm@gdlaa008 Barrier]$ junit barrier.StaticTreeBarrierTest
.Testing 13 threads, 1 rounds, radix 3
radix 3 done
.Testing 7 threads, 1 rounds, radix 2
radix 2 done

Time: 0.015

OK (2 tests)
\end{verbatim}
\hfill
\par
However, some people in the group mentioned that they found a hang in this test
which was fixed by adding the volatile qualifier to the global sense attribute.
\par
So, in this exercise we learnt another way of implementing a barrier mechanism.
\par
%%%%%%%%%%%%%%%%%%%%%%%%%%%%%%%%%%
% --------------------------- %
% StaticTreeBarrierTest End
% --------------------------- %
