% --------------------------- %
% SimpleTreeTest Start
% --------------------------- %
\section{\textbf{SimpleTreeTest Test}}
%%%%%%%%%%%%%%%%%%%%%%%%%%%%%%%%%%
\subsection{Particular Case}
\par
The problem we want to solve now is how we can implement a bounded priority
queue that can be accessed concurrently by multiple threads.
\par
Remember that priority queue is said to be \textit{bounded} when the range of
the priorities are taken from the range $0,...,m-1$.
\par
%%%%%%%%%%%%%%%%%%%%%%%%%%%%%%%%%%
\subsection{Solution}
\par
One possible solution to this problem is given by the \textit{Tree-Based Bounded
Priority Queue} data structure. This structure is esentially a binary tree where
the leaf nodes have a bin holding items of priority $i$. Internal nodes mantain
a shared counter that indicates the number of elements stored to the left of
a given node.
\par
The interesting methods in this data structure are the \textit{add()} and
\textit{removeMin()}, so let us talk about them.
\par
The \textit{add()} method first takes the bin associated with the given
priority. It puts the item into that bin and then starts increasing the counter
of the parent nodes all the way to the root. Note that it only increases the
counter if the current node happens to be a left child of the parent.
\par
\hfill
\begin{lstlisting}[style=numbers]
  /**
   *  add item to priority queue
   * @param item new item
   * @param priority item''s priority
   */
  public void add(T item, int priority) {
    TreeNode node = leaves.get(priority);
    node.bin.put(item);
    while(node != root) {
      TreeNode parent = node.parent;
      if (node == parent.left) { // increment if ascending from left
        parent.counter.getAndIncrement();
      }
      node = parent;
    }
  }
\end{lstlisting}
\hfill
\par
The \textit{removeMin()} method has to retrieve the item with the lowest
priority. In order to do that, the method starts from the root and traverses the
tree. If the current counter of the inner node is $0$, then it knows that the
min must be on the left. Otherwise, it goes through the right child. 
\par
\hfill
\begin{lstlisting}[style=numbers]
  public T removeMin() {
    TreeNode node = root;
    while(!node.isLeaf()) {
      if (node.counter.getAndDecrement() > 0 ) {
        node = node.left;
      } else {
        node = node.right;
      }
    }
    return node.bin.get(); // if null pqueue is empty
  }
\end{lstlisting}
\hfill
\par
%%%%%%%%%%%%%%%%%%%%%%%%%%%%%%%%%%
\subsection{Experiment Description}
\par
Three test cases were provided to exercise this code:
\begin{itemize}
\item testAdd. This test adds elements with random priority to the priority
queue one by one. At the end, it extracts each element and it must be the case
that the previously removed element had a lower priority than the current
element. 
\item testParallelAdd. This test spawns 8 threads. Each of the test adds 8
elements to the queue concurrently. At the end, it performs the same checking as
in the previous test case. Notice that each thread is in charge of adding
elements from a fixed range in ascending order.
\item testParallelBoth. Does the same as the previous test case but also removes
elements in parallel. Each remover thread will perform the same checking. Since
adders add elements in ascending order, it is garanteed that any removal from
any thread will retrieve an element with a lower priority than the next one.
\end{itemize}
\par
%%%%%%%%%%%%%%%%%%%%%%%%%%%%%%%%%%
\subsection{Sample Results}
\par
\par
%%%%%%%%%%%%%%%%%%%%%%%%%%%%%%%%%%
\subsection{Interpretation}
\par
\par
%%%%%%%%%%%%%%%%%%%%%%%%%%%%%%%%%%
\subsection{Proposed solution}
\par
\par
%%%%%%%%%%%%%%%%%%%%%%%%%%%%%%%%%%
% --------------------------- %
% SimpleTreeTest End
% --------------------------- %
