% --------------------------- %
% Bakery Start
% --------------------------- %
\section{\textbf{Bakery Test}}
%%%%%%%%%%%%%%%%%%%%%%%%%%%%%%%%%%
\subsection{Particular Case}
\par
In this experiment, we are dealing with the problem of mutual exclusion with a
number of participants $>2$.
\par
We have already seen that Peterson Algorithm works for two threads. It guarantees
mutual exclusion and is starvation free. Therefore, the algorithm is deadlock
free. Now let us focus in an algorithm that can be used to coordinate more
threads.
\par
%%%%%%%%%%%%%%%%%%%%%%%%%%%%%%%%%%
\subsection{Solution}
\par
The algorithm that we will play with is called \textit{Bakery}. Its name comes from the
fact that it is similar to the protocol used in bakeries where one enters the
store and picks a number that indicates the order in which each client will be
attended. 
\par
In the algorithm, the fact of entering the store is done by setting a flag.
After that, the thread calculates the next number in the machine. 
\par
The algorithm then chooses the next number to be attended. When that happens,
the thread is allowed to enter the critical section.
\par
One thing that we ought to mention is that threads can have the same number
assigned. If that is the case, the next in the line is decided using a
lexicographical ordering of the threads ids.
\par
Here we show the interesting methods used in this algorithm:
\par
\hfill
\begin{lstlisting}[style=numbers]
  public void lock() {
    int me = ThreadID.get();
    flag[me]  = true;
    int max = Label.max(label);
    label[me] = new Label(max + 1);
    while (conflict(me)) {};  // spin
  }

  public void unlock() {
    flag[ThreadID.get()] = false;
  }
  
  private boolean conflict(int me) {
    for (int i = 0; i < label.length; i++) {
      if (i != me && flag[i] && label[me].compareTo(label[i]) < 0) {
        return true;
      }
    }
    return false;
  }

  static class Label implements Comparable<Label> {
    int counter;
    int id;
    Label() {
      counter = 0;
      id = ThreadID.get();
    }
    Label(int c) {
      counter = c;
      id = ThreadID.get();
    }
    static int max(Label[] labels) {
      int c = 0;
      for (Label label : labels) {
        c = Math.max(c, label.counter);
      }
      return c;
    }
    
    public int compareTo(Bakery.Label other) {
      if (this.counter < other.counter
          || (this.counter == other.counter && this.id < other.id)) {
        return -1;
      } else if (this.counter > other.counter) {
        return 1;
      } else {
        return 0;
      }
    }
  }
\end{lstlisting}
\hfill
\par
In the \textit{lock()} method, we observe that the thread signals its intention
of accessing the critical section by setting its flag to true. After that, the
thread picks its number and then it sits and waits for its turn. The thread has
to wait if another thread which announced its intention to use the resource has
a smaller label. 
\par
%%%%%%%%%%%%%%%%%%%%%%%%%%%%%%%%%%
\subsection{Experiment Description}
\par
Now let us explain how the experiments work. In this case we have 8 threads
increasing a counter from 0 to 1024. Each thread will increase the counter 128
times. At the end, the counter must remain in 1024. If that is not the case,
then there is a problem with the mutual exclusion algorithm.
\par
%%%%%%%%%%%%%%%%%%%%%%%%%%%%%%%%%%
\subsection{Sample Results}
\par
In this exercise we observed that in the machine that we have been using, the
test for the Bakery algorithm fails consistently with the following error:
\par
\begin{verbatim}
.F
Time: 0.019
There was 1 failure:
1) testParallel(mutex.BakeryTest)junit.framework.AssertionFailedError:
expected:<1019> but was:<1024>
        at mutex.BakeryTest.testParallel(BakeryTest.java:47)
        at sun.reflect.NativeMethodAccessorImpl.invoke0(Native Method)
        at
sun.reflect.NativeMethodAccessorImpl.invoke(NativeMethodAccessorImpl.java:57)
        at
sun.reflect.DelegatingMethodAccessorImpl.invoke(DelegatingMethodAccessorImpl.java:43)

FAILURES!!!
Tests run: 1,  Failures: 1,  Errors: 0
\end{verbatim}
\par
%%%%%%%%%%%%%%%%%%%%%%%%%%%%%%%%%%
\subsection{Interpretation}
\par
Let us elaborate on the meaning of the error that we got. The assertion failure
says that at the end of the algorithm, our counter didn't reach the expected
1024. This problem might indicate that our counter wasn't actually being
protected by the lock because it looks like multiple threads found the counter
with a value of $i$ and increased it to $i+1$. This is wrong.
\par
After investigating the problem in more detail, we found out that the problem in
this code has to do with the way the \textit{label} array is accessed and
modified. For example, in the lock method, multiple threads can reach the point
where the max is calculated and hence these mutiple threads can take the same
turn. The algorithm takes care of this problem by introducing a lexicographical
ordering in the comparison: if two threads have the same number, the decision is
made based on the $threadID$.
\par
Also, as explained in the textbook, we cannot be sure that we have a correct
memory consistency: it is possible that the processor modified the order in
which operations in the \textit{lock()} method are executed. It is also possible
that the processor decided to execute an instruction before another.
\par
%%%%%%%%%%%%%%%%%%%%%%%%%%%%%%%%%%
\subsection{Proposed solution}
\par
The following fragment of code shows one possible way of fixing the code:
\par
\hfill
\begin{lstlisting}[style=numbers]
  public void lock() {
    int me = ThreadID.get();
    flag[me]  = true;
    synchronized (label){
      int max = Label.max(label);
      label[me] = new Label(max + 1);
      while (conflict(me)) {};  // spin
    }
  }
\end{lstlisting}
\hfill
Note that this solution is not quite satisfactory as we are basically using a
Java mechanism to ensure synchronized access in an algorithm that promises to
provide synchronized access.
\par
In any case, let us show the output of the program after this fix:
\begin{verbatim}
.
Time: 0.015

OK (1 test)
\end{verbatim}
\hfill
\par
After the fix the results in the proposed system were all OK. In all cases, the
threads were able to cooperate to increase the counter to 1024.
\par
One thing that is worth mentioning is that unlike the Peterson Algorithm, for
example, the Bakery algorithm is able to coordinate multiple threads (at least
in theory). This fact was shown in the test case were 8 threads were coordinated
to increase the counter.
%%%%%%%%%%%%%%%%%%%%%%%%%%%%%%%%%%
% --------------------------- %
% Bakery End
% --------------------------- %
