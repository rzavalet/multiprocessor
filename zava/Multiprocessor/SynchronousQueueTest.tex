% --------------------------- %
% SynchronousQueueTest Start
% --------------------------- %
\section{\textbf{SynchronousQueueTest}}
%%%%%%%%%%%%%%%%%%%%%%%%%%%%%%%%%%
\subsection{Particular Case}
\par
In this exercise we are dealing again with a first-in-first-out (FIFO) data
structure, i.e. a Queue. The difference with respect of the previous examples is
that in this case we have multiple producers and multiple consumers
\textit{rendezvous}ing with one another. In other words, a producer that puts
an item in the queue blocks until that item is removed by a consumer.
\par
%%%%%%%%%%%%%%%%%%%%%%%%%%%%%%%%%%
\subsection{Solution}
\par
The following is a naive implementation of a Synchronous queue. Notice that it
contains the following fields:
\par
\hfill
\begin{lstlisting}[style=numbers]
  public class SynchronousQueue<T> {
    T item = null;
    boolean enqueuing;
    Lock lock;
    Condition condition;
	...
	}
\end{lstlisting}
\hfill
\par
Let us talk about them:
\begin{itemize}
\item item. The first item in the queue (the one to be dequeued)
\item enqueuing. Used to sync the enqueuers
\item lock. Used for mutual exclusion
\item condition. Used to block partial methods
\end{itemize}
\par
Now, let us talk about how the enqueue and the dequeue actually work:
\par
\hfill
\begin{lstlisting}[style=numbers]
    public T deq() throws InterruptedException {
      lock.lock();
      try {
        while (item == null) {
          condition.await();
        }
        T t = item;
        item = null;
        condition.signalAll();
        return t;
      } finally {
        lock.unlock();
      }
    }

    public void enq(T value) throws InterruptedException {
      if (value == null) throw new NullPointerException();
      try {
        while (enqueuing) { // another enqueuer's turn
          condition.await();
        }
        enqueuing = true; // my turn starts
        item = value;
        condition.signalAll();
        while (item != null) {
          condition.await();
        }
        enqueuing = false;  // my turn ends
        condition.signalAll();
      } finally {
        lock.unlock();
      }
    }
  }
\end{lstlisting}
\hfill
\par
The enqueue method first has to check whether another enqueuer is enqueueng an
element. If that is the case, then the enqueuer has to wait till the enqueuer
is paired with a dequeuer. After that, it will set the flag \textit{enqueuing}
to true, sets the item and signals any other thread that is waiting for this
one to finish its enqueue. 
\par
Once the enqueuer has put its value in the queue, it waits till someone
consumes its value (\textit{item} becomes \textit{null}). Then the enqueuer
sets the \textit{enqueuing} flag back to false and finishes.
\par
On the other hand, the dequeuer threads after acquiring the lock, they have to
wait till an element is put in the queue. When it finds this is true, then they
take the item and signal other threads to indicate they have finished
dequeuing.
\par
%%%%%%%%%%%%%%%%%%%%%%%%%%%%%%%%%%
\subsection{Experiment Description}
\par
This exercise consists of only one test and it works as follow:
\par
It creates $2*8$ threads. Each of the threads will enqueue $512/8$ elements to
the queue. At the same time, $8$ threads will try to dequeue the elements. When
dequeueing, the test will put a value of \textit{true} in the slot
$arr[value]$. The test expects that each slot should not be set to
\textit{true} twice.
\par
%%%%%%%%%%%%%%%%%%%%%%%%%%%%%%%%%%
\subsection{Sample Results}
\par
First, we need to mention that we had a problem compiling the test cases because
we needed to add a try-catch block around the \textit{enq()} \textit{deq()}
call like this:
\par
\hfill
\begin{lstlisting}[style=numbers]
        try{
          int value = (Integer)instance.deq();
          if (map[value]) {
            fail("DeqThread: duplicate pop");
          }
          map[value] = true;                                                                                                                                                         
        } catch(InterruptedException ie) {
          System.out.println("Exception" + ie);
        }
\end{lstlisting}
\hfill
\par
After that, we noticed that they consistently fails with the following error:
\par
\hfill
\begin{verbatim}
[oraadm@gdlaa008 Queue]$ junit queue.SynchronousQueueTest
.parallel both
Exception in thread "Thread-10" Exception in thread "Thread-6" Exception in thread "Thread-2" Exception in th
read "Thread-0" Exception in thread "Thread-8" Exception in thread "Thread-4" Exception in thread "Thread-14"
 Exception in thread "Thread-12" java.lang.IllegalMonitorStateException
        at java.util.concurrent.locks.ReentrantLock$Sync.tryRelease(ReentrantLock.java:155)
        at java.util.concurrent.locks.AbstractQueuedSynchronizer.release(AbstractQueuedSynchronizer.java:1260)
        at java.util.concurrent.locks.ReentrantLock.unlock(ReentrantLock.java:460)
        at queue.SynchronousQueue.enq(SynchronousQueue.java:63)
        at queue.SynchronousQueueTest$EnqThread.run(SynchronousQueueTest.java:61)
java.lang.IllegalMonitorStateException
        at java.util.concurrent.locks.ReentrantLock$Sync.tryRelease(ReentrantLock.java:155)
        at java.util.concurrent.locks.AbstractQueuedSynchronizer.release(AbstractQueuedSynchronizer.java:1260)
        at java.util.concurrent.locks.ReentrantLock.unlock(ReentrantLock.java:460)
        at queue.SynchronousQueue.enq(SynchronousQueue.java:63)
        at queue.SynchronousQueueTest$EnqThread.run(SynchronousQueueTest.java:61)
java.lang.IllegalMonitorStateException
        at java.util.concurrent.locks.ReentrantLock$Sync.tryRelease(ReentrantLock.java:155)
        at java.util.concurrent.locks.AbstractQueuedSynchronizer.release(AbstractQueuedSynchronizer.java:1260)
        at java.util.concurrent.locks.ReentrantLock.unlock(ReentrantLock.java:460)
        at queue.SynchronousQueue.enq(SynchronousQueue.java:63)
        at queue.SynchronousQueueTest$EnqThread.run(SynchronousQueueTest.java:61)
java.lang.IllegalMonitorStateException
        at java.util.concurrent.locks.ReentrantLock$Sync.tryRelease(ReentrantLock.java:155)
        at java.util.concurrent.locks.AbstractQueuedSynchronizer.release(AbstractQueuedSynchronizer.java:1260)
        at java.util.concurrent.locks.ReentrantLock.unlock(ReentrantLock.java:460)
        at queue.SynchronousQueue.enq(SynchronousQueue.java:63)
        at queue.SynchronousQueueTest$EnqThread.run(SynchronousQueueTest.java:61)
java.lang.IllegalMonitorStateException
        at java.util.concurrent.locks.ReentrantLock$Sync.tryRelease(ReentrantLock.java:155)
        at java.util.concurrent.locks.AbstractQueuedSynchronizer.release(AbstractQueuedSynchronizer.java:1260)
        at java.util.concurrent.locks.ReentrantLock.unlock(ReentrantLock.java:460)
        at queue.SynchronousQueue.enq(SynchronousQueue.java:63)
        at queue.SynchronousQueueTest$EnqThread.run(SynchronousQueueTest.java:61)
java.lang.IllegalMonitorStateException
        at java.util.concurrent.locks.ReentrantLock$Sync.tryRelease(ReentrantLock.java:155)
        at java.util.concurrent.locks.AbstractQueuedSynchronizer.release(AbstractQueuedSynchronizer.java:1260)
        at java.util.concurrent.locks.ReentrantLock.unlock(ReentrantLock.java:460)
        at queue.SynchronousQueue.enq(SynchronousQueue.java:63)
        at queue.SynchronousQueueTest$EnqThread.run(SynchronousQueueTest.java:61)
java.lang.IllegalMonitorStateException
        at java.util.concurrent.locks.ReentrantLock$Sync.tryRelease(ReentrantLock.java:155)
        at java.util.concurrent.locks.AbstractQueuedSynchronizer.release(AbstractQueuedSynchronizer.java:1260)
        at java.util.concurrent.locks.ReentrantLock.unlock(ReentrantLock.java:460)
        at queue.SynchronousQueue.enq(SynchronousQueue.java:63)
        at queue.SynchronousQueueTest$EnqThread.run(SynchronousQueueTest.java:61)
java.lang.IllegalMonitorStateException
        at java.util.concurrent.locks.ReentrantLock$Sync.tryRelease(ReentrantLock.java:155)
        at java.util.concurrent.locks.AbstractQueuedSynchronizer.release(AbstractQueuedSynchronizer.java:1260)
        at java.util.concurrent.locks.ReentrantLock.unlock(ReentrantLock.java:460)
        at queue.SynchronousQueue.enq(SynchronousQueue.java:63)
        at queue.SynchronousQueueTest$EnqThread.run(SynchronousQueueTest.java:61)
\end{verbatim}
\par
The problem is a bad use of the monitor API. Here is the corrected code:
\par
\hfill
\begin{lstlisting}[style=numbers]
  public void enq(T value) throws InterruptedException {
    if (value == null) throw new NullPointerException();
    lock.lock();
    try {
      while (enqueuing) { // another enqueuer's turn
        condition.await();
      }
      enqueuing = true; // my turn starts
      item = value;
      condition.signalAll();
      while (item != null) {
        condition.await();
      }
      enqueuing = false;  // my turn ends
      condition.signalAll();
    } catch(Exception e) {
      System.out.println("Exception: " + e);
    } finally {
      lock.unlock();
    }
  }    
\end{lstlisting}
\hfill
\par
After this fix, here is the result:
\par
\hfill
\begin{verbatim}
[oraadm@gdlaa008 Queue]$ junit queue.SynchronousQueueTest
.parallel both

Time: 0.167

OK (1 test)
\end{verbatim}
\hfill
\par
%%%%%%%%%%%%%%%%%%%%%%%%%%%%%%%%%%
% --------------------------- %
% SynchronousQueueTest End
% --------------------------- %
