% --------------------------- %
% CoarseHashSetTest Start
% --------------------------- %
\section{\textbf{CoarseHashSetTest Test}}
%%%%%%%%%%%%%%%%%%%%%%%%%%%%%%%%%%
\subsection{Particular Case}
\par
The problem we are trying to solve now is that of building a concurrent Hash
data structure. In the particular case described in the book, the hash data
structure is used to construct a representation of a set ADT. Hence, we ought
to provide three operations:
\par
\begin{enumerate}
\item add. Aggregate a new element to the set
\item remove. Remove a given element from the set
\item contains. Determine whether a given element is already in the set
\end{enumerate}
\par
%%%%%%%%%%%%%%%%%%%%%%%%%%%%%%%%%%
\subsection{Solution}
\par
Here we present a first approach to solving the problem stated above. We do so
by providing a coarse version. As in other coarse data structures, the problem
with this implementation is its poor performance in parallel environments.
\par
Let us briefly describe how this implementation works. First we should mention
that we have a base abstract class that implements the three operations
mentioned previously. We extend this base class by providing four methods:
\par
\begin{itemize}
\item resize. Resizes the hash based on a policy. In this implementation, it
simply doubles the size of t he hash.
\item acquire. Acquires a lock on the hash.
\item release. Releases the lock.
\item policy. Returns true if each bucket contains more than 4 elements.
\end{itemize}
\par
\hfill
\begin{lstlisting}[style=numbers]
  /**
   * double the set size
   */
  public void resize() {
    int oldCapacity = table.length;
    lock.lock();
    try {
      if (oldCapacity != table.length) {
        return; // someone beat us to it
      }
      int newCapacity  = 2 * oldCapacity;
      List<T>[] oldTable = table;
      table = (List<T>[]) new List[newCapacity];
      for (int i = 0; i < newCapacity; i++)
        table[i] = new ArrayList<T>();
      for (List<T> bucket : oldTable) {
        for (T x : bucket) {
          int myBucket = Math.abs(x.hashCode() % table.length);
          table[myBucket].add(x);
        }
      }
    } finally {
      lock.unlock();
    }
  }
  /**
   * Synchronize before adding, removing, or testing for item
   * @param x item involved
   */
  public final void acquire(T x) {
    lock.lock();
  }
  /**
   * synchronize after adding, removing, or testing for item
   * @param x item involved
   */
  public void release(T x) {
    lock.unlock();
  }
  public boolean policy() {
    return size / table.length > 4;
  }
\end{lstlisting}
\hfill
\par
This implementation is said to be coarse because any operation on the hash locks the entire data structure. 
\par
%%%%%%%%%%%%%%%%%%%%%%%%%%%%%%%%%%
\subsection{Experiment Description}
\par
As in other experiments, here we have four test cases:
\begin{itemize}
\item testSequential. The master thread adds 512 elements to the set, then it
verifies that these elements were actually added and finally removes them.
\item testParallelEnq. It creates 8 \textit{adders} threads. Each of them will
add 512 elements to the set. When all of them are done, the master thread
checks that all the elements were actually added and finally removes them.
\item testParallelRemove. In this test case, the master thread adds 512
elements to the set and verifies that the elements were actually added. After
that, 8 \textit{remover} threads are spawned. Each of them will remove $512/8$
elements. 
\item testParallelBoth. This test is a mix of the two above. 8
\textit{removers} and 8 \textit{adders} are spawned. Removers should remove
elements added by the adders.
\end{itemize}
\par
%%%%%%%%%%%%%%%%%%%%%%%%%%%%%%%%%%
\subsection{Sample Results and Interpretation}
\par
Here is the result of the execution:
\par
\hfill
\begin{verbatim}
[oraadm@gdlaa008 Hash]$ junit hash.CoarseHashSetTest
.sequential add, contains, and remove
.parallel both
Exception in thread "Thread-7" Exception in thread "Thread-9"
junit.framework.AssertionFailedError: DeqThread: duplicate pop
        at junit.framework.Assert.fail(Assert.java:57)
        at junit.framework.TestCase.fail(TestCase.java:227)
        at hash.CoarseHashSetTest$RemoveThread.run(CoarseHashSetTest.java:143)
Exception in thread "Thread-13" junit.framework.AssertionFailedError: DeqThread:
duplicate pop
        at junit.framework.Assert.fail(Assert.java:57)
        at junit.framework.TestCase.fail(TestCase.java:227)
        at hash.CoarseHashSetTest$RemoveThread.run(CoarseHashSetTest.java:143)
junit.framework.AssertionFailedError: DeqThread: duplicate pop
        at junit.framework.Assert.fail(Assert.java:57)
        at junit.framework.TestCase.fail(TestCase.java:227)
        at hash.CoarseHashSetTest$RemoveThread.run(CoarseHashSetTest.java:143)
Exception in thread "Thread-15" junit.framework.AssertionFailedError: DeqThread:
duplicate pop
        at junit.framework.Assert.fail(Assert.java:57)
        at junit.framework.TestCase.fail(TestCase.java:227)
        at hash.CoarseHashSetTest$RemoveThread.run(CoarseHashSetTest.java:143)
.parallel add
.parallel remove

Time: 0.043

OK (4 tests)
\end{verbatim}
\hfill
\par
The test complains saying that when we add and remove elements from
the hash table concurrently the remover threads are removing an element twice.
However, we found that this is not necesarily correct because the test simply
assumes that if a remove fails, it is because we have repeated elements. 
\par
If we look into the code carefully, we can see that a remove operation can also
fail if it is the case that it is trying to remove an element that has not been
inserted yet. 
\par
So, in order to make this test work, we have to garantee that removers try to
remove elements that have been actually added to the set. We leave this
enhancement as a future work.
\par
%%%%%%%%%%%%%%%%%%%%%%%%%%%%%%%%%%
% --------------------------- %
% CoarseHashSetTest End
% --------------------------- %
