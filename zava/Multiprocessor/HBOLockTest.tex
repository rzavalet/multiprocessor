% --------------------------- %
% Hierarchical Backoff Lock Start
% --------------------------- %
\section{\textbf{Hierarchical Backoff Lock}}
%%%%%%%%%%%%%%%%%%%%%%%%%%%%%%%%%%
\subsection{Particular Case}
\par
In the following exercises we will be dealing with the idea of Hierarchical
locks. The motivation of this type of locks are the architecture of modern
cache-coherent computers where processors are organized as clusters.
Communication between processors is then categorized in intra-cluster
communication and inter-cluster communication. The former is faster and cheaper
than the latter. 
\par
Given the following background, the idea is  to build locking algorithms that
work efficiently in these architectures. The problem is then: \textit{how can we
create locking algorithm for this type of systems?}
\par
%%%%%%%%%%%%%%%%%%%%%%%%%%%%%%%%%%
\subsection{Solution}
\par
The first algorithm that we are going to study is the Hierarchical Back-off Lock.
This algorithm uses the \textit{test-and-test-and-set} lock that we presented
before. If a thread from a cluster releases a lock, then it is more likely that
another thread from the same cluster acquires it. This strategy reduces the
overall time needed to switch the lock ownership. Let us take a quick look at the
source code in java:
\par
\hfill
\begin{lstlisting}[style=numbers]
  public void lock() {
    int myCluster = ThreadID.getCluster();
    Backoff localBackoff =
        new Backoff(LOCAL_MIN_DELAY, LOCAL_MAX_DELAY);
    Backoff remoteBackoff =
        new Backoff(REMOTE_MIN_DELAY, REMOTE_MAX_DELAY);
    while (true) {
      if (state.compareAndSet(FREE, myCluster)) {
        return;
      }   
      int lockState = state.get();
      try {
        if (lockState == myCluster) {
          localBackoff.backoff();
        } else {
          remoteBackoff.backoff();
        }   
      } catch (InterruptedException ex) {
      }   
    }   
  }
  
  public void unlock() {
    state.set(FREE);
  }
\end{lstlisting}
\hfill
\par
For locking, we create two back-off objects which are the ones that have a method to
perform the back-off for a random period. Notice that we distinguish between two
cases: a local back-off and a remote back-off. The delays are as follows:
\par
\hfill
\begin{lstlisting}[style=numbers]
  private static final int LOCAL_MIN_DELAY = 8;
  private static final int LOCAL_MAX_DELAY = 256;
  private static final int REMOTE_MIN_DELAY = 256;
  private static final int REMOTE_MAX_DELAY = 1024;
\end{lstlisting}
\hfill
\par
A remote delay is longer than a local delay. Then we perform a
\textit{compareAndSet()} operation. If the state of the lock is free, then we
set its value to the value of our cluster. That indicates that we are holding
the lock now.
\par
If the state was not free, then we perform a back-off. If the lock is being hold
by another thread in the cluster, then we do a short back-off. Otherwise, we do a
long back-off.
\par
The unlock method simply sets the state to free.
%%%%%%%%%%%%%%%%%%%%%%%%%%%%%%%%%%
\subsection{Experiment Description}
\par
To demonstrate that this implementation works, the test that was provided does the following:
\begin{enumerate}
\item Initiate a shared counter with a value of $0$
\item Start $8$ threads
\item Each thread has to increment the counter by one $1024$ times
\item At the end, the counter must hold a value of $8*1024$
\end{enumerate}
\par
%%%%%%%%%%%%%%%%%%%%%%%%%%%%%%%%%%
\subsection{Sample Results}
\par
For this test, we saw that in every try, it always passed. Here is the output:
\par
\begin{verbatim}
[oraadm@gdlaa008 Spin]$ junit spin.HBOLockTest
.
Time: 0.262

OK (1 test)
\end{verbatim}
\par
%%%%%%%%%%%%%%%%%%%%%%%%%%%%%%%%%%
\subsection{Interpretation}
\par
In this experiment we discovered how to implement a locking algorithm in a
cache-coherent architecture by performing back-offs of different delays depending
on whether the thread holding the lock is in the same or in a different cluster
of CPUs. 
\par
One problem that we can foresee though, is that this algorithm could cause
starvation of remote threads because the algorithms give more priority to local
threads. This is not a desirable property. In the next algorithm we shall see a
way of overcoming this problem
% --------------------------- %
% Hierarchical Backoff Lock End
% --------------------------- %
